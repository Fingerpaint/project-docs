\chapter{Coding Style}
In our project, we will use a coding style that is based on and resembles closely the ESA coding standard \cite{esa-coding-style} and the Oracle Java coding conventions \cite{oracle-java-coding-style}. We will list all relevant conventions we made in this chapter. To keep this document concise, we will only list conventions that we find important. Other things that are left unspecified are done as described in the Java coding style. Note that the structure of this chapter is identical to the structure used in the referenced ESA document.

\section{Readability Standards}
In this section, general conventions that ensure the code is \emph{readable} in general are described. This means that if a programmer scans through the code, he or she can understand the general idea of that part of the program. To achieve this, we will:

\begin{itemize}
	\item Use a consistent writing style, namely British-English spelling where applicable.
	\item Use a consistent naming scheme, namely the one described in the Oracle Java coding conventions. For clarity, we quote those conventions here.
		\begin{itemize}
			\item Class names should be nouns, in mixed case with the first letter of each internal word capitalized. Try to keep your class names simple and descriptive. Use whole words - avoid acronyms and abbreviations (unless the abbreviation is much more widely used than the long form, such as URL or HTML).
		\end{itemize}
\end{itemize}