In this chapter, we will define a couple of conventions for committing and commmit messages. Developers should stick to these conventions as good as possible, to keep the history of the repositories easy-to-read and version management simple.

Committing changes is something that developers do a lot, so the conventions are simple: a developer does not want to do something complex a lot. Nobody does. Therefore, we have come to the following list of recommendations:

\begin{itemize}
	\item Make every relevant change to a repository a single commit. Do not combine multiple changes in a commit. This makes reverting changes easier.
	\paragraph{Example} A commit wherein both a document and the general layout are changed, is not okay. These should be two separate commits.
	
	\item Always write a concise yet descriptive commit message for every commit. This makes it easier to read through the commit history and find relevant commits.
	
	\item Refrain from committing binary files. These files will change a lot (probably) with every change, which does not work well in general. A separate folder should be created containing the latest stable version of all documentation in PDF-form. The actual versioning and committing should be done in textual files. We use \LaTeX\ to generate all \projectname documentation.
\end{itemize}