Internally, we will do reviews of all documents. To keep a history record of these and give an author of a document a good overview of what comments there are on his/her document, we have decided to formalize reviews in a simple format. This will make all reviews consistent and will also make updating a document in response to a review easier.

A review is a list of remarks, where each remark consists of three things:
\begin{enumerate}
	\item One of the following categories, in which the remark falls:
		\begin{itemize}
			\item Question;
			\item Typo (for ``typographical error'');
			\item Incorrect (content);
			\item Missing (content);
			\item Structure / layout;
			\item Inconsistent;
			\item Other.
		\end{itemize}
	\item Reference to chapter/section/subsection/paragraph/page to indicate where the subject of the remark is located in the document. For clarity, a (brief) quote may be added here to make finding the part of the text that the remark is about easier.
	\item The actual remark. This should be a concise yet complete description of what the problem(s) is (are) according to the reviewer.
\end{enumerate}

Furthermore, every review should contain the name of the reviewer, so that it is clear to who the author can go with questions about the review. Note however that the review should be clear enough, so that questions about the review should not be necessary.
