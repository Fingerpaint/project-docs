\chapter{General description}
This chapter describes general aspects of the application as requested by the client.

\section{Product perspective}
The aim of this project is to deliver an application that allows the user to easily visualize the mixing of fluids. A user interface should be provided to specify initial details, after which output (including intermediate results) should be shown on the screen. All of this should be possible using an easy to use, attractive interface on a mobile device.

A similar project was initiated around eleven years ago. The result of this project was a \textsc{Matlab} implementation that achieved a similar goal as our project. However, its user interface is by now outdated, and it is impossible to comfortably use this solution on a mobile device. Part of this original solution was a \textsc{Fortran} implementation for computing the necessary matrices. This implementation is still available, and we are to use it as a black box to compute the matrices we need for our solution. The client has several matrices, all denoting various kinds of mixers, which we can use for our implementation.

\section{General capabilities}
The system should be able to simulate the flow and mixing of a number of fluids, given some constraints such as movement of the walls and initial concentration of the fluids. The main user interface should be run on a mobile device, such as an iPhone. On this device, three important aspects of a problem should be specified. The first parameter to be specified is the initial concentration of the fluids in the mixer, which should be indicated by tapping on and dragging over the screen. The second is the protocol for moving the mixer and the size of these steps (for example, first move the upper wall 5 steps to the right, then move the lower wall 0.6 steps to the left). The third parameter to be specified is the When the initial parameters have been set, the computation is offloaded to a server, which computes the flow of the fluids. Intermediate results and the final result are sent back to the mobile device to be shown to the user in a graphical way (i.e. a two-dimensional image of the current separation of the fluids).

A history of past simulations is stored on the device, to compare previous runs with the current. The result of runs should also be exportable to easily sharable formats, such as .png or .pdf, with support for an alpha channel (to realize transparency). For more interactive results, entire runs should be exportable to animated .png or .gif files.

\section{General constraints}
The user interface should be suitable for mobile devices, so it is easy to visualize the results and show them to other people without much hassle. To make it even easier to quickly demonstrate mixing results to others, the actual computation on the server should not take too long (a couple of seconds at most). We do not want to be locked to one specific type of device, so we have chosen to design a cross-platform solution. To easily share results, it should also be possible to export the result of a mixing run to image files, and entire runs to animated files.

\section{User characteristics}
As documented before, the user of the application should be able to specify initial parameters, and, after the application has sent these off to the server, should be able to view the results. The user can then store these results to reference later, and to export these results to image files with transparency. 

\section{Environment description}
The main device for the user interface is the mobile device. We are planning to create a cross-platform solution, which means it will be possible to use the application on various kinds of devices. Examples of supported devices are Apple iPhones and Android phones or tablets. The initial concentration of the fluids, the mixing protocol and the shape of the mixer will be specified on such a device.

As mobile devices typically do not have the power (both processing power and battery capacity) to perform intensive computations, the hard work of computing the matrices will be offloaded to a server. The starting parameters described above will be distributed to the server, which has an efficient FORTRAN implementation to solve the problem. While solving, intermediate results are sent back to the mobile device for displaying.

\section{Assumptions and dependencies}
\todo{Assumptions and dependencies.}
