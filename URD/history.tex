\chapter*{Document Status Sheet}
\section*{General}
\begin{tabular}[!]{l p{10cm}}
    Document title:     &   \TitelFull \\
    Identification:     &   \TitelAbbr\Version\\
    Author:             &   \tessa, \roel, \benjamin, \femke, \hugo \\
    Document status:    &   Initial\\

\end{tabular}

\section*{Document history}
\begin{tabular}[!]{|l|l|l|p{7cm}|}
    \hline
    \emph{Version}    &   \emph{Date} & \emph{Author} &  \emph{Reason of change}\\
    \hline
    0.1    &   24-Apr-2013  &  \pbox{0.3\textwidth}{\tessa \\ \roel \\ \benjamin \\ \femke \\ \hugo} &  Initial version. \\
    \hline
    0.2    &   26-Apr-2013  &  \pbox{0.3\textwidth}{\tessa \\ \roel \\ \benjamin \\ \femke \\ \hugo} &  Revised version as prompted by the client meeting on 25-Apr \\
    \hline
        0.3    &   2-May-2013  &  \pbox{0.3\textwidth}{\tessa \\ \roel \\ \benjamin \\ \femke \\ \hugo} &  Revised version following from feedback of team managers and technical advisor.\\
    \hline
\end{tabular}

\clearpage

\chapter*{Document Change Records since previous issue}
\section*{General}
\begin{tabular}[!]{ll}
    Date:          &   2-May-2013 \\
    Document title: &   \TitelFull\\
    Identification:  &   \TitelAbbr\Version\\
\end{tabular}

\section*{Changes}
For each of the document changes listed here, we will refer to the pages and paragraphs in the previous version of this document, to clarify what exactly has been changed. \\


\begin{longtable}{|l|l|p{11cm}|}
    \hline
    \emph{Page} &   \emph{Paragraph}    &   \emph{Reason to change}\\
    \hline
    \endhead
    \hline
    \endfoot
    - & Abstract & Replaced `how it should function and in what environment it should function' to `how and in what environment it should function'.\\
    5 & 1 & Added an introductionary sentence.\\
    5 & 1.3 & Added definitions, and made a seperate list with abbreviations. \\
    8 & 2.1 & Reworded the first paragraph to remove ambiguity. \\
    8 & 2.1 & Removed implementation detail: constraints and a vector. \\
    8 & 2.2 & Added general description to section head. \\
    8 & 2.2.1 & Reworded first paragraph to clarify that the client device does not do simulation and to remove some ambiguity. \\
    9 & 2.2.1 & Clarified the second constraint to explain what the characteristics mean. \\
    9 & 2.2.1 & Extended the paragraph for the third constraint by giving an example protocol. \\
    9 & 2.2.1 & Reworded last paragraph to clarify how and where the user can draw the fluids. \\
    9 & 2.2.2 & Clarified that the application will also probably run on desktop PCs, but this is not actively supported. \\
    9 & 2.2.2 & Clarified that multiple runs can be compared, not just previous runs and the current.\\
    9 & 2.2.2 & Clarified what is meant by intermediate results. \\
    9 & 2.2.2 & Listed .svg as an example file format instead of a requirement, so it can be changed in the future if we desire. \\
    9 & 2.2.2 & Clarified that the result is also visualised on the client device. \\
    9 & 2.3 & Reworded the paragraph to make it easier to read. \\
    9 & 2.3 & Specified what we will save. \\
    9 & 2.4 & Added description of the target audience. \\
    9 & 2.4 & Added that it is possible to compare mixers using their performance results. \\
    9 & 2.5 & Clarified when intermediate results are visualised. \\
    11 & 3 & Added reference to Appendix A.\\
    11 & 3 & Changed formulation of sentence `Any requirements following from further request will be added here'.\\
    11-15 & 3.1/3.2 & Grouped requirements into smaller blocks.\\
    11-12 & 3.1 & Split CPR03 in CPR3 and CPR4.\\
    12-13 & 3.1 & Changed formulation of CPR5, CPR11, CPR17, CPR18 and CPR19.\\
    13 & 3.1 & Added CPR22.\\
    14 & 3.2 & Split CNR02 in CNR2 and CNR3. Split CNR03 in CNR4, CNR5 and CNR6.\\
    14 & 3.2 & Changed priority of CNR6 to won't have, because Opera isn't compatible with the testing environment.\\
    14 & 3.2 & Changed CNR10, CNR11 and CNR12 to `average waiting time', instead of `waiting time'.\\
    15 & 3.2 & Changed formulation of CNR13.\\
    - & Appendix & Added alternatives to all use cases, where applicable.\\
    18-21 & Appendix & Improved the readability. Inserted "the" a lot. The old version had more of a mechanical summation rather than fluent text.\\
    19 & Appendix & "Define an initial distribution"; Inserted intermediate steps of saving file (choose location/name). Added `at the chosen location' for safe step at system side.\\
    21 & Appendix & "Define mixing protocol(2)"; Inserted intermediate steps of saving file (choose location/name). Added `at the chosen location' for safe step at system side. Changed reference from \emph{Define mixing protocol (0)} to \emph{A.10} (also fixing the erroneous (0) (should be (1))). Inserted steps from A.9 (user can do paint actions at intermediate steps) \\
    19 & Appendix & "Define an initial distribution"; Inserted intermediate steps of saving file (choose location/name). Added 'at the chosen location' for safe step at system side.\\
    21 & Appendix & "Define mixing protocol(2)"; Inserted intermediate steps of saving file (choose location/name). Added 'at the chosen location' for safe step at system side. Changed reference from \emph{Define mixing protocol (0)} to \emph{A.10} (also fixing the erroneous (0) (should be (1))). Inserted steps from A.9 (user can do paint actions at intermediate steps) \\
    21 & Appendix & "Define mixing protocol(2)"; Inserted intermediate steps of saving file (choose location/name). Added 'at the chosen location' for safe step at system side. Changed reference from \emph{Define mixing protocol (1)} to \emph{A.10}.\\
    18-25 & Appendix & Changed preconditions of use cases to includes at step 1. \\
    22-30 & Appendix & Added 'safe mixing run', 'remove initial concentration distribution' and 'remove mixing protocol' use cases. \\
    17 & 3.1 & Rephrased requirements to clarify that images/animations are exported, also removed the corresponding delete requirements. \\
\end{longtable}

