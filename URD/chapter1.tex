\chapter{Introduction}
This chapter lists general information about this document.

\section{Purpose}
This document contains the requirements for \projectname. These requirements are a negotiated agreement between prof.dr.ir. P.D. Anderson and \projectauthor. The listed requirements will be implemented in the \applicationname\ according to their priorities. The requirements with the \emph{must have} priority (see chapter 3) will be assured and if time allows, other requirements might also be implemented. Any changes to these requirements require the full consent of both parties.

\section{Scope}
\projectname\ is an application designed and developed by \projectauthor\ for prof. dr. ir. P.D. Anderson. The application provides a cross-platform tool to visualise fluid mixing. Users can define the initial concentration distribution, as well as manipulate the mixing protocol. The resulting fluid distribution can be stored and analysed by the user for comparison purposes. \\


\section{List of definitions and abbreviations}
\subsection{Definitions}

\begin{description}
\item[Client:] Prof.dr.ir. P.D. Anderson.
\item[Firefox:] A web browser developed by Mozilla.
\item[Google Chrome:] A web browser developed by Google.
\item[Internet Explorer:] A web browser developed by Microsoft.
\item[iOS:] A mobile operating system developed by Apple.
\item[iOS Safari:] A web browser developed by Apple designed for devices running iOS.
\item[Opera:] A web browser developed by Opera Software.
\item[Safari:] A web browser developed by Apple.
\item[iPhone:] A line of smartphones developed by Apple.
\item[Concentration Distribution:] The distribution of the fluids in the mixer.
\item[Mixing Protocol:] Sequence of mixer wall movements with their step($D$).
\item[Step($D$):] Parameter of the mixing protocol that indicates the amount of time the wall should move.
\item[\#steps:] Parameter of the mixing protocol that indicates how many times the protocol is applied.
\item[Mixing Performance:] List of numbers representing the fluid segregation throughout the mixing protocol.
\item[Mixing Run:] Combination of an initial concentration distribution, and a mixing protocol.
\end{description}

\subsection{Abbreviations}
\begin{tabular}{l|l}
2IP35 & The Software Engineering Project \\
CM    &Configuration Manager \\
CPR & Capability Requirement \\
CNR & Constraint Requirement \\
PC & Personal Computer \\
TU/e  &Eindhoven University of Technology \\
SEP   &Software Engineering Project \\
SR    &Software Requirements \\
SRD   &Software Requirements Document \\
URD & This document, the User Requirements Document \\
\end{tabular}

\section{List of references}

\bibliographystyle{plain}

\bibliography{../ref}

\section{Overview}

The remaining chapters describe the user requirements in more detail. Chapter 2 gives a general description of
\begin{itemize}
\item The relation to other systems (2.1)
\item The main capabilities (2.2)
\item Constraint information and justification (2.3)
\item User characteristics (2.4)
\item The operational environment (2.5)
\item Assumptions and dependencies (2.6)
\end{itemize}

Chapter 3 gives a detailed list of the system's capability requirements in section 3.1, and a list of the constraint requirements is given in section 3.2.\\
In the appendices all use cases related to this project are described. Appendix A contains the use case diagram, which shows the relations between the different use cases. Appendix B gives detailed descriptions of all individual use cases.
