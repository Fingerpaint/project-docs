\chapter{Introduction}

\section{Purpose}
The user requirements document (URD) contains the requirements for \projectname. These requirements are a negotiated agreement between prof.dr.ir. P.D. Anderson and \projectauthor. All of the listed requirements, and only these, will be implemented in Fingerpaint, according to their priorities. Any changes to these requirements require the full consent of both parties.

\section{Scope}
\projectname is an application which visualises fluid mixing on a mobile device. Users can define the initial concentration, as well as manipulate the mixing protocol. The resulting fluid distribution can be stored and analyzed by the user for comparison purposes.

\section{List of definitions}
\begin{tabular}{l|l}
2IP35 & The Software Engineering Course \\
Client & prof.dr.ir. P.D. Anderson \\
CM    &Configuration Manager \\
CPR & Capability Requirement \\
CNR & Constraint Requirement \\
TU/e  &Eindhoven University of Technology \\
SEP   &Software Engineering Project \\
SR    &Software Requirements \\
SRD   &Software Requirements Document \\
TBC & To Be Confirmed \\
TBD & To Be Defined \\
\end{tabular}

\section{List of references}

\bibliographystyle{plain}

\bibliography{../ref}

\section{Overview}
%\todo{Short description of the rest of the URD and how it is organized.}

The remainder chapters describe the user requirements in more detail. Chapter 2 gives a general description of 
\begin{itemize}
\item 2.1 - The relation to other systems,
\item 2.2 - The main capabilities,
\item 2.3 - Constraint information and justification,
\item 2.4 - User charactaristics,
\item 2.5 - The operational environment, and
\item 2.6 - Assumptions and dependencies.
\end{itemize}

Chapter 3 gives a detailed list of the system's capabilitiy requirements in section 3.1, and a list of the constraint requirements is given in section 3.2.


