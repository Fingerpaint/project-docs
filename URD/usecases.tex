\begin{appendices}

% Voor elke sectie is er een \label aangemaakt zodat je er naar kan refereren met \ref.
TODO: uitleggen dat we aannemen dat de user toestemming gegeven heeft om de app in het geheugen van het device te laten zooien. Als dit niet zo is, hebben de use cases ... als alternatief dat er geen toestemming gegeven is en dat het resultaat daardoor niet opgeslagen/geladen/verwijderd kan worden.

\chapter{Use cases}
\section{View mixing history}
  \label{mixhist}
  \textbf{Goals}: To view the result of a mixing run.\\
  \textbf{Preconditions}: None.\\
  \textbf{Summary}: The performance of the selected mixing run is shown, accompanied by a picture of the final result.\\
  \textbf{Priority}: Should have.\\
  \textbf{Steps}: \\
    \begin{tabular}{ p{0.5\textwidth} p{0.5\textwidth} }
  	\emph{Actor actions:} & \emph{FingerPaint response:} \\
    1. \ref{mixprot2} or \ref{mixprot3} (including optional steps) & 2. Gives according responses\\
    3. The user taps the \emph{View history} button. & 4. A list with previously saved runs is displayed. \\
    5. The user taps one of the runs shown in the list. & 6. The performance result of the selected run and the final mixing result are shown. \\
    \end{tabular}

\section{Remove mixing run from history}
 \label{removemixrun}
  \textbf{Goals}: To remove the result of a mixing run.\\
  \textbf{Preconditions}: None.\\
  \textbf{Summary}: The details of the selected run (\todo{image} and performance result) are removed from the history.\\
  \textbf{Priority}: Should have.\\
  \textbf{Steps}: \\
    \begin{tabular}{ p{0.5\textwidth} p{0.5\textwidth} }
  	\emph{Actor actions:} & \emph{FingerPaint response:} \\
      1. \ref{mixprot2} or \ref{mixprot3} (including optional steps) & 2. Gives according responses\\
    3. The user taps the \emph{View history} button. & 4. A list with previously saved runs is displayed. \\
    5. The user taps one of the runs shown in the list. & 6. The performance result of the selected run and the final mixing result are shown. \\
    7. The user taps the \emph{Delete} button. & 8. The \todo{details} of the selected run are deleted from storage. \\
    \end{tabular}

\section{Save mixing run image}
  \label{savemiximage}
  \textbf{Goals}: To save the resulting image of a mixing run.\\
  \textbf{Preconditions}: None.\\
  \textbf{Summary}: The image of the executed mixing run is stored locally on the user's device.\\
  \textbf{Priority}: Should have.\\
  \textbf{Steps}: \\
  \begin{tabular}{ p{0.5\textwidth} p{0.5\textwidth} }
  	\emph{Actor actions:} & \emph{FingerPaint response:} \\
        1. \ref{mixprot2} or \ref{mixprot3} & 2. Gives according responses. (The results of the mixing run are now visualized  on the device) \\	
	 3. The user selects the \emph{Save Image} option. & 4. The save interface is displayed.\\
	 5. The user selects a location on his/her device to save the image. & \\
	 6. The user chooses a name for the image. & \\
	 7. The user taps the \emph{Save} button. & 8. A confirmation message is shown. \\
	 9. The user taps the \emph{OK} button. & 10. The output interface is displayed again. \\
  \end{tabular}

  \section{Save mixing run performance graph}
  \label{savemixgraph}
  \textbf{Goals}: To save the performance graph of a mixing run.\\
  \textbf{Preconditions}: None.\\
  \textbf{Summary}: The performance graph of the executed mixing run is stored locally on the user's device.\\
  \textbf{Priority}: Should have.\\
  \textbf{Steps}: \\
  \begin{tabular}{ p{0.5\textwidth} p{0.5\textwidth} }
  	\emph{Actor actions:} & \emph{FingerPaint response:} \\
  1. \ref{mixprot2} or \ref{mixprot3} & 2. Gives according responses. (The results of the mixing run are now visualized  on the device) \\  
	 3. The user selects the \emph{Save Performance} option. & 4. The save interface is displayed.\\
	 5. The user selects a location on his/her device to save the performance graph. & \\
	 6. The user chooses a name for the performance graph. & \\
	 7. The user taps the \emph{Save} button. & 8. A confirmation message is shown. \\
	 9. The user taps the \emph{OK} button. & 10. The output interface is displayed again. \\
  \end{tabular}

  \section{Save mixing run animation}
   \label{savemixanim}
  \textbf{Goals}: To save the resulting animation of a mixing run.\\
  \textbf{Preconditions}: None.\\
  \textbf{Summary}: The animation of the executed mixing run is stored locally on the user's device.\\
  \textbf{Priority}: Could have.\\
  \textbf{Steps}: \\
  \begin{tabular}{ p{0.5\textwidth} p{0.5\textwidth} }
  	\emph{Actor actions:} & \emph{FingerPaint response:} \\
	  1. \ref{mixprot2} or \ref{mixprot3} & 2. Gives according responses. (The results of the mixing run are now visualized  on the device) \\
	 3. The user selects the \emph{Save Animation} option. & 4. The save interface is displayed.\\
	 5. The user selects a location on his/her device to save the animation. & \\
	 6. The user chooses a name for the animation. & \\
	 7. The user taps the \emph{Save} button. & 8. A confirmation message is shown. \\
	 9. The user taps the \emph{OK} button. & 10. The output interface is displayed again. \\
  \end{tabular}

    \section{View multiple mixing performance results from previous runs}
  \label{viewmulruns}
  \textbf{Goals}: To view the mixing performance of multiple runs in the same graph.\\
  \textbf{Preconditions}: None. %At least two mixing runs must have been executed and saved.\\
  \textbf{Summary}: The performance of the selected runs are shown in the same graph.\\
  \textbf{Priority}: Should have.\\
  \textbf{Steps}: \\
  \begin{tabular}{ p{0.5\textwidth} p{0.5\textwidth} }
  	\emph{Actor actions:} & \emph{FingerPaint response:} \\
    1. \ref{mixprot2} or \ref{mixprot3} & 2. Gives according responses. (The results of the mixing run are now visualized  on the device) \\
    3. Repeats step (1), $>$ 1 number of times. & 4. Repeats step (2) the same amount of times. ($\geq$ 2 runs are now saved).\\
	5. The user taps the \emph{View history} button. & 6. The history interface is displayed. \\
	 7. The user selects the \emph{View performance} option. & 8. A list of previously saved runs is displayed.\\
	 9. The user selects two or more runs from the list. & \\
	 10. The user taps the \emph{Submit} button. & 11. The mixing performances of the selected runs are displayed in one graph.\\
  \end{tabular}

\section{Define a mixing geometry and mixer}
  \label{geomixer}
  \textbf{Goals}: To define a mixing geometry and mixer.\\
  \textbf{Preconditions}: none.\\
  \textbf{Summary}: The user selects the geometry used for the mixing process.\\
  \textbf{Priority}: Could have.\\
  \textbf{Steps}: \\
  \begin{tabular}{ p{0.5\textwidth} p{0.5\textwidth} }
  	\emph{Actor actions:} & \emph{FingerPaint response:} \\
	1. The user taps the \emph{start mixing} button. & 2. Opens the the mixing interface. \\
	3. The user selects a mixing geometry of choice from the mixing interface (rectangle, square, circle or journal bearing). & 4. Closes mixing interface, opens new menu with mixers associated with the chosen mixing geometry\\
	5. The user selects a mixer of choice from the menu. & 6.	Displays a blank initial distribution menu, conform chosen mixing geometry and mixer.\\
  \end{tabular}

Alternatief: niks geselecteerd, dan een error-message!

  \section{Load a predefined distribution}
  \label{loadpreddist}
  \textbf{Goals}: To load a predefined distribution.\\
  \textbf{Preconditions}: none.\\
  \textbf{Summary}: The user loads a predefined distribution.\\
  \textbf{Priority}: Should have.\\
  \textbf{Steps}: \\
  \begin{tabular}{ p{0.5\textwidth} p{0.5\textwidth} }
  	\emph{Actor actions:} & \emph{FingerPaint response:} \\
    1. \ref{geomixer} & 2. Gives according responses. \\
	3. The user taps the \emph{Load distribution} button. & 4. Displays a menu with the predefined distributions for the selected geometry. \\
	5. The user taps on the predefined distribution of choice & 6.	Displays the canvas with the selected distribution. \\
  \end{tabular}


  \section{Define an initial distribution}
  \label{initdist}
  \textbf{Goals}: To define an initial distribution.\\
  \textbf{Preconditions}: None. %Mixing geometry and mixer have been chosen.\\
  \textbf{Summary}: The user defines the initial concentration distribution\\
  \textbf{Priority}: Must have.\\
  \textbf{Steps}: \\
  \begin{tabular}{ p{0.5\textwidth} p{0.5\textwidth} }
  	\emph{Actor actions:} & \emph{FingerPaint response:} \\
    1. \ref{geomixer} & 2. Gives according responses. \\
    3. ($Optional_1$) \ref{loadpreddist} & 4. Gives according responses. \\
	5. The user taps the \emph{White} or \emph{Black} button. & 6. Gives visual feedback on the selected colour. \\
	7. The user moves their finger on the screen to define the initial distribution. & 8. Gives real-time visual feedback of the selected area in the selected colour.\\
    9. Repeats step 1 \& 3 until satisfied & 10.	Repeats step 2 \& 4 accordingly. \\
    11.	($Optional_2$) The user taps the \emph{save as predefined distribution} button  & \\
    12. ($Optional_2$ cont.) The user selects a location on the device to save the distribution & \\
    13. ($Optional_2$ cont.) The user chooses a name for the distribution & \\
    14. ($Optional_2$ cont.) The user taps the \emph{save} button & 15. Saves the current distribution as a predefined distribution for the selected geometry and mixer at the chosen location.\\
    16.	The user does or does not tick the \emph{Intermediate steps} tickbox & 17. Saves preference
  \end{tabular}

  \section{Define mixing protocol (1)}
  \label{mixprot1}
  \textbf{Goals}: To define the mixing protocol\\
  \textbf{Preconditions}: None. %Initial concentration distribution has been defined.\\
  \textbf{Summary}: The user defines the mixing protocol\\
  \textbf{Priority}: Must have.\\
  \textbf{Steps}: \\
  \begin{tabular}{ p{0.5\textwidth} p{0.5\textwidth} }
  	\emph{Actor actions:} & \emph{FingerPaint response:} \\
    1. \ref{initdist} & 2. Gives according responses. Hereafter it disables the \emph{Intermediate steps} checkbox. (it cannot be edited anymore). \\
    3. The user taps on the \emph{stepsize display} (D) button & 4.	Gives visual feedback on the stepsize that has been selected\\
    5. The user taps the adjacent increment/decrement buttons & 6.	Increments/decrements the value in the display (with 0.1 accuracy). \\
    7. The user moves his/her finger (left(L) to right(R) or right to left) adjacent to the geometry to indicate the movement of the geometry & 8.	* Case Rectangle/square : Interprets it as L to R or R to L movement of the top or bottom wall based on proximity. Displays which movement has been selected. * Case Circle / Journal bearing: Interprets it as clockwise/anti-clickwise movement of the (1st or 2nd ) circle based on proximity. Displays which movement has been selected.\\
    9.	Repeats step 3-5-7 until satisfactory parameters have been selected & 10. Repeats step 4-6-8 accordingly. The most recent parameter value is applied.\\
  \end{tabular}
  \\
\\Remark: Steps 3-4-5-6 and 7-8 can also be executed in reverse order.

  \section{Define mixing protocol (2)}
  \label{mixprot2}
  \textbf{Goals}: To define the mixing protocol\\
  \textbf{Preconditions}: \emph{Intermediate steps} has been ticked.\\
  \textbf{Summary}: The user defines the mixing protocol\\
  \textbf{Priority}: Must have.\\
  \textbf{Steps}: \\
  \begin{tabular}{ p{0.5\textwidth} p{0.5\textwidth} }
  	\emph{Actor actions:} & \emph{FingerPaint response:} \\
    1. \ref{mixprot1} & 2. Gives according responses. \\
    3. The user taps the \emph{Mix now} button & 4.	Adds the movement to the protocol-log. Computes the result of applying the given movement to the distribution and displays it on the canvas.\\
    5.	Repeats steps at \textbf{\ref{initdist}} (only 5-7-9), \textbf{\ref{mixprot1}} (3 and onwards) and (3) until satisfied & 6.	Gives according responses\\
    7. (Optional) Clicks on the \emph{save protocol} button & \\
    8. (Optional cont.) The user selects a location on the device to save the protocol & \\
    9. (Optional cont.) The user chooses a name for the protocol & \\
    10. (Optional cont.) The user taps the \emph{save} button & 11. Saves the protocol-log and (intermediate) visualisation(s) at the chosen location. \\

  \end{tabular}
%%%%%Tm hier edited
 \section{Define mixing protocol (3)}
  \label{mixprot3}
  \textbf{Goals}: To define the mixing protocol\\
  \textbf{Preconditions}: \emph{Intermediate steps} has \textbf{not} been ticked.\\
  \textbf{Summary}: The user defines the mixing protocol\\
  \textbf{Priority}: Must have.\\
  \textbf{Steps}: \\
  \begin{tabular}{ p{0.5\textwidth} p{0.5\textwidth} }
  	\emph{Actor actions:} & \emph{FingerPaint response:} \\
    1. \ref{mixprot1} & 2. Gives according responses. \\
    3.	The user taps the \emph{Add to protocol} button	& 4.	Adds the selected movement to protocol-log. \\
    5.	Repeats steps at \textbf{\ref{mixprot1}} (3 and onwards) and (3) until satisfied	& 6.	Gives according responses\\
    7.	The user taps on the \emph{Mix now} button	& 8.	Computes the result of applying all movements in the protocol-log to the initial concentration distribution, and displays it on the canvas.\\
    9.	(Optional) The user taps on the \emph{save protocol} button & \\
    10. (Optional cont.) The user selects a location on the device to save the protocol & \\
    11. (Optional cont.) The user chooses a name for the protocol & \\
    12. (Optional cont.) The user taps the \emph{save} button &  13. Saves the protocol-log and resulting distribution at the chosen location. \\
  \end{tabular}
\end{appendices}
