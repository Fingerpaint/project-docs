\begin{appendices}

% Voor elke sectie is er een \label aangemaakt zodat je er naar kan refereren met \ref.

\chapter{Use cases}
\section{View mixing history}
  \label{mixhist}
  \textbf{Goals}: To view the result of a mixing run.\\
  \textbf{Preconditions}: At least one mixing run must have been executed and saved.\\
  \textbf{Summary}: The performance of the selected mixing run is shown, accompanied by a picture of the final result.\\
  \textbf{Priority}: Should have.\\
  \textbf{Steps}: \\
    \begin{tabular}{ p{0.5\textwidth} p{0.5\textwidth} }
  	\emph{Actor actions:} & \emph{FingerPaint response:} \\
    1. The user taps the \emph{View history} button. & 2. A list with previously saved runs is displayed. \\
    3. The user taps one of the runs shown in the list. & 4. The performance result of the selected run and the final mixing result are shown. \\
    \end{tabular}

\section{Remove mixing run from history}
 \label{removemixrun}
  \textbf{Goals}: To remove the result of a mixing run.\\
  \textbf{Preconditions}: At least one mixing run must have been executed and saved.\\
  \textbf{Summary}: The details of the selected run (image and performance result) are removed from the history.\\
  \textbf{Priority}: Should have.\\
  \textbf{Steps}: \\
    \begin{tabular}{ p{0.5\textwidth} p{0.5\textwidth} }
  	\emph{Actor actions:} & \emph{FingerPaint response:} \\
    1. The user taps the \emph{View history} button. & 2. A list with previously saved runs is displayed. \\
    3. The user taps one of the runs shown in the list. & 4. The performance result of the selected run and the final mixing result are shown. \\
    5. The user taps the \emph{Delete} button. & 6. The details of the selected run are deleted from storage. \\
    \end{tabular}

\section{Save mixing run image}
  \label{savemiximage}
  \textbf{Goals}: To save the resulting image of a mixing run.\\
  \textbf{Preconditions}: The user has defined an initial concentration distribution and a mixing protocol, and has pressed the submit button.\\
  \textbf{Summary}: The image of the executed mixing run is stored locally on the user's device.\\
  \textbf{Priority}: Should have.\\
  \textbf{Steps}: \\
  \begin{tabular}{ p{0.5\textwidth} p{0.5\textwidth} }
  	\emph{Actor actions:} & \emph{FingerPaint response:} \\
	   & 1. The results of the mixing run are visualized on the device. \\
	 2. The user selects the \emph{Save Image} option. & 3. The save interface is displayed.\\
	 4. The user selects a location on his/her device to save the image. & \\
	 5. The user chooses a name for the image. & \\
	 6. The user taps the \emph{Save} button. & 7. A confirmation message is shown. \\
	 8. The user taps the \emph{OK} button. & 9. The output interface is displayed again. \\
  \end{tabular}

  \section{Save mixing run performance graph}
  \label{savemixgraph}
  \textbf{Goals}: To save the performance graph of a mixing run.\\
  \textbf{Preconditions}: The user has defined an initial concentration distribution and a mixing protocol, and has pressed the submit button.\\
  \textbf{Summary}: The performance graph of the executed mixing run is stored locally on the user's device.\\
  \textbf{Priority}: Should have.\\
  \textbf{Steps}: \\
  \begin{tabular}{ p{0.5\textwidth} p{0.5\textwidth} }
  	\emph{Actor actions:} & \emph{FingerPaint response:} \\
	   & 1. The results of the mixing run are visualised on the device. \\
	 2. The user selects the \emph{Save Performance} option. & 3. The save interface is displayed.\\
	 4. The user selects a location on his/her device to save the performance graph. & \\
	 5. The user chooses a name for the performance graph. & \\
	 6. The user taps the \emph{Save} button. & 7. A confirmation message is shown. \\
	 8. The user taps the \emph{OK} button. & 9. The output interface is displayed again. \\
  \end{tabular}

  \section{Save mixing run animation}
   \label{savemixanim}
  \textbf{Goals}: To save the resulting animation of a mixing run.\\
  \textbf{Preconditions}: The user has defined an initial concentration distribution and a mixing protocol, and has pressed the submit button.\\
  \textbf{Summary}: The animation of the executed mixing run is stored locally on the user's device.\\
  \textbf{Priority}: Could have.\\
  \textbf{Steps}: \\
  \begin{tabular}{ p{0.5\textwidth} p{0.5\textwidth} }
  	\emph{Actor actions:} & \emph{FingerPaint response:} \\
	   & 1. The results of the mixing run are visualized on the device. \\
	 2. The user selects the \emph{Save Animation} option. & 3. The save interface is displayed.\\
	 4. The user selects a location on his/her device to save the animation. & \\
	 5. The user chooses a name for the animation. & \\
	 6. The user taps the \emph{Save} button. & 7. A confirmation message is shown. \\
	 8. The user taps the \emph{OK} button. & 9. The output interface is displayed again. \\
  \end{tabular}

    \section{View multiple mixing performance results from previous runs}
  \label{viewmulruns}
  \textbf{Goals}: To view the mixing performance of multiple runs in the same graph.\\
  \textbf{Preconditions}: At least two mixing runs must have been executed and saved.\\
  \textbf{Summary}: The performance of the selected runs are shown in the same graph.\\
  \textbf{Priority}: Should have.\\
  \textbf{Steps}: \\
  \begin{tabular}{ p{0.5\textwidth} p{0.5\textwidth} }
  	\emph{Actor actions:} & \emph{FingerPaint response:} \\
	   1. The user taps the \emph{View history} button. & 2. The history interface is displayed. \\
	 3. The user selects the \emph{View performance} option. & 4. A list of previously saved runs is displayed.\\
	 5. The user selects two or more runs from the list. & \\
	 6. The user taps the \emph{Submit} button. & 7. The mixing performances of the selected runs are displayed in one graph.\\
  \end{tabular}

%%%%%%%%%%%%%%%%%%%%%%%%%%%%%%%%%%%%%%%%%%%%%%%%%%%%%%%%%%%%%%%%%%%%%%%%%%%%%%% Start use cases Hugo %%%%%%%%%%%%%%%%%%%%%%%%%%%%%%%%%%%%%%%%%%%%%%%%%%%%%%%%%%%%%%
    %Let op: A.9 en A.10 zijn hardcoded!! Pas op met aanpassen

\section{Define a mixing geometry and mixer}
  \label{geomixer}
  \textbf{Goals}: To define a mixing geometry and mixer.\\
  \textbf{Preconditions}: none.\\
  \textbf{Summary}: The user selects the geometry used for the mixing process.\\
  \textbf{Priority}: Could have.\\
  \textbf{Steps}: \\
  \begin{tabular}{ p{0.5\textwidth} p{0.5\textwidth} }
  	\emph{Actor actions:} & \emph{FingerPaint response:} \\
	1. The user taps the \emph{start mixing} button. & 2. Opens the the mixing interface. \\
	3. The user selects a mixing geometry of choice from the mixing interface (rectangle, square, circle or journal bearing). & 4. Closes menu, opens new menu with mixers associated with the mixing geometry\\
	5. The user selects a mixer of choice from the menu. & 6.	Display blank initial distribution menu, conform chosen mixing geometry and mixer\\
  \end{tabular}

  \section{Load a predefined distribution}
  \label{loadpreddist}
  \textbf{Goals}: To load a predefined distribution.\\
  \textbf{Preconditions}: none.\\
  \textbf{Summary}: The user loads a predefined distribution.\\
  \textbf{Priority}: Could have.\\
  \textbf{Steps}: \\
  \begin{tabular}{ p{0.5\textwidth} p{0.5\textwidth} }
  	\emph{Actor actions:} & \emph{FingerPaint response:} \\
	1. The user taps the \emph{Load $C_0$} button. & 2. Display a menu with the predefined distributions for the selected geometry. \\
	3. The user taps on the predefined distribution of choice & 4.	Displays the canvas with the selected distribution. \\
  \end{tabular}

  \section{Define an initial distribution}
  \label{initdist}
  \textbf{Goals}: To define an initial distribution.\\
  \textbf{Preconditions}: Mixing geometry and mixer have been chosen.\\
  \textbf{Summary}: The user defines the initial concentration distribution\\
  \textbf{Priority}: Must have.\\
  \textbf{Steps}: \\
  \begin{tabular}{ p{0.5\textwidth} p{0.5\textwidth} }
  	\emph{Actor actions:} & \emph{FingerPaint response:} \\
	1. The user taps the \emph{Color button} (white or black). & 2. Gives visual feedback on the selected colour. \\
	3. The user moves their finger on the screen to define the initial distribution. & 4. Gives real-time visual feedback of the selected area in the selected colour.\\
    5. Repeats step 1 \& 3 until satisfied & 6.	Repeats step 2 \& 4 accordingly. \\
    7.	Optional: The user taps the \emph{save as predefined distribution} button  & \\
    8. Optional (cont.): The user selects a location on the device to save the distribution & \\
    9. Optional (cont.): The user chooses a name for the distribution & \\
    10. Optional (cont.): The user taps the \emph{save} button & 11. Saves the current distribution as a predefined distribution for the selected geometry and mixer at the chosen location.\\
    12.	The user does or does not tick the \emph{Intermediate steps} tickbox & 13. Saves preference
  \end{tabular}

  \section{Define mixing protocol (1)}
  \label{mixprot1}
  \textbf{Goals}: To define the mixing protocol\\
  \textbf{Preconditions}: Initial concentration distribution has been defined.\\
  \textbf{Summary}: The user defines the mixing protocol\\
  \textbf{Priority}: Must have.\\
  \textbf{Steps}: \\
  \begin{tabular}{ p{0.5\textwidth} p{0.5\textwidth} }
  	\emph{Actor actions:} & \emph{FingerPaint response:} \\
    & 1.     Disables the \emph{Intermediate steps} checkbox. It cannot be edited anymore.\\
    2. The user taps on the \emph{stepsize display} (D) button & 3.	Gives visual feedback on the stepsize that has been selected\\
    4. The user taps the adjacent increment/decrement buttons & 5.	Increments/decrements the value in the display (with 0.1 accuracy). \\
    6. The user moves his/her finger (left(L) to right(R) or right to left) adjacent to the geometry to indicate the movement of the geometry & 7.	* Case Rectangle/square : Interprets it as L to R or R to L movement of the top or bottom wall based on proximity. Displays which movement has been selected. * Case Circle / Journal bearing: Interprets it as clockwise/anti-clickwise movement of the (1st or 2nd ) circle based on proximity. Displays which movement has been selected.\\
    8.	Repeats step 1-3-5 until satisfactory parameters have been selected & 9.	Repeats step 2-4-6 accordingly. The most recent parameter value is applied.\\
  \end{tabular}
  \\
\\Remark: 1-2-3-4 and 5-6 can also be executed in reverse order.

  \section{Define mixing protocol (2)}
  \label{mixprot2}
  \textbf{Goals}: To define the mixing protocol\\
  \textbf{Preconditions}: Define mixing protocol(1) and \emph{Intermediate steps} has been ticked.\\
  \textbf{Summary}: The user defines the mixing protocol\\
  \textbf{Priority}: Must have.\\
  \textbf{Steps}: \\
  \begin{tabular}{ p{0.5\textwidth} p{0.5\textwidth} }
  	\emph{Actor actions:} & \emph{FingerPaint response:} \\
    1. The user taps the \emph{Mix now!} button & 2.	Adds the movement to the protocol-log. Computes the result of applying the given movement to the distribution and displays it on the canvas.\\
    3.	Repeats steps at \textbf{\ref{initdist}} (only 1-3-5), \textbf{\ref{mixprot1}} (1) and (2) until satisfied & 4.	Give according responses\\
    %Let op: A.9 en A.10 zijn hardcoded!! Pas op met aanpassen
    5. Optional: Clicks on the \emph{save protocol} button & \\
    6. Optional (cont.): The user selects a location on the device to save the protocol & \\
    7. Optional (cont.): The user chooses a name for the protocol & \\
    8. Optional (cont.): The user taps the \emph{save} button & 9. Saves the protocol-log and (intermediate) visualisation(s) at the chosen location. \\

  \end{tabular}

 \section{Define mixing protocol (3)}
  \label{mixprot3}
  \textbf{Goals}: To define the mixing protocol\\
  \textbf{Preconditions}: Define mixing protocol(1) and \emph{Intermediate steps} has NOT been ticked.\\
  \textbf{Summary}: The user defines the mixing protocol\\
  \textbf{Priority}: Must have.\\
  \textbf{Steps}: \\
  \begin{tabular}{ p{0.5\textwidth} p{0.5\textwidth} }
  	\emph{Actor actions:} & \emph{FingerPaint response:} \\
    1.	The user taps the \emph{Add to protocol!} button	& 2.	Adds the selected movement to protocol-log . \\
    3.	Repeats steps at \textbf{\ref{mixprot1}} (all) and (1) until satisfied	& 4.	Gives according responses\\
        %Let op: A.9 en A.10 zijn hardcoded!! Pas op met aanpassen
    5.	The user taps on the \emph{Show mixture!} button	& 6.	Computes the result of applying all movements in the protocol-log to the initial concentration distribution and displays it on the canvas.\\
    7.	Optional: The user taps on the \emph{save protocol} button & \\
    8. Optional (cont.): The user selects a location on the device to save the protocol & \\
    9. Optional (cont.): The user chooses a name for the protocol & \\
    10. Optional (cont.): The user taps the \emph{save} button &  11. Saves the protocol-log and resulting distribution at the chosen location. \\
  \end{tabular}
\end{appendices}
