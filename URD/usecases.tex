% Voor elke sectie is er een \label aangemaakt zodat je er naar kan refereren met \ref.
\chapter{Use cases}
This appendix lists all use cases for the \applicationname.

%Define a mixing geometry and mixer
\section{Define a mixing geometry and mixer}
  \label{geomixer}
  \textbf{Goals}: To define a mixing geometry and mixer.\\
  \textbf{Preconditions}: none.\\
  \textbf{Summary}: The user selects the geometry used for the mixing process.\\
  \textbf{Priority}: \emph{Must have} for rectangle, \emph{should have} for square and \emph{could have} for circle and \emph{Journal Bearing}.\\
  \textbf{Steps}: \\
  \begin{tabular}{ p{0.5\textwidth} p{0.5\textwidth} }
  	\emph{Actor actions:} & \emph{\projectname\ response:} \\
	1. Taps the \emph{Start Mixing} button. & 2. Displays the mixing interface.\\
	3. Selects a mixing geometry of choice from the mixing interface (rectangle, square, circle or \emph{Journal Bearing}). & 4. Displays all mixers associated with the chosen mixing geometry. \\
	5. Selects a mixer of choice. & 6. Displays a white initial concentration distribution canvas, conform chosen mixing geometry and mixer.\\
  \end{tabular}
  
    %Load a previously saved concentration distribution
  \section{Load a previously saved concentration distribution}
  \label{loadsavedist}
  \textbf{Goals}: To load a previously saved concentration distribution.\\
  \textbf{Preconditions}: none.\\
  \textbf{Summary}: The user loads a previously saved concentration distribution.\\
  \textbf{Priority}: \emph{Should have}.\\
  \textbf{Steps}: \\
  \begin{tabular}{ p{0.5\textwidth} p{0.5\textwidth} }
  	\emph{Actor actions:} & \emph{\projectname\ response:} \\
  	1. Actor actions of \textbf{\ref{geomixer}}. & 2. \projectname\ responses of \textbf{\ref{geomixer}}. \\
	3. Taps the \emph{Load Saved Distribution} button. & 4. Displays the previously saved concentration distributions for the selected geometry. \\
	5. Taps on the concentration distribution of choice & 6. Displays the canvas with the selected concentration distribution. \\
	\end{tabular}
	        \\
     \\\textbf{Alternative}: \\
    \begin{tabular}{ p{0.5\textwidth} p{0.5\textwidth} }
  	\emph{Actor actions:} & \emph{\projectname\ response:} \\
            & 4. Displays a \emph{No Saved Distributions} message. \\
    5. Taps the \emph{OK} button. & 6. Displays the mixing interface again. \\
  \end{tabular}
  
  %Load a predefined concentration distribution
  \section{Load a predefined concentration distribution}
  \label{loadpreddist}
  \textbf{Goals}: To load a predefined concentration distribution.\\
  \textbf{Preconditions}: none.\\
  \textbf{Summary}: The user loads a predefined concentration distribution.\\
  \textbf{Priority}: \emph{Could have}.\\
  \textbf{Steps}: \\
  \begin{tabular}{ p{0.5\textwidth} p{0.5\textwidth} }
  	\emph{Actor actions:} & \emph{\projectname\ response:} \\
  	1. Actor actions of \textbf{\ref{geomixer}}. & 2. \projectname\ responses of \textbf{\ref{geomixer}}. \\
	3. Taps the \emph{Load Predefined Distribution} button. & 4. Displays the predefined concentration distributions for the selected geometry. \\
	5. Taps on the predefined concentration distribution of choice & 6. Displays the canvas with the selected concentration distribution. \\
  \end{tabular}
  
  %Define an initial concentration distribution
  \section{Define an initial concentration distribution}
  \label{initdist}
  \textbf{Goals}: To define an initial concentration distribution.\\
  \textbf{Preconditions}: None. \\ %Mixing geometry and mixer have been chosen.
  \textbf{Summary}: The user defines the initial concentration distribution\\
  \textbf{Priority}: \emph{Must have}.\\
  \textbf{Steps}: \\
  \begin{tabular}{ p{0.5\textwidth} p{0.5\textwidth} }
  	\emph{Actor actions:} & \emph{\projectname\ response:} \\
  	1. Actor actions of \textbf{\ref{geomixer}} or \textbf{\ref{loadsavedist}} or \textbf{\ref{loadpreddist}}. & 2. \projectname\ responses of \textbf{\ref{geomixer}} or \textbf{\ref{loadsavedist}} or  \textbf{\ref{loadpreddist}}.\\
	3. Selects black or white as color. & 4. Gives visual feedback on the selected colour. \\
	5. Moves their finger across the screen to define the initial concentration distribution. & 6. Gives real-time visual feedback of the selected area in the selected colour.\\
      7. Repeats step 3 \& 5 until satisfied. & 8. Repeats step 4 \& 6 accordingly. \\
      \end{tabular}
      
        % Save initial concentration distribution
  \section{Save concentration distribution}
  \label{savedist}
  \textbf{Goals}: To save a concentration distribution.\\
  \textbf{Preconditions}: None. \\ %Mixing geometry and mixer have been chosen.
  \textbf{Summary}: The user saves the concentration distribution\\
  \textbf{Priority}: \emph{Should have}.\\
  \textbf{Steps}: \\
  \begin{tabular}{ p{0.5\textwidth} p{0.5\textwidth} }
  	\emph{Actor actions:} & \emph{\projectname\ response:} \\
  	1. Actor actions of \textbf{\ref{initdist}}. & 2. \projectname\ responses of \textbf{\ref{initdist}}. \\
    3. Taps the \emph{Save Distribution} button.  & \\
    4. Chooses a name for the concentration distribution. & \\
    5. Taps the \emph{Save} button. & 6. Displays a \emph{Save was succesfull} message.\\
    7. Taps the \emph{OK} button. & 8. Displays the mixing interface again. \\
    \end{tabular}
    	 \\
    \\\textbf{Alternative S.1}: \\
    \begin{tabular}{ p{0.5\textwidth} p{0.5\textwidth} }
  	\emph{Actor actions:} & \emph{\projectname\ response:} \\
           &  8. Displays an \emph{Insufficient access rights} message. \\
	 9. Taps the \emph{OK} button. & 10. Displays the mixing interface again. \\
    \end{tabular}
    \\
    \\\textbf{Alternative S.2}: \\
    \begin{tabular}{ p{0.5\textwidth} p{0.5\textwidth} }
  	\emph{Actor actions:} & \emph{\projectname\ response:} \\
           &  8. Displays an \emph{Insufficient memory} message. \\
	 9. Taps the \emph{OK} button. & 10. Displays the mixing interface again. \\
    \end{tabular}
    \\
    \\\textbf{Alternative S.3.1}: \\
    \begin{tabular}{ p{0.5\textwidth} p{0.5\textwidth} }
  	\emph{Actor actions:} & \emph{\projectname\ response:} \\
   & 8. Displays a \emph{Name already in use} message with \emph{Overwrite} and \emph{Cancel} options. \\
	 9. Taps the \emph{Overwrite} button. & 10. Displays a \emph{Save was succesfull} message. \\
	 11. Taps the \emph{OK} button. & 12. Displays the mixing interface again. \\
    \end{tabular}
    \\
    \\\textbf{Alternative S.3.2}: \\
    \begin{tabular}{ p{0.5\textwidth} p{0.5\textwidth} }
  	\emph{Actor actions:} & \emph{\projectname\ response:} \\
   & 8. Displays a \emph{Name already in use} message with \emph{Overwrite} and \emph{Cancel} options. \\
	 9. Taps the \emph{Cancel} button. & 10. Displays the mixing interface again. \\
    \end{tabular}
    
          %Load a previously saved mixing protocol
  \section{Load a previously saved mixing protocol}
  \label{loadsaveprot}
  \textbf{Goals}: To load a previously saved mixing protocol.\\
  \textbf{Preconditions}: \emph{Intermediate Steps} has \textbf{not} been ticked.\\
  \textbf{Summary}: The user loads a previously saved mixing protocol.\\
  \textbf{Priority}: \emph{Should have}.\\
  \textbf{Steps}: \\
  \begin{tabular}{ p{0.5\textwidth} p{0.5\textwidth} }
  	\emph{Actor actions:} & \emph{\projectname\ response:} \\
  	1. Actor actions of \textbf{\ref{initdist}} or  \textbf{\ref{savedist}}. & 2. \projectname\ responses of \textbf{\ref{initdist}} or  \textbf{\ref{savedist}}. \\
	3. Taps the \emph{Load Saved Protocol} button. & 4. Displays the previously saved mixing protocols for the selected geometry. \\
	5. Taps on the mixing protocol of choice. & 6. Displays the selected mixing protocol.\\
	\end{tabular}
	        \\
     \\\textbf{Alternative}: \\
    \begin{tabular}{ p{0.5\textwidth} p{0.5\textwidth} }
  	\emph{Actor actions:} & \emph{\projectname\ response:} \\
            & 4. Displays a \emph{No Saved Protocols} message. \\
    5. Taps the \emph{OK} button. & 6. Displays the mixing interface again. \\
  \end{tabular}
  
  %Load a predefined mixing protocol
  \section{Load a predefined mixing protocol}
  \label{loadpredprot}
  \textbf{Goals}: To load a predefined mixing protocol.\\
  \textbf{Preconditions}: none.\\
  \textbf{Summary}: The user loads a predefined mixing protocol.\\
  \textbf{Priority}: \emph{Could have}.\\
  \textbf{Steps}: \\
  \begin{tabular}{ p{0.5\textwidth} p{0.5\textwidth} }
  	\emph{Actor actions:} & \emph{\projectname\ response:} \\
  	1. Actor actions of \textbf{\ref{initdist}} or  \textbf{\ref{savedist}}. & 2. \projectname\ responses of \textbf{\ref{initdist}} or  \textbf{\ref{savedist}}. \\
	3. Taps the \emph{Load Predefined Distribution} button. & 4. Displays the predefined mixing protocols for the selected geometry. \\
	5. Taps on the predefined mixing protocol of choice & 6. Displays the selected mixing protocol. \\
  \end{tabular}
  	 \\
     \\\textbf{Alternative}: \\
    \begin{tabular}{ p{0.5\textwidth} p{0.5\textwidth} }
  	\emph{Actor actions:} & \emph{\projectname\ response:} \\
            & 4. Displays a \emph{No Saved Protocols} message. \\
    5. Taps the \emph{OK} button. & 6. Displays the mixing interface again. \\
  \end{tabular}
  
     %Define mixing protocol (1)
  \section{Define mixing protocol (1)}
  \label{mixprot1}
  \textbf{Goals}: To define the mixing protocol\\
  \textbf{Preconditions}: None. \\%Initial concentration distribution has been defined.
  \textbf{Summary}: The user defines the mixing protocol\\
  \textbf{Priority}: \emph{Must have}.\\
  \textbf{Steps}: \\
  \begin{tabular}{ p{0.5\textwidth} p{0.5\textwidth} }
  	\emph{Actor actions:} & \emph{\projectname\ response:} \\
    1. Actor actions of \textbf{\ref{initdist}} or  \textbf{\ref{savedist}} or \textbf{\ref{loadsaveprot}} or \textbf{\ref{loadpredprot}}. & 2. \projectname\ responses of \textbf{\ref{initdist}} or  \textbf{\ref{savedist}} or  \textbf{\ref{loadsaveprot}} or \textbf{\ref{loadpredprot}}. \\
    3. Selects a value for step ($D$). & 4.	Gives visual feedback on the step ($D$) that has been selected\\
    5. Swipes his/her finger across the screen to indicate the movement of the geometry. & 6. Displays which movement has been selected. \\
    7.The user does or does not tick the \emph{Intermediate Steps} tickbox. & 8. Displays result in checkbox.\\
    9. Repeats step 3, 5 \& 7 until satisfactory parameters have been selected & 10. Repeats step 4, 6 \& 8 accordingly. The most recent parameter value is applied.\\
  \end{tabular}
  \\
  \\\textbf{Remark}: Steps 3-4 , 5-6 and 7-8 can also be executed in reverse order.

%Define mixing protocol (2)
  \section{Define mixing protocol (2)}
  \label{mixprot2}
  \textbf{Goals}: To define the mixing protocol\\
  \textbf{Preconditions}: \emph{Intermediate Steps} has been ticked.\\
  \textbf{Summary}: The user defines the mixing protocol\\
  \textbf{Priority}: \emph{Must have}.\\
  \textbf{Steps}: \\
  \begin{tabular}{ p{0.5\textwidth} p{0.5\textwidth} }
  	\emph{Actor actions:} & \emph{\projectname\ response:} \\
    1. Actor actions of \textbf{\ref{mixprot1}}. & 2. \projectname\ responses of \textbf{\ref{mixprot1}}. \\
    3. Taps the \emph{Mix Now} button & 4. Disables the \emph{Intermediate Steps} checkbox (it can not be edited anymore). Adds the movement to the protocol-log. Computes the result of applying the given movement to the distribution and displays it on the canvas.\\
    5. Repeats steps at \textbf{\ref{initdist}} (only 3, 5 \& 7), \textbf{\ref{mixprot1}} (only 3, 5, 7 \& 9) and (3) until satisfied. & 6. Repeats steps at \textbf{\ref{initdist}} (only 4, 6 \& 8), \textbf{\ref{mixprot1}} (only 4, 6, 8 \& 10) and (4). \\
    \end{tabular}
    \\
    \\\textbf{Alternative}:\\
      \begin{tabular}{ p{0.5\textwidth} p{0.5\textwidth} }
  	\emph{Actor actions:} & \emph{\projectname\ response:} \\
 & 4. Adds the movement to the protocol-log. Fails to connect to server. Displays a \emph{Connection Failed} message.\\
    5. Taps the \emph{OK} button. & 6. Displays the mixing interface again. \\
    \end{tabular}
    
%Define mixing protocol (3)
 \section{Define mixing protocol (3)}
  \label{mixprot3}
  \textbf{Goals}: To define the mixing protocol\\
  \textbf{Preconditions}: \emph{Intermediate Steps} has \textbf{not} been ticked.\\
  \textbf{Summary}: The user defines the mixing protocol\\
  \textbf{Priority}: Must have.\\
  \textbf{Steps}: \\
  \begin{tabular}{ p{0.5\textwidth} p{0.5\textwidth} }
  	\emph{Actor actions:} & \emph{\projectname\ response:} \\
    1. Actor actions of \textbf{\ref{mixprot1}}. & 2. \projectname\ responses of \textbf{\ref{mixprot1}}. \\
    3. Taps the \emph{Add To Protocol} button & 4. Disables the \emph{Intermediate Steps} checkbox (it can not be edited anymore). Adds the selected movement to protocol-log. \\
    5. Repeats steps at \textbf{\ref{mixprot1}} (3, 5, 7 \& 9) and (3) until satisfied & 6. Repeats steps at \textbf{\ref{mixprot1}} (4, 6, 8 \& 10) and (4). \\
    7. Taps on the \emph{Mix Now} button & 8. Computes the result of applying all movements in the protocol-log to the initial concentration distribution, and displays it on the canvas.\\
  \end{tabular}
  \\
      \\\textbf{Alternative}:\\
      \begin{tabular}{ p{0.5\textwidth} p{0.5\textwidth} }
  	\emph{Actor actions:} & \emph{\projectname\ response:} \\
 & 8. Fails to connect to server. Displays a \emph{Connection Failed} message.\\
    5. Taps the \emph{OK} button. & 6. Displays the mixing interface again. \\
    \end{tabular}
    
        %Save mixing run
     \section{Save mixing run}
  \label{saverun}
  \textbf{Goals}: To save a mixing run.\\
  \textbf{Preconditions}: None. \\
  \textbf{Summary}: The user saves the mixing run.\\
  \textbf{Priority}: \emph{Should have}.\\
  \textbf{Steps}: \\
  \begin{tabular}{ p{0.5\textwidth} p{0.5\textwidth} }
  	\emph{Actor actions:} & \emph{\projectname\ response:} \\
  	1. Actor actions of \textbf{\ref{mixprot2}} or \textbf{\ref{mixprot3}}. & 2. \projectname\ responses of \textbf{\ref{mixprot2}} or \textbf{\ref{mixprot3}}. \\
    3. Taps the \emph{Save Run} button.  & \\
    4. Chooses a name for the run. & \\
    5. Taps the \emph{Save} button. & 6. Displays a \emph{Save was succesfull} message.\\
    7. Taps the \emph{OK} button. & 8. Displays the mixing interface again. \\
      \end{tabular}
    	 \\
    \\\textbf{Alternatives}: Alternatives \textbf{S.1, S.2, S.3.1 and S.3.2} are applicable here.
    
        % Save mixing protocol
  \section{Save mixing protocol}
  \label{saveprot}
  \textbf{Goals}: To save a mixing protocol.\\
  \textbf{Preconditions}: None. \\
  \textbf{Summary}: The user saves the mixing protocol.\\
  \textbf{Priority}: Should have.\\
  \textbf{Steps}: \\
  \begin{tabular}{ p{0.5\textwidth} p{0.5\textwidth} }
  	\emph{Actor actions:} & \emph{\projectname\ response:} \\
  	1. Actor actions of \textbf{\ref{mixprot2}} or \textbf{\ref{mixprot3}}. & 2. \projectname\ responses of \textbf{\ref{mixprot2}} or \textbf{\ref{mixprot3}}. \\
    3. Taps the \emph{Save Protocol} button.  & \\
    4. Chooses a name for the protocol. & \\
    5. Taps the \emph{Save} button. & 6. Displays a \emph{Save was succesfull} message.\\
    7. Taps the \emph{OK} button. & 8. Displays the mixing interface again. \\
      \end{tabular}
    	 \\
    \\\textbf{Alternatives}: Alternatives \textbf{S.1, S.2, S.3.1 and S.3.2} are applicable here.
    
    %View previously saved mixing run
\section{View previously saved mixing run}
  \label{mixhist}
  \textbf{Goals}: To view the details of a previously saved mixing run.\\
  \textbf{Preconditions}: None.\\
  \textbf{Summary}: The performance of the selected mixing run is shown, accompanied by an image of the final result.\\
  \textbf{Priority}: \emph{Should have}.\\
  \textbf{Steps}: \\
    \begin{tabular}{ p{0.5\textwidth} p{0.5\textwidth} }
  	\emph{Actor actions:} & \emph{\projectname\ response:} \\
    1. Taps the \emph{View history} button. & 2. Displays the history interface.\\
    3. Selects the \emph{View mixing run details}. & 4. Displays all previously saved runs.\\
    5. Taps one of the runs shown. & 6. Displays the performance result of the selected run and an image of the final mixing result. \\
    \end{tabular}
    \\
     \\\textbf{Alternative 1}: \\
    \begin{tabular}{ p{0.5\textwidth} p{0.5\textwidth} }
  	\emph{Actor actions:} & \emph{\projectname\ response:} \\
            & 4. Displays an \emph{Insufficient access rights} error message. \\
    5. Taps the \emph{OK} button. & 6. Displays the history interface again. \\
    \end{tabular}
        \\
     \\\textbf{Alternative 2}: \\
    \begin{tabular}{ p{0.5\textwidth} p{0.5\textwidth} }
  	\emph{Actor actions:} & \emph{\projectname\ response:} \\
            & 4. Displays a \emph{No Saved Runs} message. \\
    5. Taps the \emph{OK} button. & 6. Displays the history interface again. \\
    \end{tabular}
    
    %Export mixing run image
\section{Export mixing run image}
  \label{savemiximage}
  \textbf{Goals}: To export the resulting image of a mixing run.\\
  \textbf{Preconditions}: None.\\
  \textbf{Summary}: The image of the executed mixing run is stored locally on the user's device.\\
  \textbf{Priority}: \emph{Should have}.\\
  \textbf{Steps}: \\
  \begin{tabular}{ p{0.5\textwidth} p{0.5\textwidth} }
  	\emph{Actor actions:} & \emph{\projectname\ response:} \\
      1. Actor actions of \textbf{\ref{mixprot2}} or \textbf{\ref{mixprot3}} or \textbf{\ref{mixhist}}. &  2. \projectname\ responses of \textbf{\ref{mixprot2}} or \textbf{\ref{mixprot3}} or \textbf{\ref{mixhist}}.\\
	 3. Selects the \emph{Export Image} option. & 4. Displays the \emph{Save} interface.\\
	 5. Selects a location on his/her device to save the image. & \\
	 6. Chooses a name for the image. & \\
	 7. Taps the \emph{Save} button. & 8. Displays a \emph{Save was succesfull} message. \\
	 9. Taps the \emph{OK} button. & 10. Displays the mixing interface again. \\
  \end{tabular}
  \\
  \\\textbf{Alternatives}: Alternatives \textbf{S.1, S.2, S.3.1 and S.3.2} are applicable here.
   
%Export mixing run performance graph
  \section{Export mixing run performance graph}
  \label{savemixgraph}
  \textbf{Goals}: To export the performance graph of a mixing run.\\
  \textbf{Preconditions}: None.\\
  \textbf{Summary}: The performance graph of the executed mixing run is stored locally on the user's device.\\
  \textbf{Priority}: \emph{Should have}.\\
  \textbf{Steps}: \\
  \begin{tabular}{ p{0.5\textwidth} p{0.5\textwidth} }
  	\emph{Actor actions:} & \emph{\projectname\ response:} \\
      1. Actor actions of \textbf{\ref{mixprot2}} or \textbf{\ref{mixprot3}} or \textbf{\ref{mixhist}}. &  2. \projectname\ responses of \textbf{\ref{mixprot2}} or \textbf{\ref{mixprot3}} or \textbf{\ref{mixhist}}.\\
      	 3. Selects the \emph{Export Performance} option. & 4. Displays the \emph{Save} interface.\\
	 5. Selects a location on his/her device to save the performance graph. & \\
	 6. Chooses a name for the performance graph. & \\
	 7. Taps the \emph{Save} button. & 8. Displays a \emph{Save was succesfull} message. \\
	 9. Taps the \emph{OK} button. & 10. Displays the mixing interface again. \\
  \end{tabular}
  \\
  \\\textbf{Alternatives}: Alternatives \textbf{S.1, S.2, S.3.1 and S.3.2} are applicable here.
  
%Export mixing run animation
  \section{Export mixing run animation}
   \label{savemixanim}
  \textbf{Goals}: To export the resulting animation of a mixing run.\\
  \textbf{Preconditions}: None.\\
  \textbf{Summary}: The animation of the executed mixing run is stored locally on the user's device.\\
  \textbf{Priority}: \emph{Could have}.\\
  \textbf{Steps}: \\
  \begin{tabular}{ p{0.5\textwidth} p{0.5\textwidth} }
  	\emph{Actor actions:} & \emph{\projectname\ response:} \\
      1. Actor actions of \textbf{\ref{mixprot2}} or \textbf{\ref{mixprot3}} or \textbf{\ref{mixhist}}. &  2. \projectname\ responses of \textbf{\ref{mixprot2}} or \textbf{\ref{mixprot3}} or \textbf{\ref{mixhist}}.\\
        	 3. Selects the \emph{Export Animation} option. & 4. Displays the \emph{Save} interface.\\
	 5. Selects a location on his/her device to save the animation. & \\
	 6. Chooses a name for the animation. & \\
	 7. Taps the \emph{Save} button. & 8. Displays a \emph{Save was succesfull} message. \\
	 9. Taps the \emph{OK} button. & 10. Displays the mixing interface again. \\
  \end{tabular}
  \\
    \\\textbf{Alternatives}: Alternatives \textbf{S.1, S.2, S.3.1 and S.3.2} are applicable here.
    
    %View multiple mixing performance results from previous runs
    \section{View multiple mixing performance results from previous mixing runs}
  \label{viewmulruns}
  \textbf{Goals}: To view the mixing performance of multiple mixing runs in the same graph.\\
  \textbf{Preconditions}: None.\\
  \textbf{Summary}: The performance of the selected mixing runs are shown in the same graph.\\
  \textbf{Priority}: \emph{Should have}.\\
  \textbf{Steps}: \\
  \begin{tabular}{ p{0.5\textwidth} p{0.5\textwidth} }
  	\emph{Actor actions:} & \emph{\projectname\ response:} \\
	1. Taps the \emph{View history} button. & 2. Displays the history interface. \\
	3. Selects the \emph{View performance} option. & 4. Displays all previously saved mixing runs.\\
	 5. Selects two or more mixing runs from the list. & \\
	 6. The user taps the \emph{Submit} button. & 7. Displays the mixing performances of the selected mixing runs in one graph.\\
  \end{tabular}
  \\
     \\\textbf{Alternative 1}: \\
    \begin{tabular}{ p{0.5\textwidth} p{0.5\textwidth} }
  	\emph{Actor actions:} & \emph{\projectname\ response:} \\
            & 4. Displays an \emph{Insufficient access rights} error message. \\
    5. Taps the \emph{OK} button. & 6. Displays the history interface again. \\
    \end{tabular}
            \\
     \\\textbf{Alternative 2}: \\
    \begin{tabular}{ p{0.5\textwidth} p{0.5\textwidth} }
  	\emph{Actor actions:} & \emph{\projectname\ response:} \\
            & 4. Displays a \emph{No Saved Runs} message. \\
    5. Taps the \emph{OK} button. & 6. Displays the history interface again. \\
    \end{tabular}
    
        %Export mixing run performance graph with multiple runs
  \section{Export performance graph of multiple mixing runs}
  \label{savemulrunsgraph}
  \textbf{Goals}: To export the performance graph of multiple mixing runs.\\
  \textbf{Preconditions}: None.\\
  \textbf{Summary}: The performance graph of multiple mixing runs is stored locally on the user's device.\\
  \textbf{Priority}: \emph{Should have}.\\
  \textbf{Steps}: \\
  \begin{tabular}{ p{0.5\textwidth} p{0.5\textwidth} }
  	\emph{Actor actions:} & \emph{\projectname\ response:} \\
      1. Actor actions of \textbf{\ref{viewmulruns}}. &  2. \projectname\ responses of \textbf{\ref{viewmulruns}}.\\
      	 3. Selects the \emph{Export Graph} option. & 4. Displays the \emph{Save} interface.\\
	 5. Selects a location on his/her device to save the performance graph. & \\
	 6. Chooses a name for the performance graph. & \\
	 7. Taps the \emph{Save} button. & 8. Displays a \emph{Save was succesfull} message. \\
	 9. Taps the \emph{OK} button. & 10. Displays the mixing interface again. \\
  \end{tabular}
  \\
  \\\textbf{Alternatives}: Alternatives \textbf{S.1, S.2, S.3.1 and S.3.2} are applicable here.

%Remove mixing run from history
\section{Remove mixing run from history}
 \label{removemixrun}
  \textbf{Goals}: To remove the result of a mixing run.\\
  \textbf{Preconditions}: None.\\
  \textbf{Summary}: The selected mixing run is removed from the history.\\
  \textbf{Priority}: \emph{Should have}.\\
  \textbf{Steps}: \\
    \begin{tabular}{ p{0.5\textwidth} p{0.5\textwidth} }
  	\emph{Actor actions:} & \emph{\projectname\ response:} \\
    1. Taps the \emph{View history} button. & 2. Displays the history interface.\\
    3. Selects the \emph{View mixing run details}. & 4. Displays all previously saved mixing runs.\\
    5. Presses and holds one of the mixing runs shown. & 6. Displays an options menu. \\
    7. Taps the \emph{Remove file} option. & 8. Displays an \emph{Are you sure?} message.\\
    9. Taps the \emph{Yes} button. & 10. Deletes the selected mixing run from storage. \\
     & 11. Displays all previously saved mixing runs.
    \end{tabular}
            \\
     \\\textbf{Alternative 1}: \\
    \begin{tabular}{ p{0.5\textwidth} p{0.5\textwidth} }
  	\emph{Actor actions:} & \emph{\projectname\ response:} \\
            & 4. Displays a \emph{No Saved Runs} message. \\
    5. Taps the \emph{OK} button. & 6. Displays the history interface again. \\
    \end{tabular}
    \\
     \\\textbf{Alternative R.1}: \\
    \begin{tabular}{ p{0.5\textwidth} p{0.5\textwidth} }
  	\emph{Actor actions:} & \emph{\projectname\ response:} \\
            & 6. Displays an \emph{Insufficient access rights} error message. \\
    7. Taps the \emph{OK} button. & 8. Displays the history interface again. \\
    \end{tabular}
    \\
         \\\textbf{Alternative R.2}: \\
        \begin{tabular}{ p{0.5\textwidth} p{0.5\textwidth} }
  	\emph{Actor actions:} & \emph{\projectname\ response:} \\
    9. Taps the \emph{No} button. & 10. Displays all previously saved mixing runs. \\
    \end{tabular}
    
%Remove concentration distribution
\section{Remove concentration distribution}
 \label{removedist}
  \textbf{Goals}: To remove a concentration distribution.\\
  \textbf{Preconditions}: None.\\
  \textbf{Summary}: The selected concentration distribution is removed from local storage.\\
  \textbf{Priority}: \emph{Should have}.\\
  \textbf{Steps}: \\
    \begin{tabular}{ p{0.5\textwidth} p{0.5\textwidth} }
  	\emph{Actor actions:} & \emph{\projectname\ response:} \\
    1. Taps the \emph{View Distributions} button. & 2. Displays all previously saved concentration distributions.\\
    3. Presses and holds one of the concentration distributions shown. & 4. Displays an options menu. \\
    5. Taps the \emph{Remove File} option. & 6. Displays an \emph{Are you sure?} message.\\
    7. Taps the \emph{Yes} button. & 8. Deletes the selected concentration distribution from local storage. \\
     & 9. Displays all previously concentration distributions.\\
    \end{tabular}
                \\
     \\\textbf{Alternative 1}: \\
    \begin{tabular}{ p{0.5\textwidth} p{0.5\textwidth} }
  	\emph{Actor actions:} & \emph{\projectname\ response:} \\
            & 4. Displays a \emph{No Saved Distributions} message. \\
    5. Taps the \emph{OK} button. & 6. Displays the history interface again. \\
    \end{tabular}
    \\
    \\\textbf{Alternatives}: Alternatives \textbf{R.1 and R.2} are applicable here.

%Remove mixing protocol
\section{Remove mixing protocol}
 \label{removedist}
  \textbf{Goals}: To remove a mixing protocol.\\
  \textbf{Preconditions}: None.\\
  \textbf{Summary}: The selected mixing protocol is removed from local storage.\\
  \textbf{Priority}: \emph{Should have}.\\
  \textbf{Steps}: \\
    \begin{tabular}{ p{0.5\textwidth} p{0.5\textwidth} }
  	\emph{Actor actions:} & \emph{\projectname\ response:} \\
    1. Taps the \emph{View Protocols} button. & 2. Displays all previously saved mixing protocols.\\
    3. Presses and holds one of the mixing protocol shown. & 4. Displays an options menu. \\
    5. Taps the \emph{Remove File} option. & 6. Displays an \emph{Are you sure?} message.\\
    7. Taps the \emph{Yes} button. & 8. Deletes the selected mixing protocol from local storage. \\
     & 9. Displays all previously mixing protocols.\\
    \end{tabular}
                \\
     \\\textbf{Alternative 1}: \\
    \begin{tabular}{ p{0.5\textwidth} p{0.5\textwidth} }
  	\emph{Actor actions:} & \emph{\projectname\ response:} \\
            & 4. Displays a \emph{No Saved Protocols} message. \\
    5. Taps the \emph{OK} button. & 6. Displays the history interface again. \\
    \end{tabular}
    \\
    \\\textbf{Alternatives}: Alternatives \textbf{R.1 and R.2} are applicable here.
