% Voor elke sectie is er een \label aangemaakt zodat je er naar kan refereren met \ref.
\todo{uitleggen dat we aannemen dat de user toestemming gegeven heeft om de app in het geheugen van het device te laten zooien. Als dit niet zo is, hebben de use cases ... als alternatief dat er geen toestemming gegeven is en dat het resultaat daardoor niet opgeslagen/geladen/verwijderd kan worden.
Aanname toetsenbord conform input ding paneel tekstveld enz.}

\chapter{Use cases}

%View mixing history
\section{View mixing history}
  \label{mixhist}
  \textbf{Goals}: To view the result of a mixing run.\\
  \textbf{Preconditions}: None.\\
  \textbf{Summary}: The performance of the selected mixing run is shown, accompanied by a picture of the final result.\\
  \textbf{Priority}: Should have.\\
  \textbf{Steps}: \\
    \begin{tabular}{ p{0.5\textwidth} p{0.5\textwidth} }
  	\emph{Actor actions:} & \emph{\projectname\ response:} \\
    1. Actor actions of \textbf{\ref{saverun}}. &  2. \projectname\ responses of \textbf{\ref{saverun}}.\\
    3. Taps the \emph{View history} button. & 4. Displays the history interface.\\
    5. Selects the \emph{View mixing run details}. & 6. Displays all previously saved runs.\\
    5. Taps one of the runs shown. & 6. Displays the performance result of the selected run and an image of the final mixing result. \\
    \end{tabular}
    \\
     \\\textbf{Alternative}: \\
    \begin{tabular}{ p{0.5\textwidth} p{0.5\textwidth} }
  	\emph{Actor actions:} & \emph{\projectname\ response:} \\
            & 6. Displays a \emph{Insufficient access rights} error message. \\
    7. Taps the \emph{Ok} button. & 8. Displays the history interface again. \\
    \end{tabular}

%Remove mixing run from history
\section{Remove mixing run from history}
 \label{removemixrun}
  \textbf{Goals}: To remove the result of a mixing run.\\
  \textbf{Preconditions}: None.\\
  \textbf{Summary}: The details of the selected run (\todo{image} and performance result) are removed from the history.\\
  \textbf{Priority}: Should have.\\
  \textbf{Steps}: \\
    \begin{tabular}{ p{0.5\textwidth} p{0.5\textwidth} }
  	\emph{Actor actions:} & \emph{\projectname\ response:} \\
    1. Actor actions of \textbf{\ref{saverun}}. &  2. \projectname\ responses of \textbf{\ref{saverun}}.\\
    3. Taps the \emph{View history} button. & 4. Displays the history interface.\\
    5. Selects the \emph{View mixing run details}. & 6. Displays all previously saved runs.\\
    5. Presses and holds one of the runs shown. & 6. Displays an options menu. \\
    7. Taps the \emph{Remove file} option. & 8. Displays an \emph{Are you sure?} message.\\
    9. Taps the \emph{Yes} button. & 10. Deletes the details of the selected run (\todo{image} and performance result) from storage. \\
     & 11. Displays all previously saved runs.
    \end{tabular}
    \\
     \\\textbf{Alternative 1}: \\
    \begin{tabular}{ p{0.5\textwidth} p{0.5\textwidth} }
  	\emph{Actor actions:} & \emph{\projectname\ response:} \\
            & 6. Displays a \emph{Insufficient access rights} error message. \\
    7. Taps the \emph{Ok} button. & 8. Displays the history interface again. \\
    \end{tabular}
    \\
         \\\textbf{Alternative 2}: \\
        \begin{tabular}{ p{0.5\textwidth} p{0.5\textwidth} }
  	\emph{Actor actions:} & \emph{\projectname\ response:} \\
    9. Taps the \emph{No} button. & 10. Displays all previously saved runs. \\
    \end{tabular}

%Save mixing run image
\section{Save mixing run image}
  \label{savemiximage}
  \todo{Change to export mixing run image}\\
  \textbf{Goals}: To save the resulting image of a mixing run.\\
  \textbf{Preconditions}: None.\\
  \textbf{Summary}: The image of the executed mixing run is stored locally on the user's device.\\
  \textbf{Priority}: Should have.\\
  \textbf{Steps}: \\
  \begin{tabular}{ p{0.5\textwidth} p{0.5\textwidth} }
  	\emph{Actor actions:} & \emph{\projectname\ response:} \\
      1. Actor actions of \textbf{\ref{mixprot2}} or \textbf{\ref{mixprot3}} or \textbf{\ref{mixhist}}. &  2. \projectname\ responses of \textbf{\ref{mixprot2}} or \textbf{\ref{mixprot3}} or \textbf{\ref{mixhist}}.\\
	 3. Selects the \emph{Save Image} option. & 4. Displays the \emph{Save} interface.\\
	 5. Selects a location on his/her device to save the image. & \\
	 6. Chooses a name for the image. & \\
	 7. Taps the \emph{Save} button. & 8. Displays a \emph{Save was succesfull} message. \\
	 9. Taps the \emph{OK} button. & 10. Displays the mixing interface again. \\
  \end{tabular}
  \\
    \\\textbf{Alternative S.1}: \\
    \begin{tabular}{ p{0.5\textwidth} p{0.5\textwidth} }
  	\emph{Actor actions:} & \emph{\projectname\ response:} \\
           &  8. Displays a \emph{Insufficient access rights} message. \\
	 9. Taps the \emph{OK} button. & 10. Displays the mixing interface again. \\
    \end{tabular}
    \\
    \\\textbf{Alternative S.2}: \\
    \begin{tabular}{ p{0.5\textwidth} p{0.5\textwidth} }
  	\emph{Actor actions:} & \emph{\projectname\ response:} \\
           &  8. Displays a \emph{Insufficient memory} message. \\
	 9. Taps the \emph{OK} button. & 10. Displays the mixing interface again. \\
    \end{tabular}
    \\
    \\\textbf{Alternative S.3.1}: \\
    \begin{tabular}{ p{0.5\textwidth} p{0.5\textwidth} }
  	\emph{Actor actions:} & \emph{\projectname\ response:} \\
   & 8. Displays a \emph{Name already in use} message with \emph{Overwrite} and \emph{Cancel} options. \\
	 9. Taps the \emph{Overwrite} button. & 10. Displays a \emph{Save was succesfull} message. \\
	 11. Taps the \emph{OK} button. & 12. Displays the mixing interface again. \\
    \end{tabular}
    \\
    \\\textbf{Alternative S.3.2}: \\
    \begin{tabular}{ p{0.5\textwidth} p{0.5\textwidth} }
  	\emph{Actor actions:} & \emph{\projectname\ response:} \\
   & 8. Displays a \emph{Name already in use} message with \emph{Overwrite} and \emph{Cancel} options. \\
	 9. Taps the \emph{Cancel} button. & 10. Displays the mixing interface again. \\
    \end{tabular}
   
%Save mixing run performance graph
  \section{Save mixing run performance graph}
  \label{savemixgraph}
    \todo{Change to export mixing run image}\\
  \textbf{Goals}: To save the performance graph of a mixing run.\\
  \textbf{Preconditions}: None.\\
  \textbf{Summary}: The performance graph of the executed mixing run is stored locally on the user's device.\\
  \textbf{Priority}: Should have.\\
  \textbf{Steps}: \\
  \begin{tabular}{ p{0.5\textwidth} p{0.5\textwidth} }
  	\emph{Actor actions:} & \emph{\projectname\ response:} \\
      1. Actor actions of \textbf{\ref{mixprot2}} or \textbf{\ref{mixprot3}} or \textbf{\ref{mixhist}}. &  2. \projectname\ responses of \textbf{\ref{mixprot2}} or \textbf{\ref{mixprot3}} or \textbf{\ref{mixhist}}.\\
      	 3. Selects the \emph{Save Performance} option. & 4. Displays the \emph{Save} interface.\\
	 5. Selects a location on his/her device to save the performance graph. & \\
	 6. Chooses a name for the performance graph. & \\
	 7. Taps the \emph{Save} button. & 8. Displays a \emph{Save was succesfull} message. \\
	 9. Taps the \emph{OK} button. & 10. Displays the mixing interface again. \\
  \end{tabular}
  \\
  \\\textbf{Alternatives}: Alternatives \textbf{S.1, S.2, S.3.1 and S.3.2} are applicable here.
  
%Save mixing run animation
  \section{Save mixing run animation}
   \label{savemixanim}
     \todo{Change to export mixing run image}\\
  \textbf{Goals}: To save the resulting animation of a mixing run.\\
  \textbf{Preconditions}: None.\\
  \textbf{Summary}: The animation of the executed mixing run is stored locally on the user's device.\\
  \textbf{Priority}: Could have.\\
  \textbf{Steps}: \\
  \begin{tabular}{ p{0.5\textwidth} p{0.5\textwidth} }
  	\emph{Actor actions:} & \emph{\projectname\ response:} \\
      1. Actor actions of \textbf{\ref{mixprot2}} or \textbf{\ref{mixprot3}} or \textbf{\ref{mixhist}}. &  2. \projectname\ responses of \textbf{\ref{mixprot2}} or \textbf{\ref{mixprot3}} or \textbf{\ref{mixhist}}.\\
        	 3. Selects the \emph{Save Animation} option. & 4. Displays the \emph{Save} interface.\\
	 5. Selects a location on his/her device to save the animation. & \\
	 6. Chooses a name for the animation. & \\
	 7. Taps the \emph{Save} button. & 8. Displays a \emph{Save was succesfull} message. \\
	 9. Taps the \emph{OK} button. & 10. Displays the mixing interface again. \\
  \end{tabular}
  \\
    \\\textbf{Alternatives}: Alternatives \textbf{S.1, S.2, S.3.1 and S.3.2} are applicable here.

%View multiple mixing performance results from previous runs
    \section{View multiple mixing performance results from previous runs}
  \label{viewmulruns}
  \textbf{Goals}: To view the mixing performance of multiple runs in the same graph.\\
  \textbf{Preconditions}: None.\\ %At least two mixing runs must have been executed and saved.
  \textbf{Summary}: The performance of the selected runs are shown in the same graph.\\
  \textbf{Priority}: Should have.\\
  \textbf{Steps}: \\
  \begin{tabular}{ p{0.5\textwidth} p{0.5\textwidth} }
  	\emph{Actor actions:} & \emph{\projectname\ response:} \\
      1. Actor actions of \textbf{\ref{saverun}}. &  2. \projectname\ responses of \textbf{\ref{saverun}}.\\
    3. Repeats step (1) at least one time. & 4. Repeats step (2) the same amount of times. (At least 2 runs are now saved).\\
	5. Taps the \emph{View history} button. & 6. Displays the history interface. \\
	 7. Selects the \emph{View performance} option. & 8. Displays all previously saved runs.\\
	 9. Selects two or more runs from the list. & \\
	 10. The user taps the \emph{Submit} button. & 11. Displays the mixing performances of the selected runs in one graph.\\
  \end{tabular}
  \\
     \\\textbf{Alternative 1}: \\
    \begin{tabular}{ p{0.5\textwidth} p{0.5\textwidth} }
  	\emph{Actor actions:} & \emph{\projectname\ response:} \\
            & 8. Displays a \emph{Insufficient access rights} error message. \\
    9. Taps the \emph{Ok} button. & 10. Displays the history interface again. \\
    \end{tabular}

%Define a mixing geometry and mixer
\section{Define a mixing geometry and mixer}
  \label{geomixer}
  \textbf{Goals}: To define a mixing geometry and mixer.\\
  \textbf{Preconditions}: none.\\
  \textbf{Summary}: The user selects the geometry used for the mixing process.\\
  \textbf{Priority}: Should have for rectangle and square, could have for circle and `Journal Bearing'.\\
  \textbf{Steps}: \\
  \begin{tabular}{ p{0.5\textwidth} p{0.5\textwidth} }
  	\emph{Actor actions:} & \emph{\projectname\ response:} \\
	1. Taps the \emph{start mixing} button. & 2. Displays the mixing interface.\\
	3. Selects a mixing geometry of choice from the mixing interface (rectangle, square, circle or `Journal Bearing'). & 4. Displays all mixers associated with the chosen mixing geometry. \\
	5. Selects a mixer of choice. & 6. Displays a white initial concentration distribution canvas, conform chosen mixing geometry and mixer.\\
  \end{tabular}

  %Load a previously saved distribution
  \section{Load a previously saved distribution}
  \label{loadsavedist}
  \textbf{Goals}: To load a previously saved distribution.\\
  \textbf{Preconditions}: none.\\
  \textbf{Summary}: The user loads a previously saved distribution.\\
  \textbf{Priority}: Should have.\\
  \textbf{Steps}: \\
  \begin{tabular}{ p{0.5\textwidth} p{0.5\textwidth} }
  	\emph{Actor actions:} & \emph{\projectname\ response:} \\
  	1. Actor actions of \textbf{\ref{savedist}}. & 2. \projectname\ responses of \textbf{\ref{savedist}}. \\
  	3. Actor actions of \textbf{\ref{geomixer}}. & 4. \projectname\ responses of \textbf{\ref{geomixer}}. \\
	5. Taps the \emph{Load Saved Distribution} button. & 6. Displays the previously saved distributions for the selected geometry. \\
	7. Taps on the distribution of choice & 8. Displays the canvas with the selected distribution. \\
  \end{tabular}
  
  %Load a predefined distribution
  \section{Load a predefined distribution}
  \label{loadpreddist}
  \textbf{Goals}: To load a predefined distribution.\\
  \textbf{Preconditions}: none.\\
  \textbf{Summary}: The user loads a predefined distribution.\\
  \textbf{Priority}: Could have.\\
  \textbf{Steps}: \\
  \begin{tabular}{ p{0.5\textwidth} p{0.5\textwidth} }
  	\emph{Actor actions:} & \emph{\projectname\ response:} \\
  	1. Actor actions of \textbf{\ref{geomixer}}. & 2. \projectname\ responses of \textbf{\ref{geomixer}}. \\
	3. Taps the \emph{Load Predefined Distribution} button. & 4. Displays the predefined distributions for the selected geometry. \\
	5. Taps on the predefined distribution of choice & 6. Displays the canvas with the selected distribution. \\
  \end{tabular}

%Define an initial distribution
  \section{Define an initial concentration distribution}
  \label{initdist}
  \textbf{Goals}: To define an initial concentration distribution.\\
  \textbf{Preconditions}: None. \\ %Mixing geometry and mixer have been chosen.
  \textbf{Summary}: The user defines the initial concentration distribution\\
  \textbf{Priority}: Must have.\\
  \textbf{Steps}: \\
  \begin{tabular}{ p{0.5\textwidth} p{0.5\textwidth} }
  	\emph{Actor actions:} & \emph{\projectname\ response:} \\
  	1. Actor actions of \textbf{\ref{geomixer}} or \textbf{\ref{loadsavedist}} or \textbf{\ref{loadpreddist}}. & 2. \projectname\ responses of \textbf{\ref{geomixer}} or \textbf{\ref{loadsavedist}} or  \textbf{\ref{loadpreddist}}.\\
	3. Selects black or white as color. & 4. Gives visual feedback on the selected colour. \\
	5. Moves their finger across the screen to define the initial concentration distribution. & 6. Gives real-time visual feedback of the selected area in the selected colour.\\
    7. Repeats step 3 \& 5 until satisfied. & 8.	Repeats step 4 \& 6 accordingly. \\
      \end{tabular}
      
  % Save initial distribution
  \section{Save initial concentration distribution}
  \label{savedist}
  \textbf{Goals}: To save a concentration distribution.\\
  \textbf{Preconditions}: None. \\ %Mixing geometry and mixer have been chosen.
  \textbf{Summary}: The user saves the initial concentration distribution\\
  \textbf{Priority}: Should have.\\
  \textbf{Steps}: \\
  \begin{tabular}{ p{0.5\textwidth} p{0.5\textwidth} }
  	\emph{Actor actions:} & \emph{\projectname\ response:} \\
  	1. Actor actions of \textbf{\ref{initdist}}. & 2. \projectname\ responses of \textbf{\ref{initdist}}. \\
    3. Taps the \emph{Save Distribution} button.  & \\
    4. Selects a location on the device to save the distribution. & \\
    5. Chooses a name for the distribution. & \\
    6. Taps the \emph{Save} button. & 7. Displays a \emph{Save was succesfull} message.\\
    8. Taps the \emph{OK} button. & 9. Displays the mixing interface again. \\
    \end{tabular}
    	 \\
    \\\textbf{Alternatives}: Alternatives \textbf{S.1, S.2, S.3.1 and S.3.2} are applicable here.

   %Define mixing protocol (1)
  \section{Define mixing protocol (1)}
  \label{mixprot1}
  \textbf{Goals}: To define the mixing protocol\\
  \textbf{Preconditions}: None. \\%Initial concentration distribution has been defined.
  \textbf{Summary}: The user defines the mixing protocol\\
  \textbf{Priority}: Must have.\\
  \textbf{Steps}: \\
  \begin{tabular}{ p{0.5\textwidth} p{0.5\textwidth} }
  	\emph{Actor actions:} & \emph{\projectname\ response:} \\
    1. Actor actions of \textbf{\ref{initdist}} or  \textbf{\ref{savedist}}. & 2. \projectname\ responses of \textbf{\ref{initdist}} or  \textbf{\ref{savedist}}. \\
    3. Selects a value for step ($D$). & 4.	Gives visual feedback on the step ($D$) that has been selected\\
    5. Swipes his/her finger across the screen to indicate the movement of the geometry. & 6. Displays which movement has been selected. \\
    7.The user does or does not tick the \emph{Intermediate steps} tickbox. & 8. Displays result in checkbox.\\
    9. Repeats step 3, 5 \& 7 until satisfactory parameters have been selected & 10. Repeats step 4, 6 \& 8 accordingly. The most recent parameter value is applied.\\
  \end{tabular}
  \\
  \\\textbf{Remark}: Steps 3-4 , 5-6 and 7-8 can also be executed in reverse order.

%Define mixing protocol (2)
  \section{Define mixing protocol (2)}
  \label{mixprot2}
  \textbf{Goals}: To define the mixing protocol\\
  \textbf{Preconditions}: \emph{Intermediate Steps} has been ticked.\\
  \textbf{Summary}: The user defines the mixing protocol\\
  \textbf{Priority}: Must have.\\
  \textbf{Steps}: \\
  \begin{tabular}{ p{0.5\textwidth} p{0.5\textwidth} }
  	\emph{Actor actions:} & \emph{\projectname\ response:} \\
    1. Actor actions of \textbf{\ref{mixprot1}}. & 2. \projectname\ responses of \textbf{\ref{mixprot1}}. \\
    3. Taps the \emph{Mix now} button & 4. Disables the \emph{Intermediate Steps} checkbox (it can not be edited anymore). Adds the movement to the protocol-log. Computes the result of applying the given movement to the distribution and displays it on the canvas.\\
    5. Repeats steps at \textbf{\ref{initdist}} (only 3, 5 \& 7), \textbf{\ref{mixprot1}} (only 3, 5, 7 \& 9) and (3) until satisfied. & 6. Repeats steps at \textbf{\ref{initdist}} (only 4, 6 \& 8), \textbf{\ref{mixprot1}} (only 4, 6, 8 \& 10) and (4). \\
    \end{tabular}
    \\
    \\\textbf{Alternative}:\\
      \begin{tabular}{ p{0.5\textwidth} p{0.5\textwidth} }
  	\emph{Actor actions:} & \emph{\projectname\ response:} \\
 & 4. Adds the movement to the protocol-log. Fails to connect to server. Displays a \emph{Connection Failed} message.\\
    5. Taps the \emph{Ok} button. & 6. Displays the mixing interface again. \\
    \end{tabular}
    
%Define mixing protocol (3)
 \section{Define mixing protocol (3)}
  \label{mixprot3}
  \textbf{Goals}: To define the mixing protocol\\
  \textbf{Preconditions}: \emph{Intermediate Steps} has \textbf{not} been ticked.\\
  \textbf{Summary}: The user defines the mixing protocol\\
  \textbf{Priority}: Must have.\\
  \textbf{Steps}: \\
  \begin{tabular}{ p{0.5\textwidth} p{0.5\textwidth} }
  	\emph{Actor actions:} & \emph{\projectname\ response:} \\
    1. Actor actions of \textbf{\ref{mixprot1}}. & 2. \projectname\ responses of \textbf{\ref{mixprot1}}. \\
    3. Taps the \emph{Add to protocol} button & 4. Disables the \emph{Intermediate Steps} checkbox (it can not be edited anymore). Adds the selected movement to protocol-log. \\
    5. Repeats steps at \textbf{\ref{mixprot1}} (3, 5, 7 \& 9) and (3) until satisfied & 6. Repeats steps at \textbf{\ref{mixprot1}} (4, 6, 8 \& 10) and (4). \\
    7. Taps on the \emph{Mix now} button & 8. Computes the result of applying all movements in the protocol-log to the initial concentration distribution, and displays it on the canvas.\\
  \end{tabular}
  \\
      \\\textbf{Alternative}:\\
      \begin{tabular}{ p{0.5\textwidth} p{0.5\textwidth} }
  	\emph{Actor actions:} & \emph{\projectname\ response:} \\
 & 8. Fails to connect to server. Displays a \emph{Connection Failed} message.\\
    5. Taps the \emph{Ok} button. & 6. Displays the mixing interface again. \\
    \end{tabular}
  
    % Save mixing protocol
  \section{Save mixing protocol}
  \label{saveprot}
  \textbf{Goals}: To save a mixing protocol.\\
  \textbf{Preconditions}: None. \\
  \textbf{Summary}: The user saves the mixing protocol.\\
  \textbf{Priority}: Should have.\\
  \textbf{Steps}: \\
  \begin{tabular}{ p{0.5\textwidth} p{0.5\textwidth} }
  	\emph{Actor actions:} & \emph{\projectname\ response:} \\
  	1. Actor actions of \textbf{\ref{mixprot2}} or \textbf{\ref{mixprot3}}. & 2. \projectname\ responses of \textbf{\ref{mixprot2}} or \textbf{\ref{mixprot3}}. \\
    3. Taps the \emph{Save Protocol} button.  & \\
    4. Selects a location on the device to save the protocol. & \\
    5. Chooses a name for the protocol. & \\
    6. Taps the \emph{Save} button. & 7. Displays a \emph{Save was succesfull} message.\\
    8. Taps the \emph{OK} button. & 9. Displays the mixing interface again. \\
      \end{tabular}
    	 \\
    \\\textbf{Alternatives}: Alternatives \textbf{S.1, S.2, S.3.1 and S.3.2} are applicable here.
    
    %Save mixing run
     \section{Save mixing run}
  \label{saverun}
  \textbf{Goals}: To save a mixing run.\\
  \textbf{Preconditions}: None. \\
  \textbf{Summary}: The user saves the mixing run.\\
  \textbf{Priority}: Should have.\\
  \textbf{Steps}: \\
  \begin{tabular}{ p{0.5\textwidth} p{0.5\textwidth} }
  	\emph{Actor actions:} & \emph{\projectname\ response:} \\
  	1. Actor actions of \textbf{\ref{mixprot2}} or \textbf{\ref{mixprot3}}. & 2. \projectname\ responses of \textbf{\ref{mixprot2}} or \textbf{\ref{mixprot3}}. \\
    3. Taps the \emph{Save Run} button.  & \\
    4. Selects a location on the device to save the run. & \\
    5. Chooses a name for the run. & \\
    6. Taps the \emph{Save} button. & 7. Displays a \emph{Save was succesfull} message.\\
    8. Taps the \emph{OK} button. & 9. Displays the mixing interface again. \\
      \end{tabular}
    	 \\
    \\\textbf{Alternatives}: Alternatives \textbf{S.1, S.2, S.3.1 and S.3.2} are applicable here.
    


  
  
