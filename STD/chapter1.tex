\chapter{Introduction}
This chapter contains general information regarding this document.

\section{Purpose}
%The purpose of this particular STD and its intended readership.
This document contains information regarding the transfer of the \applicationname{} from \projectauthor{} to the customer. It describes what is to be transferred, and how these items have been tested before the actual transfer. In chapter \ref{chap:problems}, the state of the transferred items is listed with regards to the user requirements of the URD \cite{urd}.

\section{Scope}
%Scope of the software. Identifies the product by name, explains what the software will do.
The \applicationname{} is an application designed and developed by \projectauthor{} for prof.dr.ir. P.D. Anderson, at the Eindhoven University of Technology. The application serves as an educational tool for anyone who wants to gain a deeper understanding of the process of mixing in general, and in particular for students at the TU/e.

\section{List of definitions}
%The definitions of all used terms, acronyms and abbreviations.
\subsection{Definitions}
\begin{description}
\item[Customer:] Prof.dr.ir. P.D. Anderson.
\item[Java Development Kit:] The software used to build Java applications.
\item[Ant:] Tool to build Java applications.
\item[Make:] A tool which controls the generation of executables and other non-source files of a program from the program's source files.
\end{description}

\subsection{Abbreviations}
\begin{tabular}{l|l}
2IP35 & The Software Engineering Project \\
ADD & Architectural Design Document \\
ATP & Acceptance Test Plan \\
JDK & Java Development Kit \\
TU/e  & Eindhoven University of Technology \\
\end{tabular}

\section{List of references}
\bibliography{../ref}

\section{Overview}
The remainder of this document starts with a description of the build procedures in chapter \ref{chap:build}. The installation procedures for the software built using these build procedures are described in chapter \ref{chap:install}. The list of items transferred to the customer is described in chapter \ref{chap:conf}. A summary of the test results obtained by executing the tests in the ATP \cite{atp} is provided in chapter \ref{chap:testreports}. Chapter \ref{chap:problems} lists some items that were not implemented satisfactorily. If any modifications were requested by the customer during the transfer phase, they are listed in chapter \ref{chap:changes}. If these were implemented during the transfer phase, they are documented in chapter \ref{chap:implemented}.