\chapter{Component description}
\label{chap:compdescr}
In this chapter we will describe all of the components that were indentified in chapter \ref{chap:systdesign} in detail.

%-------------------------------------------------------------------------------------------------------------------
\section{Server}
%-------------------------------------------------------------------------------------------------------------------
\subsection{Application Persistence}

\subsubsection*{Component identifier}
%A unique identifier
Server.ApplicationPersistence

\subsubsection*{Type}
%Task, procedure, package, program, file, ...
Database.

\subsubsection*{Purpose}
%Software requirements implemented
Provides functionality for the following software requirements:\\
SRQ9, SRQ10, SRQ11
%This are the requirements with labels AP-1, AP-2, AP-4

\subsubsection*{Function}
%What the component does.
This component is the database that contains the different mixers and their information. Mixers can be retrieved from this component for use within the Fortran Module. New mixers can be stored and old fixers can be removed. All the communication to execute these functions is done by the Application Persistence Communication, which is described in section 2.7.3 of the SRD \cite{srd}.

\subsubsection*{Subordinates}
%Child components (modules called, files composed of, classes used)
The Application Persistence does not have any child modules.

\subsubsection*{Dependencies}
%Components to be executed before/after, excluded operations during execution
The Application Persistence component depends on the Application Service component to receive requests from other components. For an overview of dependencies, see figure \ref{fig:compdependencies}.

\subsubsection*{Interfaces}
%Data and control flow in and out
This component it is only accessed through the Application Service component. The Application Persistence Communication is used for this purpose.  \todo{write more about this after consulting Ion}

\subsubsection*{Resources}
%Needed to perform the function
\todo{Ask Ion}

\subsubsection*{References}
%to other documents
The description of the specific requirements, mentioned in the section purpose, can be found in the SRD \cite{srd}.

\subsubsection*{Processing}
%Internal control and data flow
\todo{Follows from the component’s function and interfaces.}

\subsubsection*{Data}
%Internal data
This component is a database containing data that represents the different mixers.

%-------------------------------------------------------------------------------------------------------------------
\subsection{HTTP Server}

\subsubsection*{Component identifier}
%A unique identifier
Server.HTTPServer

\subsubsection*{Type}
%Task, procedure, package, program, file, ...
Program.

\subsubsection*{Purpose}
%Software requirements implemented
Provides functionality for the following software requirements:\\
SRQ1, SRQ3, SRQ6, SRQ101,
%This are the requirements with labels HTTP-1, AS-2, AS-6, returnParams

\subsubsection*{Function}
%What the component does.
The HTTP server is a piece of software that responds to requests by either serving a file from a static collection of content or responding with a dynamically build response. The former is used to serve the actual application, while the latter is used when the client requests a simulation, which goes through the HTTP server as well.

\subsubsection*{Subordinates}
%Child components (modules called, files composed of, classes used)
\todo{I have no idea whatsoever}

\subsubsection*{Dependencies}
%Components to be executed before/after, excluded operations during execution
The HTTP Server component doesn't depend on any other components. For an overview of dependencies, see figure \ref{fig:compdependencies}.

\subsubsection*{Interfaces}
%Data and control flow in and out
This component is accessed through the application service component, the application state component and the update application persistence component. The application service communication is used for the communication with the application service components. For communication with the two components on the client-side of the application, HTTP is used.
 \todo{write more about this after consulting Ion}

\subsubsection*{Resources}
%Needed to perform the function
\todo{Ask Ion}

\subsubsection*{References}
%to other documents
The description of the specific requirements, mentioned in the section purpose, can be found in the SRD \cite{srd}.

\subsubsection*{Processing}
%Internal control and data flow
\todo{Follows from the component’s function and interfaces.}

\subsubsection*{Data}
%Internal data
The HTTP server component contains a static collection of content files, that can be served to the client side of the application.

%-------------------------------------------------------------------------------------------------------------------

\subsection{Application Service}
\label{subsec:Appservice}

\subsection*{Component identifier}

Server.ApplicationService

\subsubsection*{Type}

Package

\subsubsection*{Purpose}
Provides functionality for the following requirements: \\
SRQ2, SRQ3, SRQ4, SRQ5, SRQ6, SRQ14 and SRQ15.
% AS-1, AS-2, AS-3, AS-5, AS-6, ASC-1, ASC-2

\subsubsection*{Function}
Whenever centralized data or other communication is required by the application running on a Client, this is done through the Application Service using the Application Service Communication channel.

\subsubsection*{Subordinates}
\todo{Child components (modules called, files composed of, classes used) - Ask Thom/Lasse}

\subsubsection*{Dependencies}
%\todo{Components to be executed before/after, excluded operations during execution}
\begin{description}
\item The Application Service depends on the Application Persistence component when the HTTP Server makes a request for data from the Application Persistence.
\item The Application Service depends on the Simulator Service component when the HTTP Server sends a protocol to be executed on the Fortran Module.
\item The Application Service depends on the HTTP Server component when the resulting concentration distribution has been calculated by the Fortran Module, and this results has to be send back to the client browser.
\end{description}
\noindent For an overview of dependencies, see figure \ref{fig:compdependencies}.

\subsubsection*{Interfaces}
\todo{Data and control flow in and out - Ask Ion}

\subsubsection*{Resources}
\todo{Needed to perform the function - Ask Ion}

\subsubsection*{References}
%\todo{to other documents}
For an in-depth description of the Application Service and the functionality described in purpose, see the SRD \cite{srd}.

\subsubsection*{Processing}
\todo{Internal control and data flow - Ask Ion}

\subsubsection*{Data}
\todo{Internal data - ask Ion}\\
none (?)

%-------------------------------------------------------------------------------------------------------------------
\section{Client}
%-------------------------------------------------------------------------------------------------------------------
\subsection{Layout}

\subsubsection*{Component identifier}
%A unique identifier
Client.Layout

\subsubsection*{Type}
%Task, procedure, package, program, file, ...
GUI.

\subsubsection*{Purpose}
%Software requirements implemented
Provides functionality for the following software requirements:\\
SRQ16, SRQ17, SRQ18, SRQ19, SRQ20, SRQ21, SRQ22, SRQ23, SRQ24, SRQ25, SRQ26, SRQ27, SRQ28, SRQ29, SRQ30, SRQ31, SRQ32, SRQ33, SRQ34, SRQ35, SRQ36, SRQ37, SRQ38, SRQ41, SRQ42, SRQ43, SRQ44, SRQ45, SRQ46, SRQ47, SRQ48, SRQ49, SRQ50, SRQ51, SRQ52, SRQ53, SRQ54, SRQ55, SRQ56, SRQ57, SRQ58, SRQ59, SRQ60, SRQ61, SRQ62, SSRQ63, SRQ64, SRQ65, SRQ66, SRQ67, SRQ68, SRQ69, SRQ70, SRQ71, SRQ72, SRQ73, SRQ74, SRQ75, SRQ76, SRQ77, SRQ78, SRQ79, SRQ80, SRQ81, SRQ82, SRQ83, SRQ84, SRQ85, SRQ86, SRQ87, SRQ88, SRQ89, SRQ90, SRQ91, SRQ92, SRQ93, SRQ94, SRQ95, SRQ96, SRQ97, SRQ98, SRQ102, SRQ103, SRQ104, SRQ105
%This are the requirements with labels: stepsize, CB-1, CB-2, CB-3, CB-4, CB-5, CB-6, CB-7, selgeomrec, selmixer, selgeomcir, selgeomjb, CB-8, prevsaved, CB-9, predef, CB-10, shapedraw, CB-12, circleshaped, squareshaped, CB-13, startMixing, CB-16, visResults, CB-16-2, CB-17, CP-0, CP-1, CP-2, CP-3, CP-4, CP-5, CP-6, CP-7, savedistr, savename, savebutton, CP-8, CP-9, CP-10, loaddistbutton, CP-11, loaddist, loadload, CP-12, CP-13, CP-14, loadpred CP-15, CP-16, listsaved, selectremove, suremessage, yesbutton, nobutton, nosaveddistr, CP-17, insuffrights, CP-18, CP-19, CP-21, CP-22, exportname, exportsave, CP-23, nameinuseexport, nomemoryexport, actualNameInUse, CP-24, CP-25, CP-26, CP-27, selectLoad, loadresults, displayLoad, CP-28, CP-29, CP-30, selLang, SSC-3, SSC-4, SSC-5, SSC-6

\subsubsection*{Function}
%What the component does.
This component is responsible for the GUI layout.

\subsubsection*{Subordinates}
%Child components (modules called, files composed of, classes used)
This component consists of the following files:
\begin{itemize}
\item fingerpaint.css
\item fingerpaint.html
\item fingerpaint.js
\end{itemize}

\subsubsection*{Dependencies}
%Components to be executed before/after, excluded operations during execution
The layout component depends on both the Application State and Update Application Persistence components. If the data inside one of these components changes, the layout component has to be updated accordingly. For an overview of dependencies, see figure \ref{fig:compdependencies}.

\subsubsection*{Interfaces}
%Data and control flow in and out
This component doesn't have any interfaces. It is part of the Client Browser, and any interfacing with this component is done throught the Application State and Update Application Persistence components.
 \todo{write more about this after consulting Ion}

\subsubsection*{Resources}
%Needed to perform the function
\todo{Ask Ion}

\subsubsection*{References}
%to other documents
The description of the specific requirements, mentioned in the section purpose, can be found in the SRD \cite{srd}.

\subsubsection*{Processing}
%Internal control and data flow
\todo{ask someone if javascript does processing}

\subsubsection*{Data}
%Internal data
This component doesn't keep any internal data.

%-------------------------------------------------------------------------------------------------------------------
\subsection{Update Application Persistence}

\subsubsection*{Component identifier}
%A unique identifier
Client.UpdateApplicationPersistence

\subsubsection*{Type}
%Task, procedure, package, program, file, ...
Program.

\subsubsection*{Purpose}
%Software requirements implemented
Provides functionality for the following software requirements:\\
SRQ12, SRQ13
%This are the requirements with labels: AP-5, AP-6

\subsubsection*{Function}
%What the component does.
This component allows for adding new mixers to and removing mixers from the application persistence, from the client side of the application.

\subsubsection*{Subordinates}
%Child components (modules called, files composed of, classes used)
\todo{I've no idea whatsover}

\subsubsection*{Dependencies}
%Components to be executed before/after, excluded operations during execution
The Update Application Persistence dependes on the layout component, since it needs the data that was inputted by the user in this component to perform its tasks. It also depends on the HTTP server, to send all its requests to the server side of the application. For an overview of dependencies, see figure \ref{fig:compdependencies}.

\subsubsection*{Interfaces}
%Data and control flow in and out
This component has an interface with the HTTP server. HTTP is used on this interface to send data to and receive data from the server side of the application. \todo{write more about this after consulting Ion}

\subsubsection*{Resources}
%Needed to perform the function
\todo{Ask Ion}

\subsubsection*{References}
%to other documents
The description of the specific requirements, mentioned in the section purpose, can be found in the SRD \cite{srd}.

\subsubsection*{Processing}
%Internal control and data flow
\todo{Follows from the component’s function and interfaces.}

\subsubsection*{Data}
%Internal data
\todo{This component doesn't keep any internal data.}
%-------------------------------------------------------------------------------------------------------------------
\subsection{Application State}
\label{subsec:Appservice}

\subsection*{Component identifier}
Server.Application State

\subsubsection*{Type}
Program

\subsubsection*{Purpose}
Provides functionality for the following requirements: \\
%Criteria : If there is any interaction required between the application state and Layout (accessing of storing information), the SRQ is listed in both components. If its 100% GUI, it's not here.
 SRQ17, SRQ19, SRQ21, SRQ23, SRQ24, SRQ25, SRQ26, SRQ27, SRQ28, SRQ29, SRQ30, SRQ31, SRQ32, SRQ33, SRQ39, SRQ40, SRQ41, SRQ42, SRQ43, SRQ44, SRQ45, SRQ46, SRQ47, SRQ48, SRQ49, SRQ50, SRQ51, SRQ52, SRQ53, SRQ54, SRQ56, SRQ58, SRQ61, SRQ62, SRQ63, SRQ64, SRQ65, SRQ67, SRQ68, SRQ70, SRQ73, SRQ76, SRQ77, SRQ78, SRQ79, SRQ80, SRQ81, SRQ83, SRQ85, SRQ86, SRQ87, SRQ89, SRQ91, SRQ92, SRQ94, SRQ95, SRQ96 and SRQ98.\\

% CB-1, CB-3, CB-5, CB-7, selgeomrec, selmixer, selgeomsq, selgeomcir, selgeomjb, CB-8, prevsaved, CB-9, predef, CB10, CB-14, CB-15, startMixing, CB-16, visResults, CB-16-2, CB-17, CP-0, CP-1, CP-2, CP-3, CP-4, CP-5, CP-6, CP-7, savedistr, savebutton, CP-9, CP-11, loaddist, loadload, CP-12, CP-13, loadpred, CP-15, listsaved, yesbutton, CP-17, insuffrights, CP-18, CP-19, CP-21, CP-22, exportsave, nameinuseexport, nomemoryexport, actualNameInUse, CP-25, CP-27, selectload, displayload, CP-28, CP-29, selLang


\subsubsection*{Function}
The Client Browser provides the user with a Graphical User Interface. All interactions of the user with the application are performed through this GUI. Whenever a user action on the GUI makes any change to the state of the application this change is kept track of, and stored in, the Application State. When any information stored in the Application State needs to displayed on the GUI it is accessed as well. For saving a file locally on the device the Application State is needed to communicate and forward this to the Client Persistent component to store it there. Finally, it also communicates mixing runs to the HTTP server component to forward it to the Fortran Module which does the computation.

\subsubsection*{Subordinates}
\todo{Child components (modules called, files composed of, classes used) - Ask Ion}

\subsubsection*{Dependencies}
%\todo{Components to be executed before/after, excluded operations during execution}
For saving a file locally on the device the Application State state depends on the Client Persistence to make this happen. To make any changes to the state of the application it depends on the Layout to issue the parameters for the new value. Finally, when the user wants to execute a mixing step/run, and the Layout communicates this to the Application State, it depends on the HTTP Server component to forward it to the Fortran Module which does the computation. \\

\noindent For an overview of dependencies, see figure \ref{fig:compdependencies}.

\subsubsection*{Interfaces}
\todo{Data and control flow in and out - Ask Ion}

\subsubsection*{Resources}
\todo{Needed to perform the function - Ask Ion}

\subsubsection*{References}
%\todo{to other documents}
For an in-depth description of the Application Service and the functionality described in purpose, see the SRD \cite{srd}.

\subsubsection*{Processing}
\todo{Internal control and data flow - Ask Ion}

\subsubsection*{Data}
\todo{Internal data - ask Ion}


