\chapter{Component description}
\label{chap:compdescr}
In this chapter we will describe all of the components that were indentified in chapter \ref{chap:systdesign} in detail.

%-------------------------------------------------------------------------------------------------------------------
\section{Server}
%-------------------------------------------------------------------------------------------------------------------

\subsection{Fortran Module}

\subsubsection*{Component Identifier}
Server.FortranModule

\subsubsection*{Type}
The Fortran Module consists of a single procedure.

\subsubsection*{Purpose}
To enable \projectname to function, the requirements SQR111 and SQR112 (as described in the SRD \cite{srd}) should be fulfilled.
%FM-1 and FM-2

\subsubsection*{Function}
The Fortran Module performs the calculations to calculate how the mixing distribution changes as a mixing step is applied.

\subsubsection*{Subordinates}
The Fortran Module does not have any known child components. This is because the module was given to \projectauthor as a black box.

\subsubsection*{Dependencies}
The Fortran Module does not have dependencies with relation to other components. This is because it uses its own data, along with the parameters for computations and only interfaces with its parent module (the Simulation Service).

\subsubsection*{Interfaces}
The Fortran Module has a single interface with the Simulation Service. This interface is the procedure call of the module in Fortran. This call uses a concentration distribution and a mixing step as arguments and returns a concentration distribution. For more detail regarding this interface, refer to  chapter \ref{chap:systcontext}.

\subsubsection*{Resources}
\todo{ask Ion}

\subsubsection*{References}
The requirements SQR111 and SQR112 can be found in subsection 3.1.10 of the SRD \cite{srd}.

\subsubsection*{Processing}
As the implementation of the Fortran Module is beyond \projectname, the internal flow is irrelevant.

\subsubsection*{Data}
As the implementation of the Fortran Module is beyond \projectname, the internal data is irrelevant.

%-------------------------------------------------------------------------------------------------------------------
\subsection{Simulation Service}

\subsubsection*{Component Identifier}
Server.SimulationService

\subsubsection*{Type}
The Simulation Service is a program.

\subsubsection*{Purpose}
For the component to successful, it must fulfull requirements SRQ99-101 and requirements SRQ106-110 (as described in the SRD \cite{srd}).
%sendParams, readParams, returnParams, execMixing and all SS-versions

\subsubsection*{Function}
The Simulation Service is meant to simulate a mixing protocol on the server and compute how the concentration distribution changes when such a protocol is applied.

\subsubsection*{Subordinates}
The Simulation Service uses the Fortran Module to perform the technical computations.

\subsubsection*{Dependencies}
The Simulation Service depends on the Fortran Module for technical calculations.

\subsubsection*{Interfaces}
The Simulation Service uses the Fortran Module's interface for communications to it. The Simulator Service itself is accessible through the Simulator Service Communication.

\subsubsection*{Resources}
\todo{ask Ion}

\subsubsection*{References}
The requirements SRQ99-101 can be found in subsection 3.1.8 of the SRD \cite{srd}.

\subsubsection*{Processing}
For processing requests, the Simulation Service keeps each request separate. For each request, the protocol is split up in single steps and these are simulated in sequence by the Fortran Module.

\subsubsection*{Data}
The Simulation Service has no internal data other than the protocol and distribution defined by the request. This data is modified as each step of the protocol is processed by the Fortran Module,and finally returned when all steps are analysed.

%-------------------------------------------------------------------------------------------------------------------
\subsection{Application Persistence}

\subsubsection*{Component identifier}
%A unique identifier
Server.ApplicationPersistence

\subsubsection*{Type}
%Task, procedure, package, program, file, ...
Database.

\subsubsection*{Purpose}
%Software requirements implemented
Provides functionality for the following software requirements:\\
SRQ9, SRQ10, SRQ11
%This are the requirements with labels AP-1, AP-2, AP-4

\subsubsection*{Function}
%What the component does.
This component is the database that contains the different mixers and their information. Mixers can be retrieved from this component for use within the Fortran Module. New mixers can be stored and old fixers can be removed. All the communication to execute these functions is done by the Application Persistence Communication, which is described in section 2.7.3 of the SRD \cite{srd}.

\subsubsection*{Subordinates}
%Child components (modules called, files composed of, classes used)
The Application Persistence does not have any child modules.

\subsubsection*{Dependencies}
%Components to be executed before/after, excluded operations during execution
The Application Persistence component depends on the Application Service component to receive requests from other components. For an overview of dependencies, see figure \ref{fig:compdependencies}.

\subsubsection*{Interfaces}
%Data and control flow in and out
This component it is only accessed through the Application Service component. The Application Persistence Communication is used for this purpose.  \todo{write more about this after consulting Ion}

\subsubsection*{Resources}
%Needed to perform the function
\todo{Ask Ion}

\subsubsection*{References}
%to other documents
The description of the specific requirements, mentioned in the section purpose, can be found in the SRD \cite{srd}.

\subsubsection*{Processing}
%Internal control and data flow
\todo{Follows from the component’s function and interfaces.}

\subsubsection*{Data}
%Internal data
This component is a database containing data that represents the different mixers.

%-------------------------------------------------------------------------------------------------------------------
\subsection{HTTP Server}

\subsubsection*{Component identifier}
%A unique identifier
Server.HTTPServer

\subsubsection*{Type}
%Task, procedure, package, program, file, ...
Program.

\subsubsection*{Purpose}
%Software requirements implemented
Provides functionality for the following software requirements:\\
SRQ1, SRQ3, SRQ6, SRQ101, 
%This are the requirements with labels HTTP-1, AS-2, AS-6, returnParams

\subsubsection*{Function}
%What the component does.
The HTTP server is a piece of software that responds to requests by either serving a file from a static collection of content or responding with a dynamically build response. The former is used to serve the actual application, while the latter is used when the client requests a simulation, which goes through the HTTP server as well.

\subsubsection*{Subordinates}
%Child components (modules called, files composed of, classes used)
\todo{I have no idea whatsoever}

\subsubsection*{Dependencies}
%Components to be executed before/after, excluded operations during execution
The HTTP Server component doesn't depend on any other components. For an overview of dependencies, see figure \ref{fig:compdependencies}.

\subsubsection*{Interfaces}
%Data and control flow in and out
This component is accessed through the application service component, the application state component and the update application persistence component. The application service communication is used for the communication with the application service components. For communication with the two components on the client-side of the application, HTTP is used.
 \todo{write more about this after consulting Ion}

\subsubsection*{Resources}
%Needed to perform the function
\todo{Ask Ion}

\subsubsection*{References}
%to other documents
The description of the specific requirements, mentioned in the section purpose, can be found in the SRD \cite{srd}.

\subsubsection*{Processing}
%Internal control and data flow
\todo{Follows from the component’s function and interfaces.}

\subsubsection*{Data}
%Internal data
The HTTP server component contains a static collection of content files, that can be served to the client side of the application.

%-------------------------------------------------------------------------------------------------------------------
\section{Client}
%-------------------------------------------------------------------------------------------------------------------
\subsection{Layout}

\subsubsection*{Component identifier}
%A unique identifier
Client.Layout

\subsubsection*{Type}
%Task, procedure, package, program, file, ...
GUI.

\subsubsection*{Purpose}
%Software requirements implemented
Provides functionality for the following software requirements:\\
SRQ16, SRQ17, SRQ18, SRQ19, SRQ20, SRQ21, SRQ22, SRQ23, SRQ24, SRQ25, SRQ26, SRQ27, SRQ28, SRQ29, SRQ30, SRQ31, SRQ32, SRQ33, SRQ34, SRQ35, SRQ36, SRQ37, SRQ38, SRQ41, SRQ42, SRQ43, SRQ44, SRQ45, SRQ46, SRQ47, SRQ48, SRQ49, SRQ50, SRQ51, SRQ52, SRQ53, SRQ54, SRQ55, SRQ56, SRQ57, SRQ58, SRQ59, SRQ60, SRQ61, SRQ62, SSRQ63, SRQ64, SRQ65, SRQ66, SRQ67, SRQ68, SRQ69, SRQ70, SRQ71, SRQ72, SRQ73, SRQ74, SRQ75, SRQ76, SRQ77, SRQ78, SRQ79, SRQ80, SRQ81, SRQ82, SRQ83, SRQ84, SRQ85, SRQ86, SRQ87, SRQ88, SRQ89, SRQ90, SRQ91, SRQ92, SRQ93, SRQ94, SRQ95, SRQ96, SRQ97, SRQ98, SRQ102, SRQ103, SRQ104, SRQ105
%This are the requirements with labels: stepsize, CB-1, CB-2, CB-3, CB-4, CB-5, CB-6, CB-7, selgeomrec, selmixer, selgeomcir, selgeomjb, CB-8, prevsaved, CB-9, predef, CB-10, shapedraw, CB-12, circleshaped, squareshaped, CB-13, startMixing, CB-16, visResults, CB-16-2, CB-17, CP-0, CP-1, CP-2, CP-3, CP-4, CP-5, CP-6, CP-7, savedistr, savename, savebutton, CP-8, CP-9, CP-10, loaddistbutton, CP-11, loaddist, loadload, CP-12, CP-13, CP-14, loadpred CP-15, CP-16, listsaved, selectremove, suremessage, yesbutton, nobutton, nosaveddistr, CP-17, insuffrights, CP-18, CP-19, CP-21, CP-22, exportname, exportsave, CP-23, nameinuseexport, nomemoryexport, actualNameInUse, CP-24, CP-25, CP-26, CP-27, selectLoad, loadresults, displayLoad, CP-28, CP-29, CP-30, selLang, SSC-3, SSC-4, SSC-5, SSC-6

\subsubsection*{Function}
%What the component does.
This component is responsible for the GUI layout.

\subsubsection*{Subordinates}
%Child components (modules called, files composed of, classes used)
This component consists of the following files:
\begin{itemize}
\item fingerpaint.css
\item fingerpaint.html
\item fingerpaint.js
\end{itemize}

\subsubsection*{Dependencies}
%Components to be executed before/after, excluded operations during execution
The layout component depends on both the Application State and Update Application Persistence components. If the data inside one of these components changes, the layout component has to be updated accordingly. For an overview of dependencies, see figure \ref{fig:compdependencies}.

\subsubsection*{Interfaces}
%Data and control flow in and out
This component doesn't have any interfaces. It is part of the Client Browser, and any interfacing with this component is done throught the Application State and Update Application Persistence components.
 \todo{write more about this after consulting Ion}

\subsubsection*{Resources}
%Needed to perform the function
\todo{Ask Ion}

\subsubsection*{References}
%to other documents
The description of the specific requirements, mentioned in the section purpose, can be found in the SRD \cite{srd}.

\subsubsection*{Processing}
%Internal control and data flow
\todo{ask someone if javascript does processing}

\subsubsection*{Data}
%Internal data
This component doesn't keep any internal data.

%-------------------------------------------------------------------------------------------------------------------
\subsection{Client Persistence}

\subsubsection*{Component Identifier}
Client.ClientPersistence

\subsubsection*{Type}
The Client Persistence is a set of files stored on the system of the client.

\subsubsection*{Purpose}
For the component to successful, it must fulfull requirements SRQ46-799 and requirements SRQ84-97 (as described in the SRD \cite{srd}).
%The entirety of 3.1.7 except for the exports.

\subsubsection*{Function}
The Client Persistence is responsible for storing results and self-defined protocols and distributions.

\subsubsection*{Subordinates}
The Client Persistence is composed of files representing mixing protocols and mixing distributions.

\subsubsection*{Dependencies}
The Client Persistence depends on stable data to store. Therefore, the Application State should provide these, aswell as not modify them during the saving.

\subsubsection*{Interfaces}
The Client Persistence uses the browser's native interface for storing. The Application State uses the Client Persistence interface for storing and retrieving files.

\subsubsection*{Resources}
\todo{ask Ion}

\subsubsection*{References}
The requirements SRQ99-110 can be found insubsection 3.1.7 of the SRD \cite{srd}.

\subsubsection*{Processing}
As the Client Persistence merely consists of files, there is no internal processing.

\subsubsection*{Data}
The data stored in the Client Persistence is composed of files representing mixing protocols and mixing distributions.

%-------------------------------------------------------------------------------------------------------------------
\subsection{Update Application Persistence}

\subsubsection*{Component identifier}
%A unique identifier
Client.UpdateApplicationPersistence

\subsubsection*{Type}
%Task, procedure, package, program, file, ...
Program.

\subsubsection*{Purpose}
%Software requirements implemented
Provides functionality for the following software requirements:\\
SRQ12, SRQ13
%This are the requirements with labels: AP-5, AP-6

\subsubsection*{Function}
%What the component does.
This component allows for adding new mixers to and removing mixers from the application persistence, from the client side of the application.

\subsubsection*{Subordinates}
%Child components (modules called, files composed of, classes used)
\todo{I've no idea whatsover}

\subsubsection*{Dependencies}
%Components to be executed before/after, excluded operations during execution
The Update Application Persistence dependes on the layout component, since it needs the data that was inputted by the user in this component to perform its tasks. It also depends on the HTTP server, to send all its requests to the server side of the application. For an overview of dependencies, see figure \ref{fig:compdependencies}.

\subsubsection*{Interfaces}
%Data and control flow in and out
This component has an interface with the HTTP server. HTTP is used on this interface to send data to and receive data from the server side of the application. \todo{write more about this after consulting Ion}

\subsubsection*{Resources}
%Needed to perform the function
\todo{Ask Ion}

\subsubsection*{References}
%to other documents
The description of the specific requirements, mentioned in the section purpose, can be found in the SRD \cite{srd}.

\subsubsection*{Processing}
%Internal control and data flow
\todo{Follows from the component’s function and interfaces.}

\subsubsection*{Data}
%Internal data
\todo{This component doesn't keep any internal data.}
