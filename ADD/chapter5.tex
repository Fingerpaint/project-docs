\makeatletter
\newcommand\addtosrqlist[1]{%
  \def\tempa{}%
  \@for\reserved:=#1\do{%
    \ifx\tempa\empty\else, \fi SRQ\srqref{\reserved}%
    \edef\tempa{something}%
  }
}
\makeatother

\chapter{Component description}
\label{chap:compdescr}
In this chapter we describe all of the components that were identified in chapter \ref{chap:systdesign} in detail. For every component, we give an identifier and the type, purpose, function, subordinates, dependencies and interfaces of the component. Furthermore, the internal control flow and internal data flow are described.

%-------------------------------------------------------------------------------------------------------------------
\section{Server}
%-------------------------------------------------------------------------------------------------------------------
\subsection{Fortran Module}
\label{subsec:fortranmodule}

\subsubsection*{Component Identifier}
\RTMSFM{}

\subsubsection*{Type}
Program.

\subsubsection*{Purpose}
Provides functionality for the following software requirements:

\noindent \addtosrqlist{CB-17,execMixing,returnMixing,SS-4,FM-1,FM-2,FM-3}

\subsubsection*{Function}
The Fortran Module calculates how the mixing distribution changes as a mixing step is applied.

\subsubsection*{Subordinates}
The Fortran Module does not have any known child components. This is because the module was given to \projectauthor{} as a black box.

\subsubsection*{Dependencies}
The Fortran Module does not have dependencies in relation to other components. This is because it uses its own data, along with the parameters for computations and only interfaces with its parent module (the Simulation Service).

\subsubsection*{Interfaces}
The Fortran Module has a single interface with the Simulation Service. This interface is the procedure call of the module in Fortran. This call receives a concentration distribution, a geometry, mixer and a mixing step as arguments and returns a concentration distribution and the segregation factor. For more detail regarding this interface, refer to chapter \ref{chap:systcontext}.

\subsubsection*{References}
The description of the specific requirements, mentioned in the ``Purpose''-section, can be found in the SRD \cite{srd}.

\subsubsection*{Processing}
As the implementation of the Fortran Module is beyond the scope of project \projectname{}, the internal flow is irrelevant.

\subsubsection*{Data}
As the implementation of the Fortran Module is beyond the scope of project \projectname{}, the internal data is irrelevant.

%-------------------------------------------------------------------------------------------------------------------
\subsection{Simulator Service}
\label{subsec:simulatorservice}

\subsubsection*{Component Identifier}
\RTMSSS{}

\subsubsection*{Type}
Program.

\subsubsection*{Purpose}
Provides functionality for the following software requirements:

\noindent \addtosrqlist{CB-17,sendParams,returnParams,execMixing,returnMixing,SS-2,SS-4}

\subsubsection*{Function}
The purpose of the Simulator Service is to simulate a mixing run on the server and compute how the concentration distribution changes when such the mixing protocol from this run is applied.

\subsubsection*{Subordinates}
The Simulator Service is composed of multiple Java classes, located in two packages. This is because one part is located on the client side, while another part resides on the server (calling the Fortran module, that runs on the server). The structure of these packages is shown below:
\begin{itemize}
	\item nl.tue.fingerpaint.server.simulator
		\begin{itemize}
			\item NativeCommunicator.java
			\item SimulatorServiceImpl.java
		\end{itemize}
	
	\item nl.tue.fingerpaint.client.simulator
		\begin{itemize}
			\item Simulation.java
			\item SimulationResult.java
			\item SimulatorService.java
			\item SimulatorServiceAsync.java
		\end{itemize}
\end{itemize}

\subsubsection*{Dependencies}
The Simulator Service depends on the Fortran Module for technical calculations.

\subsubsection*{Interfaces}
The Simulator Service uses the Fortran Module's interface to communicate with it. That interface is described in subsection \ref{subsec:fortranmodule}, more specifically the section ``Interfaces''. This communication is done via C code, which in turn calls the Fortran function.

The Simulator Service also receives information regarding mixing runs that have to be executed from the Application Service, and it sends the resulting distribution back to the Application Service. This communication is done by the Simulator Service Communication, which uses JNI.

\subsubsection*{References}
The description of the specific requirements, mentioned in the ``Purpose''-section, can be found in the SRD \cite{srd}.

\subsubsection*{Processing}
For processing requests, the Simulator Service keeps each request separate. For each request, the protocol is split up in single steps, which are separately sent to the Fortran Module.

\subsubsection*{Data}
The Simulator Service has no internal data other than the protocol and distribution defined by the request. This data is modified as each step of the protocol is processed by the Fortran Module and finally returned when all steps are analysed. Note that the concentration distribution may be duplicated a number of times, when intermediate results are requested.

%-------------------------------------------------------------------------------------------------------------------
\subsection{Application Persistence}
\label{subsec:apppersistence}

\subsubsection*{Component Identifier}
%A unique identifier
\RTMSAP{}

\subsubsection*{Type}
%Task, procedure, package, program, file, ...
Database.

\subsubsection*{Purpose}
%Software requirements implemented
Provides functionality for the following software requirements:

\noindent \addtosrqlist{AS-1,AS-2,AS-3,AS-5,AS-6,AS-7,APC-1,APC-2,AP-1,AP-2,AP-5,AP-6,CP-0}

\subsubsection*{Function}
%What the component does.
This component is the database that contains the different mixers and their information. Mixers can be retrieved from this component for use within the Fortran Module. All the communication to execute these functions is done by the Application Persistence Communication, which is described in section \ref*{SRD-sec:applicationpersistence} of the SRD \cite{srd}.

\subsubsection*{Subordinates}
%Child components (modules called, files composed of, classes used)
The Application Persistence does not have any child components.

\subsubsection*{Dependencies}
%Components to be executed before/after, excluded operations during execution
The Application Persistence component does not depend on any other components.

\subsubsection*{Interfaces}
%Data and control flow in and out
This component is accessed through the Application Service component. It receives requests to return certain mixers from the Application Service and sends the requested data to the Application Service. The Application Persistence Communication, which consists of Java calls, is used for this purpose.

\subsubsection*{References}
%to other documents
The description of the specific requirements, mentioned in the ``Purpose''-section, can be found in the SRD \cite{srd}.

\subsubsection*{Processing}
%Internal control and data flow
No processing of data is done in this component.

\subsubsection*{Data}
%Internal data
This component is a database containing data that represents the different mixers.

%-------------------------------------------------------------------------------------------------------------------
\subsection{HTTP Server}
\label{subsec:httpserver}

\subsubsection*{Component Identifier}
%A unique identifier
\RTMSHS{}

\subsubsection*{Type}
%Task, procedure, package, program, file, ...
Program.

\subsubsection*{Purpose}
%Software requirements implemented
Provides functionality for the following software requirements:

\noindent \addtosrqlist{HTTP-1,CB-17,SS-2}

\subsubsection*{Function}
%What the component does.
The HTTP server is a piece of software that responds to requests by serving a file from a static collection of content. This is used to serve the application on the client side.

\subsubsection*{Subordinates}
%Child components (modules called, files composed of, classes used)
The HTTP Server does not have any known child components. This is because this component was not built by \projectauthor.

\subsubsection*{Dependencies}
%Components to be executed before/after, excluded operations during execution
The HTTP Server component does not depend on any other components.

\subsubsection*{Interfaces}
%Data and control flow in and out
This component is accessed through the Application State Component, which sends requests for web pages to the HTTP Server. The HTTP server responds to these request by serving the requested file(s). All this communication is done through HTTP.

\subsubsection*{References}
%to other documents
The description of the specific requirements, mentioned in the ``Purpose''-section, can be found in the SRD \cite{srd}.

\subsubsection*{Processing}
%Internal control and data flow
As the implementation of the HTTP server component is beyond the scope of project \projectname, the internal flow is irrelevant.

\subsubsection*{Data}
%Internal data
The HTTP server component contains a static collection of content files, that can be served to the client side of the application.

%-------------------------------------------------------------------------------------------------------------------
\subsection{Application Service}
\label{subsec:appservice}

\subsection*{Component Identifier}
\RTMSAS{}

\subsubsection*{Type}
Program.

\subsubsection*{Purpose}
Provides functionality for the following software requirements:

\noindent \addtosrqlist{AS-1,AS-2,AS-3,AS-5,AS-6,AS-7,AP-1,AP-2,AP-5,AP-6,ASC-1,ASC-2,CB-17,sendParams,returnParams,SS-2}

\subsubsection*{Function}
Whenever centralized data or other communication is required by the application running on a Client, this is done through the Application Service using the Application Service Communication channel.

\subsubsection*{Subordinates}
%Child components (modules called, files composed of, classes used) - Ask Thom/Lasse
This component is composed of three packages, which contain Java classes. The structure of the packages is shown below:
\begin{itemize}
	\item nl.tue.fingerpaint.client.serverdata
		\begin{itemize}
			\item ServerDataCache.java
			\item ServerDataService.java
			\item ServerDataServiceAsync.java
		\end{itemize}
	
	\item nl.tue.fingerpaint.server
		\begin{itemize}
			\item ServerDataServiceImpl.java
		\end{itemize}
	
	\item nl.tue.fingerpaint.shared
		\begin{itemize}
			\item ServerDataResult.java
		\end{itemize}
\end{itemize}

\subsubsection*{Dependencies}
%\todo{Components to be executed before/after, excluded operations during execution}
The Application Service depends on the Application Persistence component when the Application State  component makes a request for data from the Application Persistence. Furthermore, the Application Service depends on the Simulator Service component when the Application State sends a mixing run to be executed on the Fortran Module.

\subsubsection*{Interfaces}
The Application Service has an interface with the Application Persistence component. This communication is done through a Java call. The Application Service sends requests to the Application persistence to return a mixer. It receives data belonging to mixers from the Application Persistence.

The Application Service also has an interface with the Simulator Service component. This communication is done through JNI. The application service sends information belonging to a mixing run to the Simulator Service. It receives the result of the calculated mixing run from the Simulator Service.

\subsubsection*{References}
%\todo{to other documents}
For an in-depth description of the Application Service and the functionality described in the ``Purpose''-section, see the SRD \cite{srd}.

\subsubsection*{Processing}
Basically, the only thing the Application Service does is passing data through from the Application State to the Simulation Service or Application Persistence, and vice versa. This would imply that no processing is done. This is not entirely true however, as some data is compressed before sending it to the client. Thus, the Application Service may format the data before sending it to the client or the back end; see section \ref{sec:datacomp} for more information.

\subsubsection*{Data}
The Application Service does not have any internal data. The only data residing in this component is the data passing through with requests.

%-------------------------------------------------------------------------------------------------------------------
\section{Client}
%-------------------------------------------------------------------------------------------------------------------
\subsection{Layout}\label{subsec:layout}

\subsubsection*{Component Identifier}
%A unique identifier
\RTMCL{}

\subsubsection*{Type}
%Task, procedure, package, program, file, ...
GUI.

\subsubsection*{Purpose}
%Software requirements implemented
Provides functionality for the following software requirements:

\noindent \addtosrqlist{shapedraw,CB-12,circleshaped,squareshaped,CB-13,CB-2,CB-3,CB-4,CB-5,CB-6,CB-7,selgeomrec,selmixer,selgeomsq,selgeomcir,selgeomjb,CB-8,prevsaved,CB-9,predef,CB-10,stepsize,protocolrepeat,CB-14,CB-15,startMixing,CB-16,CB-17,visResults,CB-16-2,CP-1,CP-2,CP-3,CP-4,CP-5,CP-6,CP-7,savedistr,savename,CP-9,CP-10,loaddistbutton,loadlist,loadload,CP-14,removeinit,listsaved,selectremove,suremessage,nosaveddistr,insuffrights,distimg,gengraph,gendist,exportname,initload,selectload,displayload,CP-28,selLang,SSC-3,SSC-4,SSC-6,englan,dutlan,NONF-6,NONF-7,NONF-8,NONF-9,NONF-10,NONF-11,NONF-12,NONF-13} 

\subsubsection*{Function}
%What the component does.
This component is responsible for the GUI layout.

\subsubsection*{Subordinates}
%Child components (modules called, files composed of, classes used)
The layout is generated by GWT, which means that there are no files that actually contain the layout. However, there are files that control the output and which are parsed/used by GWT to generate the actual layout. These files are listed below:
\begin{itemize}
	\item \makebox[0.3\textwidth][l]{fingerpaint.nocache.js}  \emph{generated by GWT: will load the application}
	\item \makebox[0.3\textwidth][l]{fingerpaint.css}         \emph{used by GWT to generate inline CSS}
	\item \makebox[0.3\textwidth][l]{Fingerpaint.html}        \emph{base HTML, populated dynamically with JavaScript}
\end{itemize}

\subsubsection*{Dependencies}
%Components to be executed before/after, excluded operations during execution
The layout component depends on the Application State component. If the data inside this component changes, the layout component has to be updated accordingly.

\subsubsection*{Interfaces}
%Data and control flow in and out
This component does not have any interfaces. It is part of the Client Browser, and any interfacing with this component is done through the Application State component.

\subsubsection*{References}
%to other documents
The description of the specific requirements, mentioned in the ``Purpose''-section, can be found in the SRD \cite{srd}.

\subsubsection*{Processing}
%Internal control and data flow
In this component data regarding information that was input by the user is sent to the Application State component. The data is processed there. This component also receives data from the Application State component regarding the contents of the GUI. The Layout component uses this data to update the GUI.

\subsubsection*{Data}
%Internal data
This component does not keep any internal data.

%-------------------------------------------------------------------------------------------------------------------
\subsection{Client Persistence}

\subsubsection*{Component Identifier}
\RTMCCP{}

\subsubsection*{Type}
The Client Persistence is a set of files stored on the system of the client.

\subsubsection*{Purpose}
Provides functionality for the following software requirements:

\noindent \addtosrqlist{CP-1,CP-2,CP-3,CP-4,CP-5,CP-6,CP-7,savebutton,CP-14,displayload,CP-28}

\subsubsection*{Function}
The Client Persistence is responsible for storing results and user-defined protocols and distributions.

\subsubsection*{Subordinates}
The Client Persistence does not have any child components.


\subsubsection*{Dependencies}
The Client Persistence component does not depend on any other components.

\subsubsection*{Interfaces}
The Application State uses the Client Persistence interface for storing and retrieving files. This communication is done through a Java-call.
This component is accessed through the Application State component. It receives requests to add, remove or return distributions or protocols from the Application State. If it receives a request to return a file, it will send the requested data to the Application State. The communication between these two components consists of Java calls.

\subsubsection*{References}
The description of the specific requirements, mentioned in the ``Purpose''-section, can be found in the SRD \cite{srd}.

\subsubsection*{Processing}
No processing of data is done in this component.

\subsubsection*{Data}
The data stored in the Client Persistence is a collection of files representing mixing protocols and concentration distributions.

%-------------------------------------------------------------------------------------------------------------------
% The Update Application Persistence will not be implemented and it therefore not important for the ADD.

%\subsection{Update Application Persistence}

%\subsubsection*{Component Identifier}
%A unique identifier
%\RTMCUAP{}

%\subsubsection*{Type}
%Task, procedure, package, program, file, ...
%Program.

%\subsubsection*{Purpose}
%Software requirements implemented
%Provides functionality for the following software requirements:

%\noindent \addtosrqlist{CP-1,CP-2,CP-3,CP-4,CP-5,CP-6,CP-7,savebutton,CP-14,displayload,CP-28}

%\subsubsection*{Function}
%What the component does.
%This component allows for adding new mixers to and removing mixers from the application persistence, from the client side of the application.

%\subsubsection*{Subordinates}
%Child components (modules called, files composed of, classes used)
%\todo{This component does not exist yet. It is subordinates will follow later on.}

%\subsubsection*{Dependencies}
%Components to be executed before/after, excluded operations during execution
%The Update Application Persistence depends on the layout component, since it needs the data that was input by the user in this component to perform its tasks. It also depends on the Application Service to send all its requests to the Application Persistence. Finally, it depends on the HTTP server to retrieve the files with web page content that needs to be displayed.

%\subsubsection*{Interfaces}
%Data and control flow in and out
%This component has an interface with the HTTP server. HTTP is used to retrieve the web content files from the HTTP server. It also has an interface with the Application Service. Java calls are used on this interface to send data to and receive data from the server side of the application. This component receives data from the Application Service regarding all mixers that are stored in the client persistence. It sends data to the Application Service regarding new mixers to add or to remove mixers.

%\subsubsection*{References}
%to other documents
%The description of the specific requirements, mentioned in the ``Purpose''-section, can be found in the SRD \cite{srd}.

%\subsubsection*{Processing}
%Internal control and data flow
%This component receives data that is input by the user from the Layout component. If this data concerns requests to add or remove a mixer, the data will be sent to the Application Service. This component also receives information regarding already existing mixers from the Application Service, as described in the section `Interfaces'. This data is sent to the Layout component to display it on the client device.
%\todo{Reconsider after this component has been build.}

%\subsubsection*{Data}
%Internal data
%This component does not keep any internal data.
%\todo{Reconsider after this component has been build.}

%-------------------------------------------------------------------------------------------------------------------
\subsection{Application State}
\label{subsec:appstate}

\subsection*{Component Identifier}
\RTMCAS{}

\subsubsection*{Type}
Program

\subsubsection*{Purpose}
Provides functionality for the following requirements:

\noindent \addtosrqlist{CP-1,CP-2,CP-3,CP-4,CP-5,CP-6,CP-7,savedistr,savename,savebutton,CP-14}

\subsubsection*{Function}
The Client Browser provides the user with a Graphical User Interface. All interactions of the user with the application are performed through this GUI. Whenever a user action on the GUI makes any change to the state of the application this change is stored in the Application State. When any information stored in the Application State needs to be displayed on the GUI it is accessed as well. For saving a file locally on the device the Application State is needed to communicate and forward this to the Client Persistent component to store it there. Finally, it also communicates mixing runs to the Application Service component to forward it to the Fortran Module which does the computation.


\subsubsection*{Subordinates}
%\todo{Child components (modules called, files composed of, classes used)}
In a similar manner as the layout component, as described in subsection \ref{subsec:layout}, the components regarding the application state are parsed by GWT. 
For this purpose, several classes from the following packages are used:
\begin{itemize}
	\item nl.tue.fingerpaint.client.model
	\item nl.tue.fingerpaint.client.serverdata
	\item nl.tue.fingerpaint.client.simulator
	\item nl.tue.fingerpaint.client.storage
\end{itemize}

\subsubsection*{Dependencies}
%\todo{Components to be executed before/after, excluded operations during execution}
For saving a file locally on the device the Application State state depends on the Client Persistence to make this happen. To make any changes to the state of the application it depends on the Layout to issue the parameters for the new value. To retrieve the files with web content that need to be displayed, it depends on the HTTP Server. Finally, when the user wants to execute a mixing step/run, and the Layout communicates this to the Application State, it depends on the Application Service component to forward it to the Fortran Module which does the computation.


\subsubsection*{Interfaces}
%\todo{Data and control flow in and out}
The Application State can send information that needs to be stored locally on the device to the Client Persistence. It also sends information about a mixing run that has to be executed to the Application Service, and receives information regarding the end results of a mixing run from the Application Service. All this communication is done through Java calls. Finally, it also receives files with web content from the HTTP Server, which is done through HTTP.


\subsubsection*{References}
%\todo{to other documents}
For an in-depth description of the Application Service and the functionality described in the ``Purpose''-section, see the SRD \cite{srd}.

\subsubsection*{Processing}
%\todo{Internal control and data flow}
The Application State converts a mixing run call to a JSON format to send to the Application Service. When files need to be saved it converts the data to String format.

\subsubsection*{Data}
The data stored in the Application State are the variables entered by the user in the Layout component.
