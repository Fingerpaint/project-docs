\chapter{System context}
\label{chap:systcontext}
  
\section{Interface with the fortran visualisation server}
Computations that need to be done by \applicationname are offloaded to the fortran module as described in section 2.7.4 of the SRD \cite{srd}. Communications to this module are done by calling its procedure in fortran. This procedure simulates one mixing step and uses the following parameters:\\
\texttt{geometry}: This is a number, each number stands for a different geometry \todo{link each used geometry to a number}\\
\texttt{matrix}: This number represents which mixer is used from those available for the given geometry \todo{link each used mixer to a number}\\
\texttt{distribution}: A vector that describes the concentration distribution of the canvas.\\
\texttt{len\_distribution}: A meta-parameter that represents the length of the distribution vector described above.\\
\texttt{step}: Size of the mixing step that is to be simulated.\\
\texttt{stepid}: Describes the type of step that should be simulated (e.g. for a rectangular geometry it contains which of the walls is moved and in which direction). \\
\texttt{len\_stepid}: A meta-parameter that represents the length of the stepid.\\
\texttt{segregration}: A value that defines how well a mixture is separated, a lower segregation means the mixture is well-mixed.\\
Of these parameters, \texttt{geometry},\texttt{ matrix}, \texttt{distribution}, \texttt{len\_distribution}, \texttt{step}, \texttt{stepid} and \texttt{len\_stepid} are used as input parameters, and \texttt{distribution} and \texttt{segregation} are used as output parameters.



