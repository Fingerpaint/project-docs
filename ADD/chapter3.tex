\chapter{System context}
\label{chap:systcontext}
In this chapter, the context of the system is discussed. With this, we mean external components that are used by the \applicationname{}.

\section{Interface with the Fortran visualisation server}
Computations that need to be done by the \applicationname{} are offloaded to the Fortran module as described in section 2.7.4 of the SRD \cite{srd}. Communication with this module are done by calling its procedure in Fortran. This procedure simulates one mixing step and uses the following parameters:

\begin{center}
\LTXtable{\textwidth}{chap3_parameters.tex}
\end{center}

\noindent In the above, all meta-parameters are not needed to make the Fortran code function, but when calling the code from C, these parameters are needed (and used) to ensure that the application only uses its own memory and thus, prevent \texttt{segmentation faults} from occuring.