\chapter{Introduction}

This chapter lists general information about this document.

\section{Purpose}
%The purpose of this particular ADD and its intended readership
The Architectural Design Document (ADD) describes the basic design of the \applicationname\ that is being developed by \projectauthor. First of all, it describes how the application is divided into different components. Then, for each component, it describes the dependencies to other components, and finally the relation to external interfaces.

\section{Scope}
%Scope of the software. Identifies the product by name, explains what the software will do
The \applicationname\ is an application designed and developed by \projectauthor\ for prof.dr.ir. P.D. Anderson, at the Eindhoven University of Technology. The application serves as an educational tool for anyone who wants to gain a deeper understanding of the process of mixing in general, and in particular for students at the TU/e.

\section{List of definitions and abbreviations}
\subsection{Definitions}
\begin{description}
\item[Client:] Prof.dr.ir. P.D. Anderson.
\item[Fortran:] A general-purpose, imperative programming language.
\item[Google Web Toolkit:] An open source set of tools that allows for the creation and maintenance of JavaScript applications in Java.
\item[Mixing step:] Part of a mixing protocol, it consists of: a wall to move, the direction the wall is moved in and the size of this step.
\item[Subordinate:] A child component.
\end{description}

\subsection{Abbreviations}
\begin{tabular}{l|l}
2IP35 & The Software Engineering Project \\
ADD & Architectural Design Document \\
GUI & Graphical User Interface \\
GWT & Google Web Toolkit \\
JNI & Java Native Interface \\
SEP   & Software Engineering Project \\
%SR    & Software Requirements \\ not used in the document
SRD   & Software Requirements Document \\
TU/e  & Eindhoven University of Technology \\
URD   & User Requirements Document \\
\end{tabular}

\section{List of references}
\bibliography{../ref}

\section{Overview}
%Short description of the rest of the ADD and how it is organized.
The remaining chapters describe the architectural design of \projectname\ in more detail. Chapter \ref{chap:systoverview} gives a system overview. Chapter \ref{chap:systcontext} describes the system context. The relationship with external components is explained in detail in this chapter. Chapter \ref{chap:systdesign} covers the sytem design. The name and reference of the used design method are given.
In chapter \ref{chap:compdescr} all components are described in detail. For each component, its type, purpose, function, subordinates, dependencies to other components, interfaces, resources needed, internal processing and  data are described. Chapter \ref{chap:feasresest} gives an overview of all resources needed to build, operate and maintain the application. Chapter \ref{chap:reqtracematrix} contains a traceability matrix. This matrix shows how each software requirement of the SRD \cite{srd} is linked to the components described in the ADD.


