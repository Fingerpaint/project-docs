\chapter{Tutorial}
\todo{short chapter intro}

\todo{For each session or lesson of the tutorial, a section of the following structure}
\section{Name of the session}

\subsection{Functional description}
\todo{What the tutorial session is supposed to achieve.}

\subsection{Cautions and warnings}
\todo{A list of precautions that may need to be taken}

\subsection{Procedures}

\subsubsection{Preparation}
\todo{how to prepare for and start the task}

\subsubsection{Description}
\todo{A step-by-step description of what the user must do and the response of the system}

\subsubsection{Results}
\todo{What final results to expect}

\subsection{Likely errors}
\todo{An informal description (not a list of errors) of likely errors and possible causes}

% STUKJES VAN FEMKE -------------------------------------------------------------------------------------------------
\section{View the performance graph of a mixing run}\label{sec:viewsinglegraph}

\subsection{Functional description}
In this tutorial, the performance graph of the previously executed mixing run is retrieved and viewed.

\subsection{Cautions and warnings}
When this tutorial is executed in Internet Explorer 10, the graph that should appear after step 1 from section  \ref{subsubsec:viewsinglegraphDesc} is not displayed. The graphs that we display are namely in SVG format and Internet Explorer 10 cannot handle this format.

\subsection{Procedures}

\subsubsection{Preparation}
% TODO: referentie naar sectie betreft uitvoeren van mixing run + Figuur van "Define a protocol" menu
Before this tutorial session can be executed, the session from section ... has to be done first. As a result, the menu from figure ... is shown.

\subsubsection{Description}\label{subsubsec:viewsinglegraphDesc}
\begin{enumerate}
	\item Press the \emph{View performance graph} button.
		\begin{itemize}
			\item The menu from figure \ref{fig:mixingGraphSingle} appears.
		\end{itemize}	
\end{enumerate}

\screenshot{mixingGraphSingle}{The pop-up menu that appears after clicking \emph{View performance graph}.}

\subsubsection{Results}
After following the steps as described above, a graph of the mixing performance of the previously executed mixing run is shown. In this graph, the mixing performance is shown on the y-axis, and the corresponding protocol step is shown on the x-axis. The initial performance of the mixing run is always 1.0, so there is always a point with this value at step 0.

\subsection{Likely errors}
None.