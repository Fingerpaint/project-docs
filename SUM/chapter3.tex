\chapter{Tutorial}
This chapter contains tutorials for performing all actions and operations supported by \projectname.

\section{Choosing a geometry and mixer type}
\label{sec:ChooseGeomAndMixer}

\subsection{Functional description}
This tutorial describes how to choose a geometry and mixer for your mixing simulation.

\subsection{Cautions and warnings}

Version 1.0 of the \applicationname{} only supports the \emph{Rectangle 400x240} geometry. It is intended to add the \emph{Square} geometry in the future, but as of now it has no functionality. Furthermore it is required to have JavaScript enabled in order for the application to display correctly.\\

\noindent To guarantee access the fingerpaint website, please make sure you have one of the following browsers:\\
\begin{itemize}
\item iOS Safari version 6.0 or higher
\item Firefox version 20 or higher
\item Google Chrome version 26 or higher.
\item Internet Explorer version 10 or higher.
\item Safari version 6.0 or higher.
\end{itemize}

\subsection{Procedures}

\subsubsection{Preparation}
Open your webbrowser and navigate to the \projectname{} website. Wait for the loading to finish and the menu to appear.

\subsubsection{Description}

\screenshot{cellBrowser}{The first window of the \applicationname.}

\begin{enumerate}
\item Select the \emph{Rectangle 400x240} geometry.
    \begin{itemize}
      \item A new menu appears to the right
    \end{itemize}
\item From this menu, select the \emph{Default} mixer.
    \begin{itemize}
      \item The main menu appears. Your selection of the \emph{Rectangle 400x240} geometry and \emph{Default} mixer is stored.
    \end{itemize}
\end{enumerate}

\subsubsection{Results}
You have now successfully selected a geometry and a mixer, and have arrived at the main menu of the application (figure \ref{fig:mainMenu}).

\screenshot{mainMenu}{The main menu of the \applicationname.}

\subsection{Likely errors}
If the message \emph{Your web browser must have JavaScript enabled in order for this application to display correctly.} appears, this means you have to enable JavaScript in order to load the application.\\

\noindent Version 1.0 of the \applicationname{} only supports the \emph{Rectangle 400x240} geometry. It is intended to add the \emph{Square} geometry in the future, but as of now it has no functionality. If you have selected the \emph{Square} geometry, you can change your selection by refreshing the browser and repeating the steps from the description.

%--------------NEW SECTION-----------------------------------------------

\section{Defining an initial concentration distribution}
\label{sec:defineDist}

\subsection{Functional description}
This tutorial describes how to define an initial concentration distribution.

\subsection{Cautions and warnings}
None.

\subsection{Procedures}
\subsubsection{Preparation}
A geometry and mixer need to have been selected to access the main menu.

\subsubsection{Description}
You can define the initial distribution by dragging across the screen. By default black fluids are added. To switch between adding black an white fluids you can press the second button from the top, the one with a black and white square displayed on it.

If the menu gets in the way of the canvas, it is possible to hide the menu by clicking the menu toggle button (the button with the "X" on it) at the top right corner of the screen.

 To adjust the size or shape of your drawing tool press the \emph{Select tool} button in the menu. A new menu will appear (figure \ref{fig:selectToolMenu}). Here you can change the shape of the drawing tool by pressing the button with either the square or circle shape on it. You can also change the size of your drawing tool by changing the number in the bottom of the menu. You can do this by pressing the + or - button, or by selecting the textfield and typing a size.

\screenshot{selectToolMenu}{The menu that appears after pressing \emph{Select tool} button in the main menu.}

In case of a mistake, simply toggle the color of your drawing tool and drag across the unwanted area.

If you want to erase the current distribution, you can press the \emph{Clear canvas} button in the main menu to return to a completely white distribution.

\subsubsection{Results}
Fluids of the selected color are added to the distribution wherever the dragging occurs. The size and shape of the added fluids depend on the chosen tool size and shape.

\subsection{Likely errors}
Only integer numbers are allowed for the drawing tool size. Furthermore the drawing tool size can't get smaller than 1, nor bigger than 50. If the menu gets in the way of the canvas, it is possible to hide the menu by clicking the menu toggle button (the button with the "X" on it) in the top right corner of the screen.

%--------------NEW SECTION-----------------------------------------------

\section{Defining and executing a mixing protocol}
\label{sec:defineMixProt}

\subsection{Functional description}
This tutorial describes how to define the mixing protocol for the mixing simulation.

\subsection{Cautions and warnings}
None.

\subsection{Procedures}
\subsubsection{Preparation}
%There are two ways to run the simulation:\\
%\begin{enumerate}
%\item Step by step
%\item Protocol
%\end{enumerate}
%
%When using step by step starts the simulation starts as soon as a step is defined. Protocol allows you to define several steps that will be executed in succession, before the mixing simulation runs.
You should be at, or navigate to, the main menu for this tutorial.

\subsubsection{Description}
A protocol can be defined by pressing the \emph{Define protocol} button in the main menu. The define protocol menu appears (figure \ref{fig:defineProtMenu}).

\screenshot{defineProtMenu}{The menu that appears after pressing the \emph{Define protocol} button in the main menu.}

The amount of time the wall should move can to be entered in the \emph{Step size} textfield. The default value is 1. You can change the value by pressing the + or - button next to the textfield, or by selecting the textfield and typing a size.

The protocol defines which wall movements should be executed in succession. You can add wall movements to the protocol by dragging the bars at the top or bottom of the canvas either left to right or right to left. The successive wall movements are displayed below the \emph{Number of steps} textfield.

The number of times the protocol should be repeated can to be entered in the \emph{Number of steps} textfield. The default value is 1. You can change the value by pressing the + or - button next to the textfield, or by selecting the textfield and typing a size.

When the protocol is satisfactory, press the \emph{Mix now} button to start the mixing simulation.

\subsubsection{Results}
The resulting concentration distribution of applying the given wall movement to the initial concentration distribution is shown on the canvas.

\subsection{Likely errors}
If the menu gets in the way of the bars on the canvas, it is possible to hide the menu by clicking the menu toggle button (the button with the "X" on it) at the top right corner of the screen. In the case that wrong wall movements are entered into the protocol, you can clear the protocol by pressing the \emph{Clear protocol} button.

%--------------NEW SECTION-----------------------------------------------

\section{Defining a single mixing step}
\label{sec:singleStepMix}

\subsection{Functional description}
This tutorial describes how to define the mixing protocol for the mixing simulation.

\subsection{Cautions and warnings}
None.

\subsection{Procedures}
\subsubsection{Preparation}
You should be at, or navigate to, the main menu for this tutorial.

\subsubsection{Description}
First, the amount of time the wall should move needs to be entered. The default value is 1, and is displayed in the \emph{Step size} textfield in the main menu. You can change the value by pressing the + or - button, or by selecting the textfield and typing a size.

After the step size has been set, you can select which wall should move by dragging the bars at the top or bottom of the canvas either left to right or right to left. The mixing simulation will run right away.

\subsubsection{Results}
The resulting concentration distribution of applying the given protocol to the initial concentration distribution is shown on the canvas.

\subsection{Likely errors}
If the menu gets in the way of the bars on the canvas, it is possible to hide the menu by clicking the menu toggle button (the button with the "X" on it) at the top right corner of the screen.

%--------------NEW SECTION-----------------------------------------------

\section{Exporting the performance graph of a mixing run}
\label{sec:exportSingleGraph}

\subsection{Functional description}
This tutorial describes how you can export the graph of the mixing performance locally to your device.

\subsection{Cautions and warnings}
The graph option is not available after a single step simulation. A performance graph can only be viewed after a successful mixing run has been executed with a defined protocol (see tutorial \ref{sec:defineMixProt}).

The performance graph functionality is not supported for the Internet Explorer web browser. Hence exporting is not possible.

\subsection{Procedures}
\subsubsection{Preparation}
Before the results of a mixing run can be viewed, a mixing run has to be executed successfully.

\subsubsection{Description}
\begin{enumerate}
	\item  Press the \emph{Define mixing protocol} button in the main menu.
		\begin{itemize}
            		\item The \emph{Define a protocol} menu appears.
		\end{itemize}
			\item Press the \emph{View performance graph} button.
		\begin{itemize}
            		\item A pop-up with the mixing performance graph appears (figure \ref{fig:singleGraph}).
		\end{itemize}

\screenshot{singleGraph}{The graph with the mixing performance of the last mixing run.}

    \item Press the \emph{Export graph} button.
		\begin{itemize}
             \item Your browser will give a prompt asking you where you want to save the image of the graph.
		\end{itemize}
    \item  Select a location and press \emph{OK} / \emph{save}.
		\begin{itemize}
           \item The image is saved to the selected location.
		\end{itemize}
\end{enumerate}

\subsubsection{Results}
A file with the performance graph of the mixing run has been saved on the selected location.

\subsection{Likely errors}
A performance graph can only be viewed after successful mixing run has been executed with a defined protocol (see tutorial \ref{sec:defineMixProt}). The graph option is not available after a single step simulations.

If you are running the \applicationname{} on a mobile device saving might not work.
%This could be elaborated, but we did not test extensive enough to give an exact description.

%--------------NEW SECTION-----------------------------------------------

\section{Exporting the mixing performance graph of multiple mixing runs}
\label{sec:exportMultiGraph}

\subsection{Functional description}
This tutorial describes how you can export the performance graph of multiple mixing runs.

\subsection{Cautions and warnings}
The performance graph functionality is not supported for the Internet Explorer web browser. Hence exporting is not possible.

\subsection{Procedures}
\subsubsection{Preparation}
Before the mixing performance of multiple mixing runs can be compared, at least 2 mix results need to be saved. See tutorial \ref{sec:savmixrun} for the tutorial on how to save a mixing result.

\subsubsection{Description}
\begin{enumerate}
	\item Press the \emph{Results} button in the main menu.
		\begin{itemize}
            The \emph{Mixing run results} menu appears.
		\end{itemize}
	\item Press the \emph{Compare performance} button.
		\begin{itemize}
            \item A pop-up with the saved mixing runs appears.
		\end{itemize}
	\item Select the mixing runs whose performance you want to compare, and press the \emph{Compare} button. \label{item:exportMultiGraphComp}
		\begin{itemize}
            \item A pop up with the mixing performance graph of the selected mixing runs appears (figure \ref{fig:multiGraph}).
		\end{itemize}
	\screenshot{multiGraph}{The graph with the mixing performance of the last mixing run.}
	\item Press the \emph{Export graph} button.
        \begin{itemize}
            \item Your browser will give a prompt asking you where you want to save the graph.
        \end{itemize}
	\item  Select a location and press \emph{OK} / \emph{save}.
\end{enumerate}


\subsubsection{Results}
A file with the performance graph of the selected mixing runs has been saved on the selected location.

\subsection{Likely errors}
If you want to compare the current mixing run with a saved performance, you will have to save the results of the current mixing run first. See tutorial \ref{sec:savmixrun} for the tutorial on how to save a mixing result.

If no items have been selected in step \ref{item:exportMultiGraphComp} when \emph{Compare} is pressed, a \emph{No data} message is shown.

If you are running the \applicationname on a mobile device exporting might not work.

%--------------NEW SECTION-----------------------------------------------

\section{Exporting a concentration distribution}
\label{sec:exportDist}

\subsection{Functional description}
This tutorial describes how to export an image of the canvas.

\subsection{Cautions and warnings}
None.

\subsection{Procedures}
\subsubsection{Preparation}
You should be at, or navigate to, the main menu for this tutorial.

\subsubsection{Description}
\begin{enumerate}
	\item Press the \emph{Distributions} button on the main menu.
		\begin{itemize}
            The \emph{Define a protocol} menu appears.
		\end{itemize}
	\item Press the \emph{Export image} button.
		\begin{itemize}
            \item Your browser will give a prompt asking you where you want to save the image of the canvas.
		\end{itemize}
	\item  Select a location and press \emph{OK} / \emph{save}.
\end{enumerate}

\subsubsection{Results}
A file with the concentration distribution of the canvas has been saved on the selected location.

\subsection{Likely errors}
If you are running the \applicationname on a mobile device exporting might not work.

%--------------NEW SECTION-----------------------------------------------

\section{Loading a concentration distribution}
\label{sec:loadDist}

\subsection{Functional description}
This tutorial describes how to load a concentration distribution to the canvas.

\subsection{Cautions and warnings}
It is only possible to load a concentration distribution if one has been saved. See tutorial \ref{sec:savdist} for the tutorial on how to save a concentration distribution.

\subsection{Procedures}
\subsubsection{Preparation}
You should be at, or navigate to, the main menu for this tutorial.

\subsubsection{Description}
\begin{enumerate}
	\item Press the \emph{Distributions} button on the main menu.
		\begin{itemize}
            \item The \emph{Distributions} menu appears.
		\end{itemize}
	\item Press the \emph{Load} button.
		\begin{itemize}
            \item  A pop-up with the saved concentration distributions appears.
		\end{itemize}
    \item Press the name of the distribution you want to load.
		\begin{itemize}
           \item The selected distribution has been loaded to the canvas.
		\end{itemize}
\end{enumerate}

\subsubsection{Results}
The concentration of the canvas has been replaced with the concentration distribution of the selected saved distribution.

\subsection{Likely errors}
If no saved distributions are available, a \emph{No saved files} message will appear.

%--------------NEW SECTION-----------------------------------------------

\section{Loading a mixing protocol}
\label{sec:loadMixProt}

\subsection{Functional description}
This tutorial describes how to load a mixing protocol.

\subsection{Cautions and warnings}
It is only possible to load a mixing protocol if one has been saved. See tutorial \ref{sec:savmixprot} for the tutorial on how to save a mixing protocol.

\subsection{Procedures}
\subsubsection{Preparation}
You should be at, or navigate to, the main menu for this tutorial.

\subsubsection{Description}
\begin{enumerate}
	\item Press the \emph{Define protocol} button on the main menu.
		\begin{itemize}
            \item The \emph{Define a protocol} menu appears.
		\end{itemize}
	\item Press the \emph{Protocols} button.
		\begin{itemize}
             \item The \emph{Protocols} menu appears.
             \screenshot{protMenu}{The menu that appears after pressing the \emph{Protocols} button in the \emph{Define a protocol} menu.}
		\end{itemize}
	\item Press the \emph{Load} button.
		\begin{itemize}
            \item A pop-up with the saved mixing protocols appears.
		\end{itemize}
    \item Press the name of the distribution you want to load.
   		\begin{itemize}
            \item The selected protocol is loaded.
		\end{itemize}
\end{enumerate}

\subsubsection{Results}
The current protocol will be replaced by the loaded protocol.

\subsection{Likely errors}
If no saved mixing protocols are available, a \emph{No saved files} message will appear.

%--------------NEW SECTION-----------------------------------------------

\section{Loading a mixing result}
\label{sec:loadMixResult}

\subsection{Functional description}
This tutorial describes how to load a mixing result.

\subsection{Cautions and warnings}
It is only possible to load a mixing result if one has been saved. See \ref{sec:savmixrun} for the tutorial on how to save a mixing protocol.

\subsection{Procedures}
\subsubsection{Preparation}
You should be at, or navigate to, the main menu for this tutorial.

\subsubsection{Description}

\begin{enumerate}
	\item Press the \emph{Results} button on the main menu.
		\begin{itemize}
           \item  The \emph{Mixing run results} menu appears.
		\end{itemize}
	\item Press the \emph{Load} button.
		\begin{itemize}
            \item A pop-up with the saved mixing results appears.
		\end{itemize}
	\item  Press the name of the mixing result you want to load.
		\begin{itemize}
           \item  The select mixing result is loaded into the application.
		\end{itemize}
\end{enumerate}

\subsubsection{Results}
The concentration of the canvas and current protocol have been replaced with the concentration distribution and mixing protocol of the selected saved mixing result.

\subsection{Likely errors}
If no saved mixing results are available, a \emph{No saved files} message will appear.

%--------------NEW SECTION-----------------------------------------------

\section{Removing a concentration distribution from local storage}
\label{sec:remdist}

\subsection{Functional description}
This tutorial describes how to remove a previously saved concentration distribution.

\subsection{Cautions and warnings}
It is only possible to remove a concentration distribution if one has been saved. See tutorial \ref{sec:savdist} for the tutorial on how to save a concentration distribution.

\subsection{Procedures}

\subsubsection{Preparation}
You should be at, or navigate to the main menu for this tutorial.

\subsubsection{Description}
\begin{enumerate}
	\item Press the \emph{Distributions} button.
		\begin{itemize}
			The \emph{Distributions} menu appears (figure \ref{fig:distMenu}).
		\end{itemize}
	\item Press the \emph{Remove} button.
		\begin{itemize}
			\item A pop-up with the saved concentration distributions appears.
		\end{itemize}
	\item Press the \emph{X}-button next to the distribution that you want to remove.
		\begin{itemize}
			\item The selected distribution has been removed.
			\item A \emph{Delete successful} message is displayed, which disappears after a few seconds.
		\end{itemize}
\end{enumerate}

\subsubsection{Results}
You have now successfully removed the selected concentration distribution locally from your device.

\subsection{Likely errors}
If no concentration distributions are available, a \emph{No saved files} message will appear.

%--------------NEW SECTION-----------------------------------------------
\section{Removing a mixing protocol from local storage}
\label{sec:remmixprot}

\subsection{Functional description}
This tutorial describes how to remove a previously saved mixing protocol.

\subsection{Cautions and warnings}
It is only possible to remove a mixing protocol if one has been saved. See tutorial \ref{sec:savmixprot} for the tutorial on how to save a mixing protocol.

\subsection{Procedures}

\subsubsection{Preparation}
You should be at, or navigate to the main menu for this tutorial.

\subsubsection{Description}
\begin{enumerate}
	\item Press the \emph{Define protocol} button on the main menu.
		\begin{itemize}
            \item The \emph{Define a protocol} menu appears.
		\end{itemize}
	\item Press the \emph{Remove} button.
		\begin{itemize}
			\item A pop-up with the saved mixing protocols appears.
		\end{itemize}
	\item Press the \emph{X}-button next to the protocol that you want to remove.
		\begin{itemize}
			\item The selected protocol has been removed.
			\item A \emph{Delete successful} message is displayed, which disappears after a few seconds.
		\end{itemize}
\end{enumerate}

\subsubsection{Results}
You have now successfully removed the selected mixing protocol locally from your device.

\subsection{Likely errors}
If no mixing protocols are available, a \emph{No saved files} message will appear.

%--------------NEW SECTION-----------------------------------------------
\section{Removing a mixing run from local storage}
\label{sec:remmixrun}

\subsection{Functional description}
This tutorial describes how to remove a previously saved mixing run.

\subsection{Cautions and warnings}
It is only possible to remove a mixing run if one has been saved. See tutorial \ref{sec:savmixrun} for the tutorial on how to save a mixing run.

\subsection{Procedures}

\subsubsection{Preparation}
You should be at, or navigate to the main menu for this tutorial.

\subsubsection{Description}
\begin{enumerate}
	\item Press the \emph{Results} button on the main menu.
		\begin{itemize}
           \item  The \emph{Mixing run results} menu appears (figure \ref{fig:mixResultMenu}).
		\end{itemize}
	\item Press the \emph{Remove} button.
		\begin{itemize}
			\item A pop-up with the saved mixing runs appears.
		\end{itemize}
	\item Press the \emph{X}-button next to the mixing run that you want to remove.
		\begin{itemize}
			\item The selected mixing run has been removed.
			\item A \emph{Delete successful} message is displayed, which disappears after a few seconds.
		\end{itemize}
\end{enumerate}

\subsubsection{Results}
You have now successfully removed the selected mixing run locally from your device.

\subsection{Likely errors}
If no mixing runs are available, a \emph{No saved files} message will appear.

%--------------NEW SECTION-----------------------------------------------

\section{Saving a concentration distribution}
\label{sec:savdist}

\subsection{Functional description}
This tutorial describes how to save the current concentration distribution.

\subsection{Cautions and warnings}
It is only possible to save a concentration distribution if one has been defined. See tutorial \ref{sec:defineDist} for the tutorial on how to define a concentration distribution.

\subsection{Procedures}

\subsubsection{Preparation}
You should be at, or navigate to the main menu for this tutorial.

\subsubsection{Description}
\begin{enumerate}
	\item Press the \emph{Distributions} button.
		\begin{itemize}
			The \emph{Distributions} menu appears (figure \ref{fig:distMenu}).
		\end{itemize}
	\item Press the \emph{Save} button.
		\begin{itemize}
			\item The pop-up from figure \ref{fig:saveItemPopup} appears.
		\screenshot{saveItemPopup}{The pop-up menu that appears after clicking the \emph{Save} button in the distributions menu. This menu is also used for saving a mixing protocol (tutorial \ref{sec:savmixprot}) and a mixing run (tutorial \ref{sec:savmixrun}).}
		\end{itemize}
	\item Enter a name for the distribution in the text area.
		\begin{itemize}
			\item The entered name appears in the text area.
		\end{itemize}
	\item Press the \emph{Save} button.
		\begin{itemize}
			\item The save results pop-up closes.
			\item A \emph{Save successful} message is displayed, which disappears after a few seconds.
		\end{itemize}
\end{enumerate}

\subsubsection{Results}
You have now successfully saved the current concentration distribution locally to your device.

\subsection{Likely errors}
If the name entered in step 3 is already in use, the pop-up from figure \ref{fig:overwriteSavePopup} is shown. Whenever you want to overwrite the existing concentration distribution, press the \emph{Overwrite} button.

\screenshot{overwriteSavePopup}{The pop-up that appears after clicking the \emph{Save} button from the menu of figure \ref{fig:saveItemPopup}, when the entered name is already in use.}

%--------------NEW SECTION-----------------------------------------------

\section{Saving a mixing protocol}
\label{sec:savmixprot}

\subsection{Functional description}
This tutorial describes how to save the current mixing protocol.

\subsection{Cautions and warnings}
It is only possible to save a mixing protocol if one has been defined. See tutorial \ref{sec:defineMixProt} for the tutorial on how to define a mixing protocol.

\subsection{Procedures}

\subsubsection{Preparation}
You should be at, or navigate to the main menu for this tutorial.

\subsubsection{Description}
\begin{enumerate}
	\item Press the \emph{Define protocol} button on the main menu.
		\begin{itemize}
            \item The \emph{Define a protocol} menu appears.
		\end{itemize}
	\item Press the \emph{Protocols} button.
		\begin{itemize}
			\item The \emph{Protocols} menu appears (figure \ref{fig:protMenu}).
		\end{itemize}
	\item Press the \emph{Save} button.
		\begin{itemize}
			\item The pop-up from figure \ref{fig:saveItemPopup} appears.
		\end{itemize}
	\item Enter a name for the mixing protocol in the text area. \label{item:savmixprotName}
		\begin{itemize}
			\item The entered name appears in the text area.
		\end{itemize}
	\item Press the \emph{Save} button.
		\begin{itemize}
			\item The save results pop-up closes.
			\item A \emph{Save successful} message is displayed, which disappears after a few seconds.
		\end{itemize}
\end{enumerate}

\subsubsection{Results}
You have now successfully saved the current mixing protocol locally to your device.

\subsection{Likely errors}
If the name entered in step \ref{item:savmixprotName} is already in use, the pop-up from figure \ref{fig:overwriteSavePopup} is shown. Whenever you want to overwrite the existing mixing protocol, press the \emph{Overwrite} button.

%--------------NEW SECTION-----------------------------------------------

\section{Saving a mixing run}
\label{sec:savmixrun}

\subsection{Functional description}
This tutorial describes how to save a previously executed mixing run.

\subsection{Cautions and warnings}
It is only possible to save a mixing run if one has been executed. See tutorial \ref{sec:defineMixProt} for the tutorial on how to execute a mixing run.

\subsection{Procedures}

\subsubsection{Preparation}
You should be at, or navigate to the main menu for this tutorial.

\subsubsection{Description}
\begin{enumerate}
	\item Press the \emph{Results} button on the main menu.
		\begin{itemize}
           \item  The \emph{Mixing run results} menu appears (figure \ref{fig:saveItemPopup}).
		\end{itemize}
	\item Press the \emph{Save} button.
		\begin{itemize}
			\item The pop-up from figure \ref{fig:saveItemPopup} appears.
		\end{itemize}
	\item Enter a name for the mixing run in the text area.
		\begin{itemize}
			\item The entered name appears in the text area.
		\end{itemize}
	\item Press the \emph{Save} button. \label{item:savmixrunName}
		\begin{itemize}
			\item The save results pop-up closes.
			\item A \emph{Save successful} message is displayed, which disappears after a few seconds.
		\end{itemize}
\end{enumerate}

\subsubsection{Results}
You have now successfully saved the previously executed mixing run locally to your device.

\subsection{Likely errors}
If the name entered in step \ref{item:savmixrunName} is already in use, the pop-up from figure \ref{fig:overwriteSavePopup} is shown. Whenever you want to overwrite the existing mixing run, press the \emph{Overwrite} button.

%--------------NEW SECTION-----------------------------------------------

\section{Viewing multiple performance results in one graph}
\label{sec:viewmultgraph}

\subsection{Functional description}
This tutorial describes how to view the performance results of previously saved mixing results simultaneously in one graph.

\subsection{Cautions and warnings}
When this tutorial is executed in Internet Explorer 10, the graph that should appear after step 4 from tutorial \ref{subsubsec:viewmultgraphDesc} is not displayed. The graph that is displayed is namely in SVG format and Internet Explorer 10 cannot handle this format.

It is only possible to view multiple performance results in one graph if at least two mixing runs have been saved. See tutorial \ref{sec:savmixrun} for the tutorial on how to save a mixing run.

\subsection{Procedures}

\subsubsection{Preparation}
You should be at, or navigate to the main menu for this tutorial.

\subsubsection{Description}\label{subsubsec:viewmultgraphDesc}
\begin{enumerate}
	\item Press the \emph{Results} button in the main menu.
		\begin{itemize}
            The \emph{Mixing run results} menu appears.
		\end{itemize}
	\item Press the \emph{Compare performance} button.
		\begin{itemize}
            \item A pop-up with the saved mixing runs appears.
		\end{itemize}
	\item Select the mixing runs whose performance you want to compare, and press the \emph{Compare} button.
		\begin{itemize}
            \item A pop-up with the mixing performance graph of the selected mixing runs appears (figure \ref{fig:multiGraph}).
		\end{itemize}
\end{enumerate}

\subsubsection{Results}
You have now successfully obtained a graph of the mixing performance of the selected mixing runs. In this graph, the mixing performance is shown on the y-axis, and the corresponding protocol step is shown on the x-axis. The initial performance of the mixing run is always 1.0, so there is always a point with this value at step 0. Moreover, the mixing performance from the mixing results are displayed in different colours; the legend explains which colour corresponds to which mixing run.

\subsection{Likely errors}
If no mixing runs are available, a \emph{No saved files} message will appear.

If no items have been selected in step \ref{item:viewmultgraphSelect} when \emph{Compare} is pressed, a \emph{No data} message is shown.

%--------------NEW SECTION-----------------------------------------------

\section{Viewing the performance graph of a mixing run}
\label{sec:viewsinglegraph}

\subsection{Functional description}
This tutorial describes how to view the performance graph of the previously executed mixing run.

\subsection{Cautions and warnings}
When this tutorial is executed in Internet Explorer 10, the graph that should appear after step 1 from tutorial  \ref{subsubsec:viewsinglegraphDesc} is not displayed. The graph that is displayed is namely in SVG format and Internet Explorer 10 cannot handle this format.

It is only possible to view the performance graph of a mixing run if one has been executed. See tutorial \ref{sec:defineMixProt} for the tutorial on how to execute a mixing run.

\subsection{Procedures}

\subsubsection{Preparation}
You should be at, or navigate to the main menu for this tutorial.

\subsubsection{Description}\label{subsubsec:viewsinglegraphDesc}
\begin{enumerate}
	\item Press the \emph{Define protocol} button on the main menu.
		\begin{itemize}
            \item The \emph{Define a protocol} menu appears (figure \ref{fig:defineProtMenu}).
		\end{itemize}
	\item Press the \emph{Protocols} button.
		\begin{itemize}
			\item The \emph{Protocols} menu appears (figure \ref{fig:protMenu}).
		\end{itemize}
	\item Press the \emph{View performance graph} button.
		\begin{itemize}
			\item A pop-up with the mixing performance graph appears (figure \ref{fig:singleGraph}).
		\end{itemize}	
\end{enumerate}

\subsubsection{Results}
You have now successfully obtained a graph of the mixing performance of the previously executed mixing run. In this graph, the mixing performance is shown on the y-axis, and the corresponding protocol step is shown on the x-axis. The initial performance of the mixing run is always 1.0, so there is always a point with this value at step 0.

\subsection{Likely errors}
None.

% --------------------------------------------------------------------------------------------------

%\section{Name of the session}
%\label{sec:}
%
%\subsection{Functional description}
%%\todo{What the tutorial session is supposed to achieve.}
%This tutorial describes
%
%\subsection{Cautions and warnings}
%%\todo{A list of precautions that may need to be taken}
%
%\subsection{Procedures}
%\subsubsection{Preparation}
%%\todo{how to prepare for and start the task}
%
%\subsubsection{Description}
%%\todo{A step-by-step description of what the user must do and the response of the system}
%
%\subsubsection{Results}
%%\todo{What final results to expect}
%
%\subsection{Likely errors}
%%\todo{An informal description (not a list of errors) of likely errors and possible causes}
