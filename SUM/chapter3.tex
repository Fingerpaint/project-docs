\chapter{Tutorial}
%\todo{short chapter intro}
This chapter contains tutorials for performing all actions and operations supported by \projectname. It is assumed the \applicationname has been accessed successfully. 

%\todo{For each session or lesson of the tutorial, a section of the following structure}
%\section{Name of the session}
%
%\subsection{Functional description}
%%\todo{What the tutorial session is supposed to achieve.}
%This section describes 
%\subsection{Cautions and warnings}
%%\todo{A list of precautions that may need to be taken}
%
%\subsection{Procedures}
%
%\subsubsection{Preparation}
%%\todo{how to prepare for and start the task}
%
%\subsubsection{Description}
%%\todo{A step-by-step description of what the user must do and the response of the system}
%
%\subsubsection{Results}
%%\todo{What final results to expect}
%
%\subsection{Likely errors}
%%\todo{An informal description (not a list of errors) of likely errors and possible causes}

\section{Choosing a geometry and mixer type}
\label{sec:ChooseGeomAndMixer}

\subsection{Functional description}
This tutorial describes how to choose a geometry and mixer for your mixing simulation.

\subsection{Cautions and warnings}

Version 1.0 of the \applicationname only supports the \emph{Rectangle 400x240} geometry. It is intended to add the \emph{Square} geometry in the future, but as of now it has no functionality.\\

\subsection{Procedures}

\subsubsection{Preparation}
Open your webbrowser and navigate to the \projectname website. Wait for the loading to finish and the menu to appear.

\subsubsection{Description}
%\todo{A step-by-step description of what the user must do and the response of the system}

\begin{enumerate}
\item Select the \emph{Rectangle 400x240} geometry.
\begin{itemize}
  \item A new menu appears to the right
\end{itemize}
\item From this menu, select the \emph{Default} mixer.
\begin{itemize}
  \item The main menu appears. Your selection of the \emph{Rectangle 400x240} geometry and \emph{Default} mixer is stored.
\end{itemize}
\end{enumerate}

\subsubsection{Results}
%\todo{What final results to expect}
You have now successfully selected a geometry and a mixer, and are at the main menu of the application.

\subsection{Likely errors}
Version 1.0 of the \applicationname\ only supports the \emph{Rectangle 400x240} geometry. It is intended to add the \emph{Square} geometry in the future, but as of now it has no functionality. If you have selected the \emph{Square} geometry, you can change your selecting by refreshing the browser and repeating step 1 \& 2.\\

\noindent If you cannot access the fingerpaint website, please make sure you have one of the following browsers:\\
\begin{itemize}
\item iOS Safari version 6.0 or higher
\item Firefox version 20 or higher
\item Google Chrome version 26 or higher.
\item Internet Explorer version 10 or higher.
\item Safari version 6.0 or higher.
\end{itemize}

% STUKJES VAN FEMKE -------------------------------------------------------------------------------------------------
\section{View the performance graph of a mixing run}\label{sec:viewsinglegraph}

\subsection{Functional description}
In this tutorial, the performance graph of the previously executed mixing run is retrieved and viewed.

\subsection{Cautions and warnings}
When this tutorial is executed in Internet Explorer 10, the graph that should appear after step 1 from section  \ref{subsubsec:viewsinglegraphDesc} is not displayed. The graphs that we display are namely in SVG format and Internet Explorer 10 cannot handle this format.

\subsection{Procedures}

\subsubsection{Preparation}
% TODO: referentie naar sectie betreft uitvoeren van mixing run + Figuur van "Define a protocol" menu
Before this tutorial session can be executed, the session from section ... has to be done first. As a result, the menu from figure ... is shown.

\subsubsection{Description}\label{subsubsec:viewsinglegraphDesc}
\begin{enumerate}
	\item Press the \emph{View performance graph} button.
		\begin{itemize}
			\item The menu from figure \ref{fig:mixingGraphSingle} appears.
		\end{itemize}	
\end{enumerate}

\screenshot{mixingGraphSingle}{The pop-up menu that appears after clicking \emph{View performance graph}.}

\subsubsection{Results}
After following the steps as described above, a graph of the mixing performance of the previously executed mixing run is shown. In this graph, the mixing performance is shown on the y-axis, and the corresponding protocol step is shown on the x-axis. The initial performance of the mixing run is always 1.0, so there is always a point with this value at step 0.

\subsection{Likely errors}
None.