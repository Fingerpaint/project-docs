\chapter{Reference}
This chapter describes various parts of the \applicationname{}. For each reference, a short functional description is provided, along with warnings and errors. Furthermore, related references are listed for each item.

%-------------------------------------------------------------------------------------------------
\section{Cell Browser}
\label{sec:cellBrows}
  \subsection*{Functional description}
  The Cell Browser is the first menu encountered by the user. Using the first level of the Cell Browser, the user can select the desired geometry for mixing. Currently, available geometries are the \emph{Rectangle} and the \emph{Square}. When one of the possible choices has been selected, the second level of the Cell Browser appears. This allows the user to select the desired mixer for the selected geometry. Currently, only the \emph{Default} mixers are available for both the \emph{Rectangle} and the \emph{Square} geometries.
 
  \subsection*{Cautions and warnings}
  Even though the \emph{Square} geometry has been implemented, its mixer has not. Upon trying to mix with this geometry, an error message is shown, informing the user that this mixer has not been implemented.

  \subsection*{Formal description}
  \begin{tabularx}{\textwidth}{XXX}
    \toprule
    \emph{Operation} & \emph{Steps} & \emph{Results} \\
    \midrule
    Select a geometry & Choose one of the \emph{Rectangle} and the \emph{Square} geometries. & The selected geometry is highlighted, and a list of available mixers is shown. \\
    \midrule
    Select a mixer & Choose one of the available mixers & The chosen geometry appears on the screen, along with the main menu. Mixer choice is saved. \\
    \bottomrule
  \end{tabularx}

  \subsection*{Possible errors}
  None.

  \subsection*{Related operations}
  \begin{itemize}
    \item Section \ref{sec:drawingarea}
    \item Section \ref{sec:mainmenu}
  \end{itemize}

%-------------------------------------------------------------------------------------------------
\section{Drawing Area}
\label{sec:drawingarea}
  \subsection*{Functional description}
  The drawing area is used for drawing the initial concentration result. This is achieved by dragging over the canvas. Depending on the chosen colour and shape of the drawing tool, the result of fluids on the canvas changes accordingly.
  
  When loading a saved result or a saved collection of results, the canvas is used to display the loaded result. After loading, everything that was on the canvas prior to loading is removed, and replaced by what was loaded. However, after the result has been loaded, the user can freely draw on the canvas again.
  
  Lastly, the canvas is used for dragging the walls of the geometry.
  
  \subsection*{Cautions and warnings}
  The resolution of the canvas depends on the screen resolution. This means that a higher screen resolution means the canvas contains more pixels, and the drawing appears sharper. However, the internal representation for the canvas is dependant on the resolution of the geometry. In the case of the \emph{Rectangle} geometry, this resolution is a fixed $400 \times 240$. When loading results, it is possible that the resulting result appears more pixelated than it was at its creation, but the underlying data structure is unchanged.
  
  \subsection*{Formal description}
  \begin{tabularx}{\textwidth}{XXX}
    \toprule
    \emph{Operation} & \emph{Steps} & \emph{Results} \\
    \midrule
    Add some fluid & Drag on the canvas & The fluids on the canvas change accordingly. \\
    \midrule
    Move a wall & Drag one of the walls to the left or right. & A mixing step is added if the \emph{Define Protocol} menu is opened, or a mixing step is executed if it is not. \\
    \bottomrule
  \end{tabularx}

  \subsection*{Possible errors}
  None.

  \subsection*{Related operations}
  \begin{itemize}
    \item Section \ref{sec:selecttoolmenu}
    \item Section \ref{sec:loaditempanel}
  \end{itemize}  

%-------------------------------------------------------------------------------------------------
\section{Main Menu}\label{sec:mainmenu}
  \subsection*{Functional description}
  The main menu contains a button to switch to the \emph{Tool selection} menu. When this button is clicked, the main menu disappears, and the \texttt{Tool selection} menu appears in its place.
  
  A button to change the colour of the drawing tool is also present. The canvas can be cleared by pressing the \texttt{Clear canvas} button.
  
  From the main menu, the user can switch to the \emph{Distributions} and \emph{Results} menus by clicking their respective buttons.
  
  The user can select the step size in a spinner. This step size is used for defining the protocol and immediately executing a mixing step.
  
  The \emph{Define protocol} menu can be accessed from the main menu by pressing the corresponding button.
  
  \subsection*{Cautions and warnings}
  None.

  \subsection*{Formal description}
  \begin{tabularx}{\textwidth}{XXX}
    \toprule
    \emph{Operation} & \emph{Steps} & \emph{Results} \\
    \midrule
    Open the \emph{Tool Selection} menu & Press the \texttt{Select Tool} button & The \emph{Tool Selection} menu is opened. \\
    \midrule
    Change the colour of the drawing tool & Press the button containing black and white squares. & The colour of the tool changes from black to white or from white to black. \\
    \midrule
    Open the \emph{Distribution} menu & Press the \texttt{Distribution} button & The \emph{Distribution} menu is opened. \\
    \midrule
    Open the \emph{Results} menu & Press the \texttt{Select Tool} button & The \emph{Results} menu is opened. \\
    \midrule
    Change the step size & Change the value in the spinner. & The size of new steps is changed. \\
    \midrule
    Open the \emph{Define Protocol} menu & Press the \texttt{Define Protocol} button & The \emph{Define Protocol} menu is opened. \\
    \bottomrule
  \end{tabularx}

  \subsection*{Possible errors}
  None.

  \subsection*{Related operations}
  \begin{itemize}
    \item Section \ref{sec:selecttoolmenu}
    \item Section \ref{sec:distmenu}
    \item Section \ref{sec:resultsmenu}
    \item Section \ref{sec:defprot}
  \end{itemize}

%-------------------------------------------------------------------------------------------------
\section{Select Tool Menu}\label{sec:selecttoolmenu}
  \subsection*{Functional description}
  This menu can be used to change the colour of the drawing tool. This is achieved through a button that toggles the colour either from black to white or from white to black.
  
  Furthermore, this menu contains two buttons to change the shape of the drawing tool. Clicking the square button changes the drawing tool to a square. Clicking the circle button changes the drawing tool to a circle.
  
  The spinner in this menu can be used to change the size of the drawing tool. The minimum size of the drawing tool is 1, and the maximum size of the drawing tool is 50.
  
  The \texttt{Back} button can be used to return to the main menu.

  \subsection*{Cautions and warnings}
  When a large size is selected for the drawing tool, drawing might slow down slightly.

  \subsection*{Formal description}
  \begin{tabularx}{\textwidth}{XXX}
    \toprule
    \emph{Operation} & \emph{Steps} & \emph{Results} \\
    \midrule
    Change the colour of the drawing tool & Press the button containing black and white squares. & The colour of the tool changes from black to white or from white to black. \\
    \midrule
    Change the shape of the drawing tool to a square & Press the square button. & The shape of the tool changes to a square. \\
    \midrule
    Change the shape of the drawing tool to a circle & Press the circle button. & The shape of the tool changes to a circle. \\
    \midrule
    Change the size of the drawing tool & Change the value of the spinner. & The size of the drawing tool changes accordingly. \\ 
    \midrule
    Close this menu & Press the \texttt{Back} button. & The main menu reappears.
    \bottomrule
  \end{tabularx}

  \subsection*{Possible errors}
  None.

  \subsection*{Related operations}
  \begin{itemize}
  \item Section \ref{sec:drawingarea}
    \item Section \ref{sec:mainmenu}
  \end{itemize}

%-------------------------------------------------------------------------------------------------
\section{Distributions Menu}\label{sec:distmenu}
  \subsection*{Functional description}
  The \emph{Distributions} menu can be used to save, load and remove results, and to export an image of the current results to disk. Pressing the \texttt{Save} button pops up a menu, in which the desired name can be entered. If that name already exists, the user can choose to overwrite that name, or choose a new name.
  
  Pressing the \texttt{Load} button causes a menu to appear, containing the names of all saved results. If there are no saved results, the user is notified of this fact. If there are more results than fit on the screen, scrollbars appear to allow the user to view all parts of list of saved results.
  
  The \texttt{Remove} button creates a pop-up panel containing the names of all saved results. Next to each of these names is a delete button that can be used to delete the result. Deleting a result cannot be undone. 
  
  The \texttt{Export Image} button allows the user to download an image of the current result. This generates a \emph{PNG} file that can be downloaded to the client's device using the browser's download functionality.
  
  The \texttt{Back} button can be used to return to the main menu.

  \subsection*{Cautions and warnings}
  \begin{itemize}
    \item The load and remove buttons have reduced functionality if no results were saved.
    \item The capacity of the local storage is limited. When it is full, results or other saved \projectname{} items must be removed, before saving is possible again. The user is notified of a full local storage. Note that the local storage's capacity can be exceeded even though there is enough free disk space, as each website can only save a certain amount of data in the local storage.
  \end{itemize}
  
  \subsection*{Formal description}
  \begin{tabularx}{\textwidth}{XXX}
    \toprule
    \emph{Operation} & \emph{Steps} & \emph{Results} \\
    \midrule
    Save a distribution & Press the \texttt{Save} button. & The \emph{Save Item Panel} appears. \\
    \midrule
    Load a distribution & Press the \texttt{Load} button. & The \emph{Load Item Panel} appears. \\
    \midrule
    Remove a distribution & Press the \texttt{Remove} button. & The \emph{Remove Item Panel} appears. \\
    \midrule
    Export an image of the distribution & Press the \texttt{Export Image} button. & A download prompt appears. \\
    \midrule
    Close this menu & Press the \texttt{Back} button. & The main menu reappears.
    \bottomrule
  \end{tabularx}

  \subsection*{Possible errors}
  None.

  \subsection*{Related operations}
  \begin{itemize}
  \item Section \ref{sec:drawingarea}
    \item Section \ref{sec:mainmenu}
    \item Section \ref{sec:saveitem}
    \item Section \ref{sec:overwriteitem}
    \item Section \ref{sec:loaditempanel}
    \item Section \ref{sec:removeitem}
  \end{itemize}

%-------------------------------------------------------------------------------------------------
\section{Results Menu}\label{sec:resultsmenu}
  \subsection*{Functional description}
  The \emph{Results} menu can be used to save, load, remove and compare results. Pressing the \texttt{Save} button pops up a menu, in which the desired name can be entered. If that name already exists, the user can choose to overwrite that name, or choose a new name.
  
  Pressing the \texttt{Load} button causes a menu to appear, containing the names of all saved results. If there are no saved results, the user is notified of this fact. If there are more results than fit on the screen, scrollbars appear to allow the user to view all parts of list of saved results.
  
  The \texttt{Remove} button creates a pop-up panel containing the names of all saved results. Next to each of these names is a delete button that can be used to delete the result. Deleting a result cannot be undone. 
  
  The \texttt{Compare Performance} button can be used to compare the performance of multiple runs. Pressing this button causes a pop-up panel to appear, in which the user can select results. The performance of all selected results are plotted in a graph once the user presses the \texttt{Compare} button.
  
  The \texttt{Back} button can be used to return to the main menu.

  \subsection*{Cautions and warnings}
  \begin{itemize}
    \item The \texttt{Load}, \texttt{Remove} and \texttt{Compare Performance} buttons have reduced functionality if no results were saved.
    \item The capacity of the local storage is limited. When it is full, results or other saved \projectname{} items must be removed, before saving is possible again. The user is notified of a full local storage. Note that the local storage's capacity can be exceeded even though there is enough free disk space, as each website can only save a certain amount of data in the local storage.
  \end{itemize}  

  \subsection*{Formal description}
  \begin{tabularx}{\textwidth}{XXX}
    \toprule
    \emph{Operation} & \emph{Steps} & \emph{Results} \\
    \midrule
    Save a result & Press the \texttt{Save} button. & The \emph{Save Item Panel} appears. \\
    \midrule
    Load a result & Press the \texttt{Load} button. & The \emph{Load Item Panel} appears. \\
    \midrule
    Remove a result & Press the \texttt{Remove} button. & The \emph{Remove Item Panel} appears. \\
    \midrule
    Compare the performance of multiple runs & Press the \texttt{Compare Performance} button. & The \emph{Compare Multiple Graphs Panel} appears. \\
    \midrule
    Close this menu & Press the \texttt{Back} button. & The main menu reappears.
    \bottomrule
  \end{tabularx}

  \subsection*{Possible errors}
  None.
  
  \subsection*{Related operations}
  \begin{itemize}
    \item Section \ref{sec:drawingarea}
    \item Section \ref{sec:compareperf}
    \item Section \ref{sec:mulperfgraph}
  \end{itemize}

%-------------------------------------------------------------------------------------------------
\section{Define Protocol Menu}\label{sec:defprot}
  \subsection*{Functional description}
  The \texttt{Mix Now} button is used to execute a mixing run using the defined distribution and protocol. Once a mixing run has been executed, the performance graph of the run can be visualised using the \texttt{View Performance Graph} button.
  
  To define a protocol, the \texttt{Step Size} spinner can be used to change the size of the step. The \texttt{Number Of Steps} button can be used to change the number of iterations of the protocol. The \texttt{Clear Protocol} button can be used to reset the protocol to the empty protocol.
  
  To actually define the steps of the protocol, the user can drag the indicated parts of the geometry. This generates appropriate mixing steps.
  
  The \texttt{Protocols} button is used to change to the menu to manage protocols. The \texttt{Stop Defining Protocol} button is used to return to the main menu.

  \subsection*{Cautions and warnings}
  None.

  \subsection*{Formal description}
  \begin{tabularx}{\textwidth}{XXX}
    \toprule
    \emph{Operation} & \emph{Steps} & \emph{Results} \\
    \midrule
    Change the step size & Enter a different value in the spinner. & The size of future steps is changed accordingly. \\
    \midrule
    Change the number of iterations & Enter a different value in the spinner. & The number of iterations of the protocol that is executed when the mixing run is executed changes accordingly. \\
    \midrule
    View a graph of the current mixing run & Press the \texttt{View Performance Graph} button. & The \emph{View Single Graph Panel} appears. \\
    \midrule
    Add a step to the protocol & Drag one of the geometry's walls to the left or right & A new step is added. \\
    \midrule
    Close this menu & Press the \texttt{Stop Defining Protocol} button. & The main menu reappears. \\
    \midrule
    Change to the \emph{Protocols} menu & Press the \texttt{Protocols} button. & The \emph{Protocols} menu appears. \\
    \bottomrule
  \end{tabularx}
  
  %\subsection*{Examples}
  %The user can go and get pizza. Then pet a cow. Then be crushed by the cow because the cow was annoyed. By the user petting the cow. The cow is called Karel. Karel is always happy. Except when the user goes and pets Karel. Karel normally spends the day eating grass. If Karel is not eating grass, Karel is probably sleeping. Karel does not do many other things.
  
  \subsection*{Possible errors}
  None.

  \subsection*{Related operations}
  \begin{itemize}
    \item Section \ref{sec:drawingarea}
    \item Section \ref{sec:mainmenu}
    \item Section \ref{sec:protmenu}
    \item Section \ref{sec:singperfgraph}
  \end{itemize}

%-------------------------------------------------------------------------------------------------
\section{Protocols Menu}\label{sec:protmenu}
  \subsection*{Functional description}
  The \emph{Protocols} menu can be used to save, load and remove protocols. Pressing the \texttt{Save} button pops up a menu, in which the desired name can be entered. If that name already exists, the user can choose to overwrite that name, or choose a new name.
  
  Pressing the \texttt{Load} button causes a menu to appear, containing the names of all saved protocols. If there are no saved protocols, the user is notified of this fact. If there are more protocols than fit on the screen, scrollbars appear to allow the user to view all parts of list of saved protocols.
  
  The \texttt{Remove} button creates a pop-up panel containing the names of all saved protocols. Next to each of these names is a delete button that can be used to delete the protocol. Deleting a protocol cannot be undone.
  
  The \texttt{Back} button can be used to return to the main menu.

  \subsection*{Cautions and warnings}
  \begin{itemize}
    \item The \texttt{Load} and \texttt{Remove} buttons have reduced functionality if no protocols were saved.
    \item The capacity of the local storage is limited. When it is full, protocols or other saved \projectname{} items must be removed, before saving is possible again. The user is notified of a full local storage. Note that the local storage's capacity can be exceeded even though there is enough free disk space, as each website can only save a certain amount of data in the local storage.
  \end{itemize}  

  \subsection*{Formal description}
    \begin{tabularx}{\textwidth}{XXX}
    \toprule
    \emph{Operation} & \emph{Steps} & \emph{Results} \\
    \midrule
    Save a protocol & Press the \texttt{Save} button. & The \emph{Save Item Panel} appears. \\
    \midrule
    Load a protocol & Press the \texttt{Load} button. & The \emph{Load Item Panel} appears. \\
    \midrule
    Remove a protocol & Press the \texttt{Remove} button. & The \emph{Remove Item Panel} appears. \\
    \midrule
    Export an image of the protocol & Press the \texttt{Export Image} button. & A download prompt appears. \\
    \midrule
    Close this menu & Press the \texttt{Back} button. & The main menu reappears.
    \bottomrule
  \end{tabularx}

  \subsection*{Possible errors}
  None.
  
  \subsection*{Related operations}
    \begin{itemize}
    \item Section \ref{sec:defprot}
    \item Section \ref{sec:saveitem}
    \item Section \ref{sec:overwriteitem}
    \item Section \ref{sec:loaditempanel}
    \item Section \ref{sec:removeitem}
  \end{itemize}

%-------------------------------------------------------------------------------------------------
\section{Save Item Panel}
\label{sec:saveitem}
  \subsection*{Functional description}
  %What the operation(command, menu item, button, ...) achieves
  This panel contains a text box. Furthermore it provides functionality to save an item, and to cancel the operation.

  \subsection*{Cautions and warnings}
  %A list of cautions and warnings that apply to the operation
  \begin{itemize}
  \item Saving the file is only possible with a new of at least 1 character.
  \item File names can be no longer than 30 characters.
  \item Characters other then the letters of the Latin script, the numbers 0 through 9 and spaces cannot be used in file names.
  \item File names cannot start with a space.
  \end{itemize}

  \subsection*{Formal description}
  %A formal description of what the operation does and how it is used: required parameters, optional parameters, defaults, syntax and semantics
    \begin{tabularx}{\textwidth}{XXX}
    \toprule
    \emph{Operation} & \emph{Steps} & \emph{Results} \\
    \midrule
    Insert (part of) a file name & Press inside the text box. Insert any of the valid characters (see section \textbf{Cautions and warnings}). & The chosen character appears in the text box. \\
    \midrule
    Remove (part of) a file name & Press inside the text box. Press \emph{backspace}. & The last character of the name disappears. \\
    \midrule
    Save an item & Press the \emph{Save} button. & The panel closes, and a \emph{Save successful} message appears and disappears again. \\
    \midrule
    Cancel operation & Press the \emph{Cancel} button. & The panel is closed. \\
    \bottomrule
\end{tabularx}

  \subsection*{Possible errors}
  %A list of all possible errors for this operation and their causes
  \begin{description}
  \item[The message \emph{This name is already in use} is shown.] Another file has already been saved with this name. Also see section \ref{sec:overwriteitem}.
  \end{description}

  \subsection*{Related operations}
  %References to, for example, operations to complete a task or logically related operations.
   \begin{itemize}
   \item Section \ref{sec:distmenu}
   \item Section \ref{sec:resultsmenu}
   \item Section \ref{sec:protmenu}
   \item Section \ref{sec:overwriteitem}
  \end{itemize}

%-------------------------------------------------------------------------------------------------
\section{Overwrite Item Panel}
\label{sec:overwriteitem}
  \subsection*{Functional description}
  %What the operation(command, menu item, button, ...) achieves
  This panel shows the message \emph{This name is already in use. Choose whether to overwrite existing file or to cancel}. Furthermore it provides functionality to overwrite the file, or to cancel the operation.

  \subsection*{Cautions and warnings}
  %A list of cautions and warnings that apply to the operation
  \begin{itemize}
  \item After making use of the fuctionality to overwrite a file, it's not possible to retrieve the overwritten file.
  \end{itemize}

  \subsection*{Formal description}
  %A formal description of what the operation does and how it is used: required parameters, optional parameters, defaults, syntax and semantics
    \begin{tabularx}{\textwidth}{XXX}
    \toprule
    \emph{Operation} & \emph{Steps} & \emph{Results} \\
    \midrule
    Overwrite file & Press the \emph{Overwrite} button. & The panel is closed, and a \emph{Save successful} message appears and disappears again. \\
    \midrule
    Cancel operation & Press the \emph{Cancel} button. & The panel is closed, and the Save Item Panel (see section \ref{sec:saveitem}) is displayed again. \\
    \bottomrule
\end{tabularx}

  \subsection*{Possible errors}
  %A list of all possible errors for this operation and their causes
  \begin{description}
  \item[The message \emph{Local storage error} is shown.] Something went wrong with the the local storage of the browser.
  \item[The message \emph{Your storage is full. Please remove some items} is shown.] The browser can't store any more items. Remove items using the Remove Item Panel (see section \ref{sec:removeitem}).
  \item[The message \emph{An unknown error has occurred} is shown.] Cause unknown.
  \end{description}

  \subsection*{Related operations}
  %References to, for example, operations to complete a task or logically related operations.
   \begin{itemize}
   \item Section \ref{sec:distmenu}
   \item Section \ref{sec:resultsmenu}
   \item Section \ref{sec:protmenu}
   \item Section \ref{sec:saveitem}
  \end{itemize}

%-------------------------------------------------------------------------------------------------
\section{Load Item Panel}
\label{sec:loaditempanel}
  \subsection*{Functional description}
  %What the operation(command, menu item, button, ...) achieves
  This panel provides functionality for loading items. It provides functionality to load the shown items, and to close the panel.

  \subsection*{Cautions and warnings}
  %A list of cautions and warnings that apply to the operation
  \begin{itemize}
  \item This panel shows the text \emph{No saved files} when no items have been saved yet.
  \end{itemize}

  \subsection*{Formal description}
  %A formal description of what the operation does and how it is used: required parameters, optional parameters, defaults, syntax and semantics
    \begin{tabularx}{\textwidth}{XXX}
    \toprule
    \emph{Operation} & \emph{Steps} & \emph{Results} \\
    \midrule
    Load an item & Press the one of the items in the list & The selected item is loaded in the drawing area (distribution), define protocol menu (protocol) or both (mixing result). \\
    \midrule
    Close panel & Press the \emph{Close} button. & The panel is closed. \\
    \bottomrule
\end{tabularx}

  \subsection*{Possible errors}
  %A list of all possible errors for this operation and their causes
  None.

  \subsection*{Related operations}
  %References to, for example, operations to complete a task or logically related operations.
   \begin{itemize}
   \item Section \ref{sec:distmenu}
   \item Section \ref{sec:resultsmenu}
   \item Section \ref{sec:protmenu}
  \end{itemize}

%-------------------------------------------------------------------------------------------------
\section{Remove Item Panel}
\label{sec:removeitem}
  \subsection*{Functional description}
  %What the operation(command, menu item, button, ...) achieves
  This panel provides functionality for removing items. It provides functionality to remove the shown items, and to close the panel.

  \subsection*{Cautions and warnings}
  %A list of cautions and warnings that apply to the operation
  \begin{itemize}
  \item This panel shows the text \emph{No saved files} when no items have been saved yet.
  \end{itemize}

  \subsection*{Formal description}
  %A formal description of what the operation does and how it is used: required parameters, optional parameters, defaults, syntax and semantics
    \begin{tabularx}{\textwidth}{XXX}
    \toprule
    \emph{Operation} & \emph{Steps} & \emph{Results} \\
    \midrule
    Remove an item & Press the \texttt{X} next to one of the items. &  The chosen item disappears from the list, and a \emph{Delete successful} message appears and disappears again. \\
    \midrule
    Close panel & Press the \emph{Close} button. & The panel is closed. \\
    \bottomrule
\end{tabularx}

  \subsection*{Possible errors}
  %A list of all possible errors for this operation and their causes
  None.

  \subsection*{Related operations}
  %References to, for example, operations to complete a task or logically related operations.
   \begin{itemize}
   \item Section \ref{sec:distmenu}
   \item Section \ref{sec:resultsmenu}
   \item Section \ref{sec:protmenu}
  \end{itemize}
  
%-------------------------------------------------------------------------------------------------
\section{Compare Performance Panel}
\label{sec:compareperf}
  \subsection*{Functional description}
  %What the operation(command, menu item, button, ...) achieves
  This panel shows a list of all saved mixing runs. Furthermore it contains options to select one or more of these mixing runs, to compare the  performance of the selected runs and to cancel the operation.

  \subsection*{Cautions and warnings}
  %A list of cautions and warnings that apply to the operation
  \begin{itemize}
  \item This panel shows the text \emph{No saved files} and doesn't have the option to compare when no mixing runs have been saved yet. Also see section \ref{sec:resultsmenu}.
  \end{itemize}

  \subsection*{Formal description}
  %A formal description of what the operation does and how it is used: required parameters, optional parameters, defaults, syntax and semantics
    \begin{tabularx}{\textwidth}{XXX}
    \toprule
    \emph{Operation} & \emph{Steps} & \emph{Results} \\
    \midrule
    Select a result & Press a result that has a white or gray background. & The result is displayed in a white font and with a blue background. \\
    \midrule
    Deselect a result & Press a result that has a blue background. & The result is displayed in a black font and with a white or gray background. \\
    \midrule
    Compare performance of selected results & Press the \emph{Compare} button. & The \emph{Mixing Performance Graph Panel (Multiple)} is displayed. Also see section \ref{sec:mulperfgraph}. \\
    \midrule
    Cancel operation & Press the \emph{Cancel} button. & The panel is closed. \\
    \bottomrule
\end{tabularx}

  \subsection*{Possible errors}
  %A list of all possible errors for this operation and their causes
  None.

  \subsection*{Related operations}
  %References to, for example, operations to complete a task or logically related operations.
   \begin{itemize}
   \item Section \ref{sec:resultsmenu}
   \item Section \ref{sec:mulperfgraph}
  \end{itemize}

%-------------------------------------------------------------------------------------------------
\section{Mixing Performance Graph Panel (Single)}
\label{sec:singperfgraph}
  \subsection*{Functional description}
  %What the operation(command, menu item, button, ...) achieves
  This panel shows a figure with the performance graph of the last executed mixing run. Furthermore it contains options to export this figure and to close this panel.

  \subsection*{Cautions and warnings}
  %A list of cautions and warnings that apply to the operation
  \begin{itemize}
  \item The figure of the performance graphs doesn't display in Internet Explorer (all versions).
  \end{itemize}

  \subsection*{Formal description}
  %A formal description of what the operation does and how it is used: required parameters, optional parameters, defaults, syntax and semantics
    \begin{tabularx}{\textwidth}{XXX}
    \toprule
    \emph{Operation} & \emph{Steps} & \emph{Results} \\
    \midrule
    View performance of a single point & Press one of the points in the graph. & A popup appears with the performance value of this point. \\
    \midrule
    Export graph & Press the \emph{Export graph} button. & The default download window of your browser is displayed. \\
    \midrule
    Close panel & Press the \emph{Close} button. & The panel is closed. \\
    \bottomrule
\end{tabularx}

  \subsection*{Possible errors}
  %A list of all possible errors for this operation and their causes
  None.

  \subsection*{Related operations}
  %References to, for example, operations to complete a task or logically related operations.
   \begin{itemize}
   \item Section \ref{sec:defprot}
  \end{itemize}

%-------------------------------------------------------------------------------------------------
\section{Mixing Performance Graph Panel (Multiple)}
\label{sec:mulperfgraph}
  \subsection*{Functional description}
  %What the operation(command, menu item, button, ...) achieves
  This panel shows a figure with the performance graphs of one or more mixing runs. Furthermore it contains options to export this figure, to create a new figure with the performance of different mixing runs, and to close this panel.

  \subsection*{Cautions and warnings}
  %A list of cautions and warnings that apply to the operation
  \begin{itemize}
  \item The figure of the performance graphs doesn't display in Internet Explorer (all versions).
  \end{itemize}

  \subsection*{Formal description}
  %A formal description of what the operation does and how it is used: required parameters, optional parameters, defaults, syntax and semantics
    \begin{tabularx}{\textwidth}{XXX}
    \toprule
    \emph{Operation} & \emph{Steps} & \emph{Results} \\
    \midrule
    View performance of a single point & Press one of the points in the graph. & A popup appears with the performance value of this point. \\
    \midrule
    Export graph & Press the \emph{Export graph} button. & The default download window of your browser is displayed. \\
    \midrule
    Create new graph & Press the \emph{New comparison} button. & The panel from section \ref{sec:compareperf} is displayed. \\
    \midrule
    Close panel & Press the \emph{Close} button. & The panel is closed. \\
    \bottomrule
\end{tabularx}

  \subsection*{Possible errors}
  %A list of all possible errors for this operation and their causes
  \begin{description}
  \item[The panel shows a blank field with the text \emph{No data}] A comparison was made with no results selected. Also see section \ref{sec:compareperf}.\\
  \end{description}

  \subsection*{Related operations}
  %References to, for example, operations to complete a task or logically related operations.
   \begin{itemize}
   \item Section \ref{sec:resultsmenu}
   \item Section \ref{sec:compareperf}
  \end{itemize}

%-------------------------------------------------------------------------------------------------
\section{Toggle Menu Button}
\label{sec:togmenu}
  \subsection*{Functional description}
  %What the operation(command, menu item, button, ...) achieves
  This button is always shown in the top right corner of the application. It can be used to hide the menu bar when it is shown, and to show it again when it is hidden.

  \subsection*{Cautions and warnings}
  %A list of cautions and warnings that apply to the operation
  None.

  \subsection*{Formal description}
  %A formal description of what the operation does and how it is used: required parameters, optional parameters, defaults, syntax and semantics
  \begin{tabularx}{\textwidth}{XXX}
    \toprule
    \emph{Operation} & \emph{Steps} & \emph{Results} \\
    \midrule
    Hide menu & Press the blue circular button with a white \texttt{X} sign. & The menu bar on the right side of the application slides to the right until it isn't visible anymore. Simulatenously the blue circular button rotates until it shows a \texttt{+} sign.\\
    \midrule
    Show menu & Press the blue circular button with a white \texttt{+} sign. & The menu bar on the right side of the application appears and slides to the left until it's entirely visible. Simulatenously the blue circular button rotates until it shows an \texttt{X} sign. \\
    \bottomrule
\end{tabularx}

  \subsection*{Possible errors}
  %A list of all possible errors for this operation and their causes
  None.

  \subsection*{Related operations}
  %References to, for example, operations to complete a task or logically related operations.
  \begin{itemize}
  \item Section \ref{sec:drawingarea}
  \item Section \ref{sec:mainmenu}
  \item Section \ref{sec:selecttoolmenu}
  \item Section \ref{sec:distmenu}
  \item Section \ref{sec:resultsmenu}
  \item Section \ref{sec:defprot}
  \item Section \ref{sec:protmenu}
  \end{itemize}
