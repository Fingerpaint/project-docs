\chapter{Introduction}
%\todo{short chapter intro}

\section{Intender readership}
%\todo{User categories (end user, operator), level of experience assumed, which sections are most relevant}
This document is intended for all end-users of the \applicationname. If accessing the application from a mobile device basic knowledge about interacting with touch-devices is assumed.

\section{Applicability}
%\todo{Software releases the SUM applies to}
This document applies to the latest release of the \applicationname, which is release 1.0.

\section{Purpose}
%\todo{Purpose of the SUM and the software}
The \applicationname serves as an educational tool for anyone who wants to gain a deeper understanding of the process of mixing in general, and in particular for students at the TU/e.

\section{How to use this document}
%\todo{What each section contains and the relationship between sections}
First-time users are encouraged to read chapter 3 and follow each tutorial to develop a basic understanding of the \applicationname.
More experienced users can use the reference in chapter 4 to find out more about specific features of \projectname.

\section{Related documents}
%\todo{All applicable documents}
\begin{tabular}{l l}
URD & The User Requirements Document for the \applicationname.\\
\end{tabular}

\section{Conventions}
%\todo{All symbols, stylistic conventions and command syntax conventions used}
\begin{tabular}{l p{10cm}}
\emph{Italics} & All words in \emph{italic text} are names of buttons or features you can find in the \applicationname.\\
\end{tabular}

\section{Problem reporting}
%\todo{A summary how to report software problems}
Since \projectauthor{} will be dissolved after completion of the \projectname project, problems cannot be reported to \projectauthor{} and are not our responsibility.


