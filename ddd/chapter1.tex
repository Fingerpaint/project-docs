\chapter{Introduction}

\section{Purpose}
\todo{The purpose of this particular SRD and its intended readership}

\section{Scope}
\todo{Scope of the software. Identifies the product by name, explains what the software will do}

\section{List of definitions}
\begin{tabular}{l|l}
2IP35 & The Software Engineering Course \\
Client & prof.dr.ir. P.D. Anderson \\
CM    &Configuration Manager \\
CPR & Capability Requirement \\
CNR & Constraint Requirement \\
TU/e  &Eindhoven University of Technology \\
SEP   &Software Engineering Project \\
SR    &Software Requirements \\
URD   &User Requirements Document \\
TBC & To Be Confirmed \\
TBD & To Be Defined \\
\todo{add more if needed} & \\
\end{tabular}

\section{List of references}

\bibliographystyle{plain}

\bibliography{../ref}

\section{Overview}
\todo{Short description of the rest of the SRD and how it is organized.}

%The remaining chapters describe the system requirements in more detail. Chapter 2 gives a general description of 
%\begin{itemize}
%\item 2.1 - The relation to other systems,
%\item 2.2 - The main capabilities,
%\item 2.3 - Constraint information and justification,
%\item 2.4 - User charactaristics,
%\item 2.5 - The operational environment, and
%\item 2.6 - Assumptions and dependencies.
%\end{itemize}

%Chapter 3 gives a detailed list of the system's capabilitiy requirements in section 3.1, and a list of the constraint requirements is given in section 3.2.


