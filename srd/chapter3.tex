\chapter{Specific requirements}
The specific requirements discussed in this section are divided into logical subsections, with one subsection for every public function we use. Methods are discussed in a subsection. \\
For prioritizing the specific requirements for \projectname we will use to the MoSCoW model. The capital letters in MoSCoW stand for:
\begin{itemize}
\item[M] \emph{Must have}; these requirements are essential for the functionality of our product.
\item[S] \emph{Should have}; these frequirements generally form the core of the product and are essential for an important part of the functionality.
\item[C] \emph{Could have}; these requirements generally provide optional functionality. They are thus not essential for the success of the product, but provide good functionality.
\item[W] \emph{Won't have}; these requirements are not essential and likely won't be implemented in this project. They could possibly be implemented in extentions of this project.
\end{itemize}
\todo{chapter description}

\section{Functional requirements}

\subsection{defining a mixing protocol}
This function allows a user to specify a mixing protocol in a format the backend can process.



\todo{individual sections for Functional requirements}
\todo{A list of all functional requirements (what should the system do).}

\todo{individual sections for non-functional requirements}
\todo{A list of all non functional requirements (performance, interface, operational, resource, verification/testing, portability, maintainability, reliability, security, safety, documentation, other,...), linked to functional requirements. Each category of non-functional requirements has its own subsection.}