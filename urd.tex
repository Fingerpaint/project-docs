\documentclass[11pt,a4paper]{report}

\usepackage{amsmath, amsfonts, amssymb, array, a4wide, fancyhdr, lineno, graphicx, epsfig, soul, color, hyperref}
\usepackage[square, numbers]{natbib}

% Running line numbers:
\linenumbers
% Number only every 5:th line:
\modulolinenumbers[5]

%Nodig om een bibliography midden in het artikel te zetten, ipv aan het einde zoals eigenlijk gebruikelijk is
\renewcommand{\bibsection}{}
\bibliographystyle{ieeetr}

% TODO command
\newcommand{\todo}[1]{
    \hl{#1}
}

% The things that should be filled in by each group, depending on their situation, are written in a todo command, \todo{like this text}. All text in normal the normal font, is applicable for any group. However, everyone is free to adapt any text, and it is even suggested to look at all text critically and make changes if needed.

% Project specific commands
\newcommand{\projectauthor}{Group Fingerpaint\ }
\author{\projectauthor}
\newcommand{\projectname}{Fingerpaint\ }
\newcommand{\applicationname}{Fingerpaint application\ }

\newcommand{\aanpassen}[1]{ {\sethlcolor{green} \hl{#1}} }

\title{User Requirements Document}

%Variables
\newcommand{\TitelAbbr}{SQAP}
\newcommand{\Version}{0.0}

\begin{document}

\maketitle

\begin{abstract}
This is the User Requirements Document for the Software Engineering Project. This document is based on the ESA standard for software development and the work of many previous SEP groups.
\end{abstract}

\tableofcontents 
\chapter*{Document Status Sheet}
\section*{Document status overview}
\subsection*{General}
\begin{tabular}[!]{ll}
    Document title:     &   User Requirements Document \\
    Identification:     &   \todo{Choose some id, e.g.: \TitelAbbr\Version.pdf}\\
    Author:             &   \\
    Document status:    &   Draft\\
\end{tabular}

\subsection*{Document history}
\begin{tabular}[!]{|l|l|l|l|}
    \hline
    \emph{Version}    &   \emph{Date} & \emph{Author} &  \emph{Reason of change}\\
    \hline
    \todo{version}    &   \todo{date}  &  \todo{author} &  \todo{reason} \\    
    \hline
\end{tabular}

\clearpage

\section*{Document Change Records since previous issue}
\subsection*{General}
\begin{tabular}[!]{ll}
    Datum:          &   2012-07-09 \\
    Document title: &   Software Quality Assurance Plan\\
    Identification:  &   \TitelAbbr\Version.pdf\\
\end{tabular}

\subsection*{Changes}
\begin{tabular}[!]{|l|l|p{8cm}|}
    \hline
    \emph{Page} &   \emph{Paragraph}    &   \emph{Reason to change}\\
    \hline
    \todo{ \texttt{pageref} } & \todo{ \texttt{ref} }      & \todo{reason} \\
    \hline
\end{tabular} 

\chapter{Introduction}

\section{Purpose}
This document describes the procedures and control methods to obtain the desired quality level of the end products and the process by which these end products are created. This document serves as a guide for the managers and developers of the \projectname project. All team members must read this document and apply the procedures stated in it. The document applies to all phases of software development as defined in the Project Management Plan \cite{spmp}. Detailed information about the software quality assurance activities for these phases will be added in appendices during the project.

\section{Scope}
\todo{A list of software products to be developed and their intended use.}
\section{List of definitions}
\begin{tabular}{l|l}
2IP35 & The Software Engineering Course \\ 
AD    &Architectural Design \\ 
ADD   &Architectural Design Document \\ 
AT    &Acceptance Test \\ 
ATP   &Acceptance Test Plan \\ 
Client & \todo{The client} \\ 
CM    &Configuration Manager \\ 
DD    &Detailed Design \\ 
DDD   &Detailed Design Document \\ 
ESA   &European Space Agency \\ 
TU/e  &Eindhoven University of Technology \\ 
OM    &Operations and Maintenance Plan \\ 
PM    &Project Manager \\ 
QM    &Quality Manager \\ 
SCMP  &Software Configuration Management Plan \\ 
SEP   &Software Engineering Project \\ 
SL    &Software Librarian \\ 
SPMP  &Software Project Management Plan \\ 
SQAP  &Software Quality Assurance Plan \\ 
SR    &Software Requirements \\ 
SRD   &Software Requirements Document \\ 
STD   &Software Transfer Document \\ 
SUM   &Software User Manual \\ 
SVVP  &Software Verification and Validation Plan \\ 
SVVR  &Software Verification and Validation Report \\ 
TR    &Transfer phase \\ 
UR    &User Requirements \\ 
URD   &User Requirements Document \\ 
VPM   &Vice Project Manager \\ 
\end{tabular}
\section{List of references}

\bibliographystyle{plain}
TODO: only all applicable documents! \\
\bibliography{../ref}

\section{Overview}
Short description of the rest of the SRD and how it is organized.

\chapter{General description}
\section{Relation to current projects}
The context of this project in relation to other current projects.

\section{Relation to predecessor and successor projects}
The context of this project in relation to past and future projects.

\section{Function and purpose}
A general overview of the function and purpose of the product.

\section{Environment}
Hardware and operating system of target system and development system. Who will use the system (user roles in URD).

\section{Relation to other systems}
Is the project an independent system, part of a larger system, replacing another system? The essential characteristics of these other systems.

\section{General constraints}
Reasons why constraints exist: background information and justification (analogues to URD).

\section{Model description}
A description of the logical model.

\chapter{Specific requirements}
\section{Functional requirements}
A list of all functional requirements (what should the system do).

\begin{center}
\begin{tabular}{ p{0.85\textwidth} p{0.15\textwidth}}

01 & could have \\
\multicolumn{2}{p{\textwidth}}{Users can set a geometry for the canvas} \\
\hline

02 & must have \\
\multicolumn{2}{p{\textwidth}}{Users can define a initial concentration distribution with black and white} \\
\hline

03 & could have \\
\multicolumn{2}{p{\textwidth}}{Users can choose which two colors are used for the initial concentration distribution} \\
\hline

04 & should have? \\
\multicolumn{2}{p{\textwidth}}{Users can define a initial concentration distribution with more than two different colors} \\
\hline

05 & must have \\
\multicolumn{2}{p{\textwidth}}{Users can define a mixing protocol for a rectangular geometry as a sequence of movements of the upper and lower walls}\\
\hline

06 & could have \\
\multicolumn{2}{p{\textwidth}}{Users can define a mixing protocol for a non-rectangular geometry as a sequence of movements that are applicable to the geometry}\\
\hline

07 & must have \\
\multicolumn{2}{p{\textwidth}}{Users can define a step to indicate the timeperiod that each movement from the mixing protocol is applied}\\
\hline

08 & could have \\
\multicolumn{2}{p{\textwidth}}{Users can define a different step for each separate movement in the mixing protocol}\\
\hline

09 & must have \\
\multicolumn{2}{p{\textwidth}}{Users can view an image of the endresult of applying the mixing protocol on the initial concentration distribution} \\
\hline

10 & should have \\
\multicolumn{2}{p{\textwidth}}{Users can save the image from 06 locally to their device, without losing transparency (i.e. PNG or GIF format)} \\
\hline

11 & should have \\
\multicolumn{2}{p{\textwidth}}{Users can remove previously stored images from their device} \\
\hline

12 & should have \\
\multicolumn{2}{p{\textwidth}}{Users can view an animation of applying the mixing protocol on the initial concentration distribution} \\
\hline

13 & should have \\
\multicolumn{2}{p{\textwidth}}{Users can save the animation from 09 locally to their device, without losing transparency (i.e. APNG or AGIF format} \\
\hline

14 & should have \\
\multicolumn{2}{p{\textwidth}}{Users can remove previously stored animations from their device} \\
\hline

15 & should have \\
\multicolumn{2}{p{\textwidth}}{Users can view the mixing performance of the mixing protocol in a graph} \\
\hline

16 & should have \\
\multicolumn{2}{p{\textwidth}}{Users can save the performance results locally on their device} \\
\hline

17 & should have \\
\multicolumn{2}{p{\textwidth}}{Users can retrieve the performance results that are stored locally on their device} \\
\hline

18 & should have \\
\multicolumn{2}{p{\textwidth}}{Users can retrieve performance results from multiple mixing protocols simultaneously, after which they are depicted in one graph} \\
\hline

19 & should have \\
\multicolumn{2}{p{\textwidth}}{Users can remove performance results that are stored on their device} \\
\hline

\end{tabular}
\end{center}

\section{Non-functional requirements}
A list of all non-functional requirements (performance, interface, operational, resource, verification/testing, portability, maintainability, reliability, security, safety, documentation, other, ...), linked to functional requirements. Each category of non-functional requirements has its own subsection.

\begin{center}
\begin{tabular}{ p{0.85\textwidth} p{0.15\textwidth}}

01 & must have \\
\multicolumn{2}{p{\textwidth}}{The interface contains a canvas which represents the mixing area} \\
\hline

02 & should have \\
\multicolumn{2}{p{\textwidth}}{The user can define the initial concentration distribution by painting on the canvas with his/her finger} \\
\hline

03 & must have \\
\multicolumn{2}{p{\textwidth}}{The interface contains an easy to use input element to define the sequence of movements of the mixing protocol (i.e. a button or by swiping over the screen)} \\
\hline

04 & must have \\
\multicolumn{2}{p{\textwidth}}{The interface contains a numberfield to set the step parameter of the mixing protocol} \\
\hline

05 & must have \\
\multicolumn{2}{p{\textwidth}}{Waiting time between submitting input and receiving output should not be more than 5 seconds} \\
\hline

06 & should have \\
\multicolumn{2}{p{\textwidth}}{Waiting time between submitting input and receiving output should not be more than 3 seconds} \\
\hline

01 & could have \\
\multicolumn{2}{p{\textwidth}}{Waiting time between submitting input and receiving output should not be more than 1 seconds} \\
\hline

\end{tabular}
\end{center}

\chapter{Requirements traceability matrix}
A table showing how each user requirement of the URD is linked to software requirements in the SRD.

% ============================================================
% Appendices
% ============================================================

\end{document}