\chapter{Specific requirements}
\label{chap:specreq}
\todo{chapter description}
The structure of the software requirements in this section are conform the user requirements described in chapter 3 of the URD \cite{urd}. \\
Note that as \applicationname\ is developed using GWT, some of the software requirements listed in this chapter are described in terms of specific GWT widgets or panels\footnote{The website for the GWT widgetlist can be found at \url{https://developers.google.com/web-toolkit/doc/latest/RefWidgetGallery}.}.

\todo{Nog even vermelden dat we de won't haves uit het URD niet beschrijven in dit SRD.}

\section{Functional requirements}
\label{sec:funcreq}

Requirements regarding the mixer geometry input: \\
\SRQ{must have}{The user can select ``rectangle" as a geometry, via the first column in a Cell Browser.} % CPR1
\SRQ{must have}{After having selected a geometry, the user can select a mixer that fits the selected geometry. The mixer can be selected via a second column in a Cell Browser.} % CPR2
\SRQ{should have}{The user can select ``square" as a geometry, via the first column in a Cell Browser.}  % CPR3
\SRQ{could have}{The user can select ``circle" as a geometry, via the first column in a Cell Browser.}  % CPR4
\SRQ{could have}{The user can select ``\emph{Journal Bearing}" as a geometry, via the first column in a Cell Browser.}  % CPR5

Requirements regarding the initial concentration distribution input: \\

%VRAAG: moeten we er een losse requirement van maken dat de user op een canvas tekent? Ik vind het uit onderstaande
% requirements nu niet geheel duidelijk dat het om exact hetzelfde canvas gaat iedere keer.

\SRQ{must have}{The user can select a colour (black or white) to paint with for the initial concentration distribution, via a Toggle Button.} % CPR6
\SRQ{must have}{The user can define an initial concentration distribution with the selected colour on the canvas, by drawing with their finger.} % CPR6
\SRQ{must have}{The user can select a circle-shaped drawing tool via a pop-up panel.} % CPR7
\label{srq:circleshaped}
\SRQ{must have}{The user can draw with a circle-shaped drawing tool on the canvas.} % CPR7, los omdat de square een lagere prioriteit heeft in het URD
\SRQ{must have}{The user can reset the current concentration distribution to a completely white concentration distribution, by clicking on the ``Reset" button.} % CPR8
\SRQ{should have}{The user can select a square-shaped drawing tool via a pop-up panel.} % CPR9
\label{srq:squareshaped}
\SRQ{should have}{The user can draw with a square-shaped drawing tool on the canvas.} % CPR9
\SRQ{should have}{The user can adjust the size of the drawing tool, using a slider in the Popup Panel from \ref{srq:circleshaped} and \ref{srq:squareshaped}. % CPR10
\SRQ{should have}{The user can save an initial concentration distribution locally on their device, using a Button and Popup Panel.} % CPR11
\SRQ{should have}{The user can remove a previously saved initial concentration distribution.} % CPR12
% Hetzelfde als Roel hier, een button met ``Remove", dan een lijst met checkboxes en dan met dezelfde knop Removen?



Requirements regarding the presentation and exportation of results: \\
\SRQ{should have}{The user can use a 'load' button to open a popup. In this popup he can select the files he wants to load. The 'load' button changes to 'load selected' and now loads all the selected files. The loaded data is then depicted in one graph.} \\
%CPR36

\SRQ{should have}{The user can use a 'export graph' button`to export the current result graph as a picture. If this button is used, the graph is exported as an SVG using the browser's native exporting support.} \\
%CPR37

\SRQ{could have}{After the  user has defined a mixing run, he can press the button 'visualise mixing'. After this button is pressed, the application processes the mixing run, together with the mixer data and responds by showing an animation of this mixing simulation to the user.} \\
%CPR38

\SRQ{could have}{After the user has visualised his mixing run (as described above), he can press the 'export animation' button. When this button is pressed, the animation is exported as an AVG using the browser's native exporting support.} \\
%CPR39

\section{Non-functional requirements}
\label{sec:nonfuncreq}
\todo{individual sections for non-functional requirements}

\subsection{Performance}
\begin{center}
\begin{tabular}{ >{\bfseries}p{0.84\textwidth} >{\itshape}p{0.16\textwidth}}
SCR & must have \\
\multicolumn{2}{p{\textwidth}}{Waiting time between submitting input and receiving output is no longer than 5 seconds.} \\
\hline

SCR & should have \\
\multicolumn{2}{p{\textwidth}}{Waiting time between submitting input and receiving output is no longer than 3 seconds.} \\
\hline

SCR & could have \\
\multicolumn{2}{p{\textwidth}}{Waiting time between submitting input and receiving output is no longer than 1 seconds.} \\
\hline

\end{tabular}
\end{center}
\subsection{Interface}

\SRQ{could have}{In the menu bar a flag of the currently selected language is visible. When clicked, a popup appears containing several language options, each symbolised by the flag of their country. These flags can be clicked to change the language to that specific language.} \\
%CPR40&41

\todo{Hier zegguh dat wij Fortran servertje gebruikuh.}

\subsection{Operational}
\subsection{Resource}
\subsection{Verification and testing}
\subsection{Portability}
\begin{center}
\begin{tabular}{ >{\bfseries}p{0.84\textwidth} >{\itshape}p{0.16\textwidth}}
SCR & must have \\
\multicolumn{2}{p{\textwidth}}{The application runs on iOS Safari version 6.0 and higher.} \\
\hline

SCR & should have \\
\multicolumn{2}{p{\textwidth}}{The application runs on Firefox version 20 and higher, and Google Chrome version 26 and
higher.} \\
\hline

SCR & could have \\
\multicolumn{2}{p{\textwidth}}{The application runs on Internet Explorer version 10 and higher, Opera version 12.1 and
higher and Safari version 6.0 and higher.} \\
\hline

SCR & must have \\
\multicolumn{2}{p{\textwidth}}{The application runs on devices running on iOS version 6 and higher.} \\
\hline

SCR & should have \\
\multicolumn{2}{p{\textwidth}}{The application runs on devices running on Android version 4.0 and higher.} \\
\hline

SCR & could have \\
\multicolumn{2}{p{\textwidth}}{The application runs on devices running on Windows 8.} \\
\hline

\end{tabular}
\end{center}
\subsection{Maintainability}
\subsection{Reliability}
\subsection{Security}
\subsection{Safety}
\subsection{Documentation}
\subsection{Extensibility}
\begin{center}
\begin{tabular}{ >{\bfseries}p{0.84\textwidth} >{\itshape}p{0.16\textwidth}}
SCR & must have \\
\multicolumn{2}{p{\textwidth}}{The application should be easily extendable with new mixers.} \\
\hline
\end{tabular}
\end{center}
