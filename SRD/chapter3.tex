\chapter{Specific requirements}
\label{chap:specreq}
This chapter lists the requirements the \projectname application should fulfill once it has been completed. Each requirement has a specific priority, based on the MoSCoW model \cite{moscow}:

\begin{itemize}
    \item \emph{must have}; requirements with this priority are essential for the product, and must be implemented.
    \item \emph{should have}; requirements with this priority are not essential for the product to work. However, they are nearly as important as the \emph{must have}'s and are therefore expected to be implemented.
    \item \emph{could have}; requirements with this priority are a nice addition to the product, and may be implemented, if time and budget allow this.
    \item \emph{won't have}; requirements with this priority will not be implemented in this version of the product, but may be nice to implement in future versions.
\end{itemize}

The structure of the software requirements in this section are conform the user requirements described in chapter 3 of the URD \cite{urd}. \\
Note that as \applicationname\ is developed using GWT, some of the software requirements listed in this chapter are described in terms of specific GWT widgets or panels\footnote{The website for the GWT widgetlist can be found at \url{https://developers.google.com/web-toolkit/doc/latest/RefWidgetGallery}.}.

\todo{Nog even vermelden dat we de won't haves uit het URD niet beschrijven in dit SRD.}
\todo{Requirements groeperen naar: MixerProvider, MiserPersistance, Client, ClientPersistence, SimulatorServer}

\section{Functional requirements}
This section lists the functional requirements for the \projectname app.

The user can select a rectangular mixer geometry. To this end, a cell browser is available that lists all the available geometries in the first column. This first column will contain several geometries, including ``rectangle", `` square", ``circle" or ``\emph{Journal Bearing}". After clicking on a certain geometry, the available mixers for this geometry are displayed in the second column of the cell browser. The user can now select a mixer of choice and then, the third column in the cell browser lists all the available starting concentration distributions for the selected mixer. There are three types of starting concentration distributions that the user can choose from. The first type is a ``blank" concentration distribution, which simply means that the user will be presented with a clean canvas when this option is selected. The second type is a ``load" option, so the user can load previously saved concentration distributions that are stored on their device. The third type of concentration distribution is a ``predefined" distribution, meaning that the user can choose from a few predefined distributions that are already present in the application. In case of the second and third option, there will be a third column in the cell browser that lists all the available concentration distributions that can be loaded. When the user clicks on an item in the third column, the required distribution is immediately loaded on an appropriate canvas. For the first option, a blank canvas is displayed on the screen. \\
This user requirement can then be described with the following software requirements:

\SRQ[srq:selgeomrec]{must have}{The user can select ``rectangle" as a geometry, via the first column in a cell browser.} % CPR1
\SRQ[srq:selmixer]{must have}{After having selected a geometry, the user can select a mixer that fits the selected geometry. The mixer can be selected via a second column in a cell browser.} % CPR2
\SRQ[srq:selgeomsq]{should have}{The user can select ``square" as a geometry, via the first column in a cell browser.}  % CPR3
\SRQ[srq:selgeomcir]{could have}{The user can select ``circle" as a geometry, via the first column in a cell browser.}  % CPR4
\SRQ[srq:selgeomjb]{could have}{The user can select ``\emph{Journal Bearing}" as a geometry, via the first column in a cell browser.}  % CPR5

Requirements regarding the initial concentration distribution input: \\

\todo{VRAAG: moeten we er een losse requirement van maken dat de user op een canvas tekent? Ik vind het uit onderstaande
 requirements nu niet geheel duidelijk dat het om exact hetzelfde canvas gaat iedere keer.}
\todo{VRAAG: moeten we naar GWT componenten verwijzen met hoofdletters of niet?}

\SRQ{must have}{The user can select a colour (black or white) to paint with for the initial concentration distribution, via a Toggle Button.} % CPR6
\SRQ{must have}{The user can define an initial concentration distribution with the selected colour on the canvas, by drawing with their finger.} % CPR6
\SRQ[circleshaped]{must have}{The user can select a circle-shaped drawing tool via a pop-up panel.} % CPR7
\SRQ{must have}{The user can draw with a circle-shaped drawing tool on the canvas.} % CPR7, los omdat de square een lagere prioriteit heeft in het URD
\SRQ{must have}{The user can reset the current concentration distribution to a completely white concentration distribution, by clicking on the ``Reset" button.} % CPR8
\SRQ[squareshaped]{should have}{The user can select a square-shaped drawing tool via a pop-up panel.} % CPR9
\SRQ{should have}{The user can draw with a square-shaped drawing tool on the canvas.} % CPR9
\SRQ{should have}{The user can adjust the size of the drawing tool, using a slider in the popup panel from \srqref{circleshaped} and \srqref{squareshaped}}. % CPR10
\SRQ{should have}{The user can save an initial concentration distribution locally on their device, using a button and popup panel.} % CPR11

The user can remove a previously saved initial concentration distribution. To this end, a ``Remove saved distributions" button is available. When the user clicks this button, a new screen opens with a cell list that shows all the available saved concentration distributions, along with a check box next to each item. The ``Remove saved distributions" button will remain visible in this new screen. The user can then select one or more items to be removed using the check boxes. When the remove button is pressed, all the selected items from the cell list will be removed and the user returns to the previous screen. \todo{Moet ik hier al iets zeggen over de verschillende schermen?} \\
This user requirement can then be described with the following software requirements:

\SRQ{should have}{There is a ``Remove saved distributions" button to remove previously saved distributions.} % CPR12
\SRQ{should have}{After clicking the ``Remove saved distributions" button, a cell list is shown that provides all the previously saved concentration distributions.} % CPR12
\SRQ{should have}{Each item from in the cell list of PREVIOUS contains a check box that can be selected or deselected, to indicate which distributions should be removed.} % CPR12
\SRQ{should have}{After selecting at least one item from PREVIOUS, the user can press the ``Remove saved distributions" button and the selected items are removed from the device.} % CPR12

The user can select an initial concentration distribution from a list of previously saved distributions.

\SRQ{should have}{} % CPR13

Requirements regarding the presentation and exportation of results: \\
\SRQ{should have}{The user can use a 'load' button to open a popup. In this popup he can select the files he wants to load. The 'load' button changes to 'load selected' and now loads all the selected files. The loaded data is then depicted in one graph.}
%CPR36

\SRQ{should have}{The user can use a 'export graph' button`to export the current result graph as a picture. If this button is used, the graph is exported as an SVG using the browser's native exporting support.}
%CPR37

\SRQ{could have}{After the  user has defined a mixing run, he can press the button 'visualise mixing'. After this button is pressed, the application processes the mixing run, together with the mixer data and responds by showing an animation of this mixing simulation to the user.}
%CPR38

\SRQ{could have}{After the user has visualised his mixing run (as described above), he can press the 'export animation' button. When this button is pressed, the animation is exported as an AVG using the browser's native exporting support.}

\SRQ{must have}{After the user has defined a mixing run, he can press the button `start mixing'. When this button is pressed, the result of the user's mixing protocol is shown.}
%CPR32

\SRQ{should have}{After the user pressed the `start mixing' button and the result is shown, he can press the `export picture' button. When this button is pressed, the result picture is exported as an SVG using the browser's native exporting support.}
%CPR33

\SRQ{must have}{After the user has defined a mixing run, he can press the button `start mixing'. When this button is pressed, the result of the user's mixing protocol is shown. Now the user can press a `show statistics' button. When pressed, a graph containing the mixing performance is shown.}
%CPR34

\SRQ{should have}{After the user has pressed the `show statistics' button and the mixing performance is shown, he can press the `export picture' button. When this button is pressed, the result picture is exported as an SVG using the browser's native exporting support.}
%CPR35

\SRQ{should have}{The user can use a `load' button to open a popup. In this popup he can select the files he wants to load. The `load' button changes to `load selected' and now loads all the selected files. The loaded data is then depicted in one graph.}
%CPR36

\SRQ{should have}{The user can use a `export graph' button`to export the current result graph as a picture. If this button is used, the graph is exported as an SVG using the browser's native exporting support.}
%CPR37

\SRQ{could have}{After the  user has defined a mixing run, he can tick a box titled `visualise mixing'. When this box is checked and he presses `start mixing', the application processes the mixing run, together with the mixer data and responds by showing an animation of this mixing simulation to the user.}
%CPR38

\SRQ{could have}{After the user has visualised his mixing run (as described above), he can press the `export animation' button. When this button is pressed, the animation is exported as an AVG using the browser's native exporting support.}
%CPR39

\SRQ{could have}{In the menu bar a flag of the currently selected language is visible. When clicked, a popup appears containing several language options, each symbolised by the flag of their country. These flags can be clicked to change the language to that specific language. Standard language is English.}
%CPR40&41

\subsubsection{Set options for a mixing protocol}
A duration can be any value that can be created by combining the possible values for $D$ (the step size), namely 4, 2, 1, 0.5, 0.25 and 0.1. Such a value $D$ can be entered in a text box near the drawing canvas. If an invalid value is entered -- i.e. a value that cannot be created by combining the mentioned values -- the user is prompted to enter a different value.

To indicate how many times the protocol created using the above items is executed, the user can enter a number in a spinner. We use a spinner because the number it contains can easily be incremented or decremented by using the arrow keys, and a value that is not near the original value can easily be entered using its text box.

To manage mixing protocols, the user can save and load the mixing protocols they have created. It is also possible to load a predefined mixing protocol.

\SRQ[srq:stepsize]{must have}{ % CPR20
    The user can enter the step size in a text box.
}

\SRQ{must have}{ % CPR21
    The user can indicate via a spinner how many times the protocol should be applied.
}

\SRQ{should have}{ % CPR22
    The user can click the \emph{Save} button to save the mixing protocol to their device. The desired name for the protocol can then be specified in a pop-up.
}

\SRQ{should have}{ % CPR23
    The user can press the \emph{Remove} button to remove the protocol from the device.
}

\SRQ{should have}{ % CPR24
    After the user has selected the geometry and the mixer using \srqref{srq:selgeomrec}, \srqref{srq:selgeomsq}, \srqref{srq:selgeomcir} or \srqref{srq:selgeomjb} and \srqref{srq:selmixer}, they can choose to load a previously saved mixing protocol by choosing the \emph{Load} option in the third level of the menu. This spawns a fourth menu level with the names of all applicable saved protocols. Clicking one such protocol loads it.
}

\SRQ{could have}{ % CPR28
    After \srqref{srq:selgeomrec}, \srqref{srq:selgeomsq}, \srqref{srq:selgeomcir} or \srqref{srq:selgeomjb} and \srqref{srq:selmixer}, the user selects the \emph{Predefined} option in the third level of the \emph{Mixer} menu. This spawns a fourth level of the menu containing all suitable predefined mixing protocols for the selected geometry. Clicking one such protocol selects it, which means it will be used once the mixing is executed.
}

\subsubsection{Define mixing protocol for specific geometries}
For the rectangular or square geometries, a wall movement is defined by either $T$ or $B$, denoting a movement of the top or bottom wall, respectively. Each movement is performed for $D$ time units, which can be set using \srqref{srq:stepsize}. Looking directly at the screen, a $T$ movement indicates that the top wall should move to the right, and a $B$ movement indicates that the bottom wall should move to the left. Each of these wall movements can be combined with a `$-$' sign, to indicate that the direction of movement should be reversed. This means $-T$, for example, indicates that the top wall should move to the \emph{left}.

The circular geometry does not support movements, and as such a mixing protocol is immediately defined after selecting a mixer.

For the \emph{Journal bearing} geometry, the user can swipe across the screen to indicate movements. Swiping to the right near the top of the outer circle means the outer circle rotates clockwise for $D$ time units, and swiping to the right near the top of the inner circle means the inner circle rotates clockwise for $D$ time units.

\SRQ{must have}{ % CPR17
    The \emph{Step} class contains a movement and a duration for this movement.
}

\SRQ{must have}{ % CPR17
    The \emph{Protocol} class contains a list of \emph{Step}s.
}

\subsubsection{Clear the current settings for mixing protocol}
\SRQ{must have}{ % CPR19
    The user can press the \emph{Reset} button to clear the current canvas, which is equivalent to completely colouring it white.
}

\subsection{Client-Server communication}
\SRQ{must have}{ % CPR29a
    After defining a mixer, a mixing protocol and an initial distribution, the user can press the \emph{Mix} button to execute the actual mixing.
}

\SRQ{must have}{ % CPR29b
    The parameters are JSON'ed and sent to the server using a socket. The server un-JSON's the parameters and executes the mixing using the \textsc{Fortran} implementation.
}

\SRQ{must have}{ % CPR29c
    The server un-JSON's the parameters and executes the mixing using the \textsc{Fortran} implementation.
}

\SRQ{must have}{ % CPR29d
    The server JSON's the resulting concentration distribution and the performance result and sends this JSON object back to the client.
}

\SRQ{must have}{ % CPR29e
    The client un-JSON's the concentration distribution and the performance result. This concentration distribution is visualised on the client's screen, using the canvas that was originally used to draw the initial concentration distribution. The performance result is visualised on the screen at the same time using a number.
}

\subsubsection{Mixing runs}
\SRQ{must have}{ % CPR30
    After the mixing has been executed, the user can save the results on their device by pressing the \emph{Save} button. This opens a pop-up using which the user can enter a name for the results.
}

\SRQ{must have}{ % CPR31
    In the menu with previously saved mixing runs, the user can press the \emph{Remove} button to remove this saved run from storage.
}

\section{Non-functional requirements}
\label{sec:nonfuncreq}
\todo{individual sections for non-functional requirements}

\subsection{Performance}
\SRQ{must have}{Waiting time between submitting input and receiving output is no longer than 5 seconds.}
%CNR10

\SRQ{should have}{Waiting time between submitting input and receiving output is no longer than 3 seconds.}
%CNR11

\SRQ{could have}{Waiting time between submitting input and receiving output is no longer than 1 seconds.}
%CNR12

\subsection{Interface}

\todo{Hier zegguh dat wij Fortran servertje gebruikuh.}

\subsection{Operational}
\subsection{Resource}
\subsection{Verification and testing}
\subsection{Portability}
\SRQ{must have}{The application runs on iOS Safari version 6.0 and higher.}
%CNR1

\SRQ{should have}{The application runs on Firefox version 20 and higher.}
%CNR2

\SRQ{should have}{The application runs on Google Chrome version 26 and
higher.}
%CNR3

\SRQ{could have}{The application runs on Internet Explorer version 10 and higher.}
%CNR4

\SRQ{could have}{The application runs on Safari version 6.0 and higher.}
%CNR5

\SRQ{must have}{The application runs on devices running on iOS version 6 and higher.}
%CNR7

\SRQ{should have}{The application runs on devices running on Android version 4.0 and higher.}
%CNR8

\SRQ{could have}{The application runs on devices running on Windows 8.}
%CNR9

\subsection{Maintainability}
\subsection{Reliability}
\subsection{Security}
\subsection{Safety}
\subsection{Documentation}
\subsection{Extensibility}

\SRQ{must have}{The application should be easily extendable with new mixers.}
%CNR13
