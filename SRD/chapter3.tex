\chapter{Specific requirements}
\label{chap:specreq}
This chapter lists the requirements the \projectname application should fulfill once it has been completed. Each requirement has a specific priority, based on the MoSCoW model \cite{moscow}:

\begin{itemize}
    \item \emph{must have}; requirements with this priority are essential for the product, and must be implemented.
    \item \emph{should have}; requirements with this priority are not essential for the product to work. However, they are nearly as important as the \emph{must have}'s and are therefore expected to be implemented.
    \item \emph{could have}; requirements with this priority are a nice addition to the product, and may be implemented, if time and budget allow this.
    \item \emph{won't have}; requirements with this priority will not be implemented in this version of the product, but may be nice to implement in future versions.
\end{itemize}

\section{Functional requirements}
This section lists the functional requirements for the \projectname app.
% TODO: Remove these two lines
\newcounter{count}
\stepcounter{count}

\subsubsection{Define a mixing protocol}
A wall movement is defined by either $T$ or $B$, denoting a movement of the top or bottom wall, respectively. Looking directly at the screen, a $T$ movement indicates that the top wall should move to the right, and a $B$ movement indicates that the bottom wall should move to the left. Each of these wall movements can be combined with a `$-$' sign, to indicate that the direction of movement should be reversed. This means $-T$, for example, indicates that the top wall should move to the \emph{left}. A duration can be any value that can be created by combining the possible values for $D$ (the step size), namely 4, 2, 1, 0.5, 0.25 and 0.1.

\SRQ{must have}{ % CPR20
    The user can enter a step size in a text area. If the user enters a value that cannot be created by combining possible values for $D$, the user should enter a different value.
}

\SRQ{must have}{ % CPR18
    The \emph{Step} class contains a wall movement and a duration.
}

\SRQ{must have}{ % CPR21
    The user can indicate via a spinner how many times the protocol should be applied.
}

\SRQ{should have}{ % CPR22
    The user can click the \emph{Save} button to save the mixing protocol to their device. The desired name for the protocol can then be specified in a pop-up.
}

\SRQ{should have}{ % CPR23
    The user can press the \emph{Remove} button to remove the protocol from the device.
}

\SRQ{should have}{ % CPR24
    After the user has selected the geometry and the mixer as in SRQ\todo{insert correct SRQ here}, they can choose to load a previously saved mixing protocol by choosing the \emph{Load} option. This spawns a fourth menu column with the names of all applicable saved protocols. Clicking one such protocol loads it.
}

\SRQ{should have}{ % CPR25
    Similar to SRQ\todo{insert SRQ for CPR18 here}, the user can create a protocol for a square geometry as a sequence of wall movements and their durations.
}

\SRQ{could have}{ % CPR26
    Using SRQ\todo{insert select a mixer SRQ here}, the user selects a circular geometry and a mixer. The mixing protocol is now defined, as there are no explicit mixer movements for the circular geometry.
}

\SRQ{must have}{ % CPR select a mixer
    After the user has selected a geometry using SRQ\todo{insert Mix of various CPRs SRQ here}, a new menu pops up, containing the list of all suitable mixers for the selected geometry.
}

\SRQ{could have}{ % CPR27
    Using SRQ\todo{insert mixer geometry SRQ here}, the user selects the \emph{Journal bearing} geometry and a mixer. A step size $D$ is selected using SRQ\todo{insert step size SRQ here}. The user can then swipe across the screen to indicate movements. Swiping to the right near the top of the outer circle means the outer circle rotates clockwise for $D$ time units, and swiping to the right near the top of the inner circle means the inner circle rotates clockwise for $D$ time units.
}

\SRQ{could have}{ % CPR28
    After SRQ\todo{insert mixer geometry SRQ here}, the user selects the \emph{Predefined} option in the third level of the \emph{Mixer} menu. This spawns a fourth level of the menu containing all suitable predefined mixing protocols for the selected geometry. Clicking one such protocol selects it, which means it will be used once the mixing is executed.
}

\SRQ{must have}{ % CPR30
    After the mixing has been executed, the user can save the results on their device by pressing the \emph{Save} button. This opens a pop-up using which the user can enter a name for the results.
}

\SRQ{must have}{ % CPR31
    In the menu with previously saved mixing runs, the user can press the \emph{Remove} button to remove this saved run from storage.
}

\subsubsection{Define mixing protocol for a rectangular geometry}
This class allows the user to define a mixing protocol as a sequence of wall movements for a rectangular mixer geometry.

\SRQ{must have}{ % CPR17
    The \emph{Protocol} class contains a list of \emph{Step}s.
}

\subsubsection{Clear the current settings for mixing protocol}
\SRQ{must have}{ % CPR19
    The user can press the \emph{Reset} button to clear the current canvas, which is equivalent to completely colouring it white.
}

\subsection{Client-Server communication}
\SRQ{must have}{ % CPR29
    After \todo{loads of SRQs which still have to be linked and as this linking functionality has not been implemented yet I will wait with choosing all correct SRQs and instead leave this notice here}, the user can press the \emph{Mix} button to execute the actual mixing. 
}

\SRQ{must have}{ % CPR29cont1
    The parameters are JSON'ed and sent to the server using a socket. The server un-JSON's the parameters and executes the mixing using the \textsc{Fortran} implementation.
}

\SRQ{must have}{ % CPR29cont2
    The server un-JSON's the parameters and executes the mixing using the \textsc{Fortran} implementation.
}

\SRQ{must have}{ % CPR29cont3
    The server JSON's the resulting concentration distribution and the performance result and sends this JSON object back to the client.
}

\SRQ{must have}{ % CPR29cont4
    The client un-JSON's the concentration distribution and the performance result. This concentration distribution is visualised on the client's screen, using the canvas that was originally used to draw the initial concentration distribution. The performance result is visualised on the screen at the same time using a number.
}


\section{Non-functional requirements}
\label{sec:nonfuncreq}
\todo{individual sections for non-functional requirements}

\todo{Add numbers to SCR}
\subsection{Performance}
\begin{center}
\begin{tabular}{ >{\bfseries}p{0.84\textwidth} >{\itshape}p{0.16\textwidth}}
SCR\arabic{count} & must have \\
\multicolumn{2}{p{\textwidth}}{Waiting time between submitting input and receiving output is no longer than 5 seconds.} \\
\hline
\stepcounter{count}

SCR\arabic{count} & should have \\
\multicolumn{2}{p{\textwidth}}{Waiting time between submitting input and receiving output is no longer than 3 seconds.} \\
\hline
\stepcounter{count}

SCR\arabic{count} & could have \\
\multicolumn{2}{p{\textwidth}}{Waiting time between submitting input and receiving output is no longer than 1 seconds.} \\
\hline
\stepcounter{count}

\end{tabular}
\end{center}
\subsection{Interface}
\subsection{Operational}
\subsection{Resource}
\subsection{Verification and testing}
\subsection{Portability}
\begin{center}
\begin{tabular}{ >{\bfseries}p{0.84\textwidth} >{\itshape}p{0.16\textwidth}}
SCR\arabic{count} & must have \\
\multicolumn{2}{p{\textwidth}}{The application runs on iOS Safari version 6.0 and higher.} \\
\hline
\stepcounter{count}

SCR\arabic{count} & should have \\
\multicolumn{2}{p{\textwidth}}{The application runs on Firefox version 20 and higher, and Google Chrome version 26 and
higher.} \\
\hline
\stepcounter{count}

SCR\arabic{count} & could have \\
\multicolumn{2}{p{\textwidth}}{The application runs on Internet Explorer version 10 and higher, Opera version 12.1 and
higher and Safari version 6.0 and higher.} \\
\hline
\stepcounter{count}

SCR\arabic{count} & must have \\
\multicolumn{2}{p{\textwidth}}{The application runs on devices running on iOS version 6 and higher.} \\
\hline
\stepcounter{count}

SCR\arabic{count} & should have \\
\multicolumn{2}{p{\textwidth}}{The application runs on devices running on Android version 4.0 and higher.} \\
\hline
\stepcounter{count}

SCR\arabic{count} & could have \\
\multicolumn{2}{p{\textwidth}}{The application runs on devices running on Windows 8.} \\
\hline
\stepcounter{count}

\end{tabular}
\end{center}
\subsection{Maintainability}
\subsection{Reliability}
\subsection{Security}
\subsection{Safety}
\subsection{Documentation}
\subsection{Extensibility}
\begin{center}
\begin{tabular}{ >{\bfseries}p{0.84\textwidth} >{\itshape}p{0.16\textwidth}}
SCR\arabic{count} & must have \\
\multicolumn{2}{p{\textwidth}}{The application should be easily extendable with new mixers.} \\
\hline
\stepcounter{count}

\end{tabular}
\end{center}
