\chapter{Specific requirements}
\label{chap:specreq}
This chapter lists all specific software requirements of the application to be developed, both functional and non-functional requirements. The requirements are categorised according to the interface they belong to, as described in section \ref{sec:moddesc}.

Each requirement has a specific priority, based on the MoSCoW model \cite{moscow}:

\begin{itemize}
    \item \emph{must have}; requirements with this priority are essential for the product, and must be implemented.
    \item \emph{should have}; requirements with this priority are not essential for the product to work. However, they are nearly as important as the \emph{must have}'s and are therefore expected to be implemented.
    \item \emph{could have}; requirements with this priority are a nice addition to the product, and may be implemented, if time and budget allow this.
    \item \emph{won't have}; requirements with this priority will not be implemented in this version of the product, but may be nice to implement in future versions.
\end{itemize}

Only those user requirements from the URD \cite{urd} with a priority higher then \emph{won't have} will be translated to software requirements in this chapter.

Note that as \applicationname\ is developed using GWT, some of the software requirements listed in this chapter are described in terms of specific GWT widgets or panels\footnote{The website for the GWT widgetlist can be found at \url{https://developers.google.com/web-toolkit/doc/latest/RefWidgetGallery}.}.

% Functional requirements --------------------------------------------------------------------------------------------------------
\section{Functional requirements}
\label{sec:funcreq}
This section lists the functional requirements for the \applicationname.

\subsection[HTTP]{HTTP Server}
\SRQ[HTTP-1]{must have}{The HTTP Server must be able to serve files to the Client Browser.}

\subsection[AS]{Application Service} % Oude Mixer Provider
\SRQ[AS-1]{must have}{The Application Service must be able to retrieve all the available mixers from the Application Persistence.}
\SRQ[AS-2]{must have}{The Application Service must be able to send all the available mixers to the Client Browser.}
\SRQ[AS-3]{must have}{The Application Service must allow for adding new mixers to the Application Persistence.} %CNR13
\SRQ[AS-4]{must have}{The Application Service must allow for editing existing mixers in the Application Persistence.}
\SRQ[AS-5]{must have}{The Application Service must allow for removing existing mixers to the Application Persistence.}
\SRQ[AS-6]{must have}{The Application Service must be able to provide all available geometries to the Client Browser.}

\subsection[APC]{Application Persistence Communication}
\SRQ[APC-1]{must have}{The Application Persistence Communication must be able to access all the data from the Application Persistence.}
\SRQ[APC-2]{must have}{The Application Persistence Communication must be \todo{fast enough} to serve the data from the Application Persistence to the Application Service.}

\subsection[AP]{Application Persistence} % Voorheen MixerPersistence
\SRQ[AP-1]{must have}{The Application Persistence must deliver all saved mixers to the Application Service.}
\SRQ[AP-2]{must have}{The Application Persistence must save new mixers handed by the Application Service.}
\SRQ[AP-3]{must have}{The Application Persistence must save changes to existing mixers handed by the Application Service.}
\SRQ[AP-4]{must have}{The Application Persistence must remove existing mixers handed by the Application Service.}

\subsection[ASC]{Application Service Communication} % Voorheen MixerProviderCommunication
\SRQ[ASC-1]{must have}{The Application Service Communication must be able to access all the functionality from the Application Service.}
\SRQ[ASC-2]{must have}{The Application Service Communication must be \todo{fast enough} to provide the functionality from the Application Service to the Client Browser.}

\subsection[CB]{Client Browser}
\subsubsection{Set options for a mixing protocol}
A duration can be any value that can be created by combining the possible values for $D$ (the step size), namely 4, 2, 1, 0.5 and 0.25. Such a value $D$ can be entered in a text box near the drawing canvas. If an invalid value is entered -- i.e. a value that cannot be created by combining the mentioned values -- this value is rounded up or down to the nearest valid value, whichever is closest. For example, 4.2 is rounded up to 4.25, while 4.1 is rounded down to 4.

To indicate how many times the protocol created using the above items is executed, the user can enter a number in a spinner. We use a spinner because the number it contains can easily be incremented or decremented by using the arrow keys, and a value that is not near the original value can easily be entered using its text box.

\SRQ[stepsize]{must have}{ % CPR20
    The user can enter a valid step size in a text box.
}

\SRQ[CB-1]{must have}{ % CPR21
    The user can indicate via a spinner how many times the protocol should be applied. This value must be an integer number, and must be at least 1.
}

\subsubsection{Defining an initial concentration distribution}
\SRQ[CB-2]{must have}{The user can select a colour (black or white) to paint with for the initial concentration distribution, via a toggle button.} % CPR6
\SRQ[CB-3]{must have}{The user can define an initial concentration distribution with the selected colour on the canvas, by drawing with their finger.} % CPR6
\SRQ[CB-4]{must have}{After selecting the circle-shaped drawing tool in \srqref{circleshaped}, the user can draw with this tool on the canvas.} % CPR7, los omdat de square een lagere prioriteit heeft in het URD
\SRQ[CB-5]{must have}{The user can reset the current concentration distribution to a completely white concentration distribution, by clicking on the ``Reset" button.} % CPR8
\SRQ[CB-6]{should have}{After selecting the square-shaped drawing tool in \srqref{squareshaped}, the user can draw this tool on the canvas.} % CPR9

\subsubsection{Select a geometry, mixer and initial concentration distribution}
%The user can select a rectangular mixer geometry. To this end, a cell browser is available that lists all the available geometries in the first column. This first column will contain several geometries, including \texttt{rectangle}, \texttt{square}, \texttt{circle} or \texttt{Journal Bearing}. After clicking on a certain geometry, the available mixers for this geometry are displayed in the second column of the cell browser. The user can now select a mixer of choice and then, the third column in the cell browser lists all the available starting concentration distributions for the selected mixer. There are three types of starting concentration distributions that the user can choose from. The first type is a \texttt{blank} concentration distribution, which simply means that the user will be presented with a clean canvas when this option is selected. The second type is a \texttt{load} option, so the user can load previously saved concentration distributions that are stored on their device. The third type of 
% concentration distribution is a \texttt{predefined} distribution, meaning that the user can choose from a few predefined distributions that are already present in the application. In case of the second and third option, there will be a third column in the cell browser that lists all the available concentration distributions that can be loaded. When the user clicks on an item in the third column, the required distribution is immediately loaded on an appropriate canvas. For the first option, a blank canvas is displayed on the screen when the user clicks on \texttt{blank}. \\
%This user requirement can then be described with the following functional requirements:

\SRQ[CB-7]{must have}{The mixing interface contains a cell browser with four columns, to select an initial mixer, geometry and initial distribution.} % CPR1
\SRQ[selgeomrec]{must have}{The user can select \texttt{rectangle} as a geometry, via the first column in the cell browser.} % CPR1
\SRQ[selmixer]{must have}{After having selected a geometry, the user can select a mixer that fits the selected geometry. The mixer can be selected via a second column in the cell browser.} % CPR2
\SRQ[selgeomsq]{should have}{The user can select \texttt{square} as a geometry, via the first column in the cell browser.}  % CPR3
\SRQ[selgeomcir]{could have}{The user can select \texttt{circle} as a geometry, via the first column in the cell browser.}  % CPR4
\SRQ[selgeomjb]{could have}{The user can select \texttt{Journal Bearing} as a geometry, via the first column in the cell browser.}  % CPR5
\SRQ[CB-8]{must have}{The user can select a blank canvas after choosing a geometry and mixer, by choosing the \texttt{blank} option in the third column in the cell browser.} % CPR6
\SRQ[prevsaved]{should have}{The user can view a list of all previously saved concentration distributions, by choosing the \texttt{load} option in the third column in the cell browser.} % CPR13
\SRQ[CB-9]{should have}{After selecting the \texttt{load} option from \srqref{prevsaved}, the user can select a previously saved distribution by clicking on one distribution in the fourth column in the cell browser.} % CPR13
\SRQ[predef]{should have}{The user can view a list of all predefined concentration distributions, by choosing the \texttt{predefined} option in the third column in the cell browser.} % CPR14
\SRQ[CB-10]{should have}{After selecting the \texttt{predefined} option from \srqref{predef}, the user can select a predefined initial concentration distribution by clicking on one distribution in the fourth column in the cell browser.} % CPR14

\subsubsection{Change the drawing tool}
The user can change the drawing tool that is used to paint on the canvas. The current shape and size of the drawing tool is displayed on the mixing interface, as a button with an image. When the user clicks on this button, a pop-up menu appears with two vertical panels. The left panel contains a list of all the different shapes for the drawing tool, such as circle and square. The right panel contains a slider to adjust the size of the drawing tool. The user can change the settings for the drawing tool and close the pop-up menu by clicking on the button again.

\SRQ[shapedraw]{must have}{The mixing interface contains a button that displays the current shape of the drawing tool.} % CPR7
\SRQ[CB-12]{should have}{The button from \srqref{shapedraw} also displays the current size of the drawing tool.} % CPR10
\SRQ[circleshaped]{must have}{After clicking the button from \srqref{shapedraw}, the user can select a circle-shaped drawing tool via a pop-up panel.} % CPR7
\SRQ[squareshaped]{should have}{After clicking the button from \srqref{shapedraw}, the user can select a square-shaped drawing tool via a pop-up panel.} % CPR9
\SRQ[CB-13]{should have}{The user can adjust the size of the drawing tool, using a slider in the popup panel from \srqref{circleshaped} and \srqref{squareshaped}}. % CPR10

\subsubsection{Define mixing protocol for specific geometries}
A mixing protocol consists of movements of the mixer (if applicable), and how long these movements are executed. Different movement types are possible for each geometry. Each movement is performed for $D$ time units, which can be set using \srqref{stepsize}.

For the rectangular or square geometries, a wall movement is defined by either $T$ or $B$, denoting a movement of the top or bottom wall, respectively. Looking directly at the screen, a $T$ movement indicates that the top wall should move to the right, and a $B$ movement indicates that the bottom wall should move to the left. This way, the walls normally move in a clockwise direction. Each of these wall movements can be combined with a `$-$' sign, to indicate that the direction of movement should be counterclockwise. This means $-T$, for example, indicates that the top wall should move to the \emph{left}.

The circular geometry does not support movements, and as such a mixing protocol is immediately defined after selecting a mixer.

For the \emph{Journal bearing} geometry, movements are defined by rotating the outer or inner circle. Swiping to the right near the top of the outer circle means the outer circle rotates clockwise for $D$ time units, and swiping to the right near the top of the inner circle means the inner circle rotates clockwise for $D$ time units. Here, too, a `-' sign implies the rotation is executed in the opposite direction.

\SRQ[CB-14]{must have}{ % CPR17a, CPR25a, CPR26a, CPR27a
    The \emph{Step} class contains a movement and a duration for this movement.
}

\SRQ[CB-15]{must have}{ % CPR17b, CPR25b, CPR26b, CPR27b
    The \emph{Protocol} class contains a list of \emph{Step}s.
}

\SRQ[startMixing]{must have}{ % CPR29a
    After defining a mixer, a mixing protocol and an initial distribution, the user can press the \texttt{Mix} button to execute the actual mixing. This button is greyed out (it cannot be pressed) if no mixing protocol has been defined.
}

\SRQ[CB-16]{must have}{ % CPR19
    The user can reset the protocol, which erases all \emph{Step}s, effectively starting over with defining the protocol.
}

\subsubsection{Visualising the results}
When the client receives the JSON object containing the concentration distribution and the performance result from the server, the client parses these items. The concentration distribution is visualised on the client's screen, using the canvas that was originally used to draw the initial concentration distribution. The performance result is visualised in a graph, which contains all previous performance results for this particular run.

\SRQ[visResults]{must have}{ % CPR29e
    Concentration distributions can be drawn on the client's canvas.
}

\SRQ[CB-16-2]{must have}{ % CPR29f
    Performance results can be plotted in a graph.
}

\subsubsection{Execute mixing runs}
The user can choose to execute a single step of a mixing protocol, by defining a single \emph{Step}. This is done by choosing one wall movement and its duration. This invokes \srqref{startMixing}, \srqref{sendParams}, \srqref{readParams}, \srqref{execMixing}, \srqref{returnParams} and \srqref{visResults}.

\SRQ[CB-17]{must have}{ % CPR18
    After selecting a geometry, a mixer, an initial distribution and a single \emph{Step}, the user can press the \texttt{Execute step} button to execute a single step. This button is greyed out (cannot be pressed) if no \emph{Step} has been defined.
}

\subsection[CP]{ClientPersistence}
\SRQ[CP-0]{should have}{The user can save an initial concentration distribution locally on their device, using a button and popup panel.} % CPR11

\subsubsection{Managing mixing protocols}
To manage mixing protocols, the user can save and load the mixing protocols they have created. It is also possible to load a predefined mixing protocol.

\SRQ[CP-1]{should have}{The user can save an initial concentration distribution locally on their device, using a button and popup panel.} % CPR11

\SRQ[CP-2]{should have}{ % CPR22
    The user can click the \texttt{Save} button to save the mixing protocol to their device. The desired name for the protocol can then be specified in the pop-up that appears.
}

\SRQ[CP-3]{should have}{ % CPR23
    The user can press the \texttt{Remove} button to remove the saved protocol from the device.
}

\SRQ[CP-4]{should have}{ % CPR24
    After the user has selected the geometry and the mixer using \srqref{selgeomrec}, \srqref{selgeomsq}, \srqref{selgeomcir} or \srqref{selgeomjb} and \srqref{selmixer}, they can choose to load a previously saved mixing protocol. This opens a menu with the names of all applicable saved protocols. Pressing the name of one such protocol loads it.
}

\SRQ[CP-5]{could have}{ % CPR28
    After \srqref{selgeomrec}, \srqref{selgeomsq}, \srqref{selgeomcir} or \srqref{selgeomjb} and \srqref{selmixer}, the user can load a predefined mixing protocol. This opens a menu containing all suitable predefined mixing protocols for the selected geometry. Pressing the name of one such protocol selects it, which means it will be used once the mixing is executed.
}

\subsubsection{Mixing runs}
\SRQ[CP-6]{must have}{ % CPR30
    After the mixing has been executed, the user can save the results on their device by pressing the \texttt{Save} button. This opens a pop-up using which the user can enter a name for the results.
}

\SRQ[CP-7]{must have}{ % CPR31
    In the menu with previously saved mixing runs, the user can press the \texttt{Remove} button to remove this saved run from storage.
}

\subsubsection{Save an initial concentration distribution}
%The user can save an initial concentration distribution locally on their device. When the user clicks the \texttt{Save Distribution} button, a pop-up menu appears. The user can now enter a name for the concentration distribution and click the \texttt{Save} button to save the distribution. When the save was successful, a \emph{Save was successful} message is displayed and the user can press \texttt{OK} to go back to the mixing interface. If there is not enough memory to save the image, a \emph{Insufficient memory message} is displayed; the user can then press the \texttt{OK} button to go to the mixing interface. \\
%If the entered name is already in use, a new pop-up back menu is displayed, giving the user the choice to either \texttt{Overwrite} or \texttt{Cancel} the current save process. If the user presses the \texttt{Overwrite} button and the save was successful, a \emph{Save was successful} message is displayed and the user can press \texttt{OK} to go back to the mixing interface. The user can abort the saving process in either dialogues, by clicking the \texttt{Cancel} button. \\

\SRQ[savedistr]{should have}{The user can save an initial concentration distribution locally on their device, using a \texttt{Save Distribution} button.} % CPR11
\SRQ[savename]{should have}{After clicking the save button from \srqref{savedistr}, the user can specify a name for the concentration distribution.} % CPR11
\SRQ[savebutton]{should have}{After specifying a name in \srqref{savename}, the user can save the distribution with this name using the \texttt{Save} button.} % CPR11
\SRQ[CP-8]{should have}{When the save from \srqref{savebutton} was successful,  the user can click the \texttt{OK} button to go back to the mixing interface.} % CPR11
\SRQ[CP-9]{should have}{If the specified name from \srqref{savename} is already taken, the user can click the \texttt{Overwrite} button to overwrite the previous distribution with the same name.} % CPR11
\SRQ[CP-10]{should have}{The user can cancel the saving process, by clicking the \texttt{Cancel} button; the user will now return to the mixing interface.} % CPR11

\subsubsection{Load previously saved concentration distribution}
\SRQ[loaddistbutton]{should have}{The mixing interface has a \texttt{Load} button that allows the user to load previously saved concentration distributions.} % CPR13
\SRQ[CP-11]{should have}{After clicking the \texttt{Load} button from \srqref{loaddistbutton}, the \texttt{Load distribution} interface appears.} % CPR13
\SRQ[loaddist]{should have}{The \texttt{Load distribution} interface contains a cell browser with two rows.} % CPR13
\SRQ[loadload]{should have}{The first row of the cell browser from \srqref{loaddist} contains a \texttt{Load} option, to load previously saved distributions.} % CPR13
\SRQ[CP-12]{should have}{The second row of the cell browser from \srqref{loadload} displays all previously saved concentration distributions for the selected option of \srqref{loadload}.} % CPR13
\SRQ[CP-13]{should have}{The user can load a specific concentration distribution by tapping on it; the mixing interface is now shown, with the selected distribution displayed on the canvas.} % CPR13+14
\SRQ[CP-14]{should have}{The user can press the \texttt{OK} button in the message from \srqref{nosaveddistr} to return to the \texttt{Load distribution} interface.} % CPR13
\SRQ[loadpred]{could have}{The first row of the cell browser from \srqref{loaddist} contains a \texttt{Predefined} option, to load predefined concentration distributions.} % CPR14
\SRQ[CP-15]{could have}{The second row of the cell browser from \srqref{loaddist} displays all predefined concentration distributions for the selected option of \srqref{loadpred}.} % CPR14

\todo{Dit is niet in overeenstemming met use case B.20, want ik heb geen \texttt{View Saved Files} button en ik sta hier toe dat meerdere dingen tegelijkertijd verwijderd worden.}
\subsubsection{Remove previously saved concentration distributions}
%The user can remove a previously saved initial concentration distribution. To this end, a \texttt{Remove saved distributions} button is available. When the user clicks this button, a new screen opens with a cell list that shows all the available saved concentration distributions, along with a check box next to each item. The \texttt{Remove saved distributions} button will remain visible in this new screen. The user can then select one or more items to be removed using the check boxes. When the remove button is pressed, an \emph{Are you sure?} message is displayed. When the user presses \texttt{Yes}, all the selected distributions will be removed and the user returns to the history interface. If \texttt{No} is pressed, removal process is aborted and the history interface is shown again. \\
%If no concentration distributions have been saved yet, a \emph{No saved distributions} message is displayed and the user can press \texttt{OK} to continue and display the history interface again. If the user has not given access rights to the application, a \emph{Insufficient access rights} error message is shown. The user can press \texttt{OK} to return to the history interface. \\
%This user requirement can then be described with the following functional requirements:

\SRQ[CP-16]{should have}{The mixing interface has a \texttt{Remove saved distributions} button to remove previously saved distributions.} % CPR12
\SRQ[listsaved]{should have}{After clicking the \texttt{Remove saved distributions} button, the history interface is shown, with a cell list that provides all the previously saved concentration distributions.} % CPR12
\SRQ[selectremove]{should have}{Each item from in the cell list of \srqref{listsaved} contains a check box that can be selected or deselected, to indicate which distributions should be removed.} % CPR12
\SRQ[suremessage]{should have}{After selecting at least one item from \srqref{selectremove}, the user can press the \texttt{Remove saved distributions} button and a \emph{Are you sure?} message is shown.} % CPR12
\SRQ[yesbutton]{should have}{The user can remove all selected items from \srqref{selectremove} by clicking the \texttt{Yes} button in the dialogue from \srqref{suremessage}.} % CPR12
\SRQ[nobutton]{should have}{The user can cancel the removal process by clicking the \texttt{No} button in the dialogue from \srqref{suremessage}. } % CPR12
\SRQ[nosaveddistr]{should have}{If no distributions have been saved yet, a \emph{No saved distributions} message is shown.} % CPR12+13
\SRQ[CP-17]{should have}{The user can press the \texttt{OK} button in the message from \srqref{nosaveddistr} to return to the history interface.} % CPR12
\SRQ[insuffrights]{should have}{If the user has not given access rights to the application, a \emph{Insufficient access rights} error message is shown.} % CPR12
\SRQ[CP-18]{should have}{The user can press the \texttt{OK} button in the message from \srqref{insuffrights} to return to the history interface.} % CPR12
\SRQ[CP-19]{should have}{After pressing the button from requirements \srqref{yesbutton} or \srqref{nobutton}, the user returns to the history interface.} % CPR12

\subsubsection{Saving and exporting results}
The results window can be opened by choosing \texttt{Results} from the main interface. This window contains a \texttt{save} button, an \texttt{export} button and a \texttt{load results} button. If the \texttt{export} button is pressed, a pop-up with several available export options is shown: \texttt{export performance}, \texttt{export picture} and \texttt{export animation}. \texttt{export animation} is only available if the \texttt{animate mixing} checkbox was checked in the mixing interface during the run. The user can then name the file. If during saving the specified name for saving/exporting is already in use, the application returns a \texttt{name already in use} message. Additionally, a \texttt{load performance} button is available in the results window. The user can use this button to open a popup. In this popup they can select the files they want to load. The button \texttt{load selected} can now be pressed and now loads all the selected files. If no files are selected, the button named \texttt{cancel} simply cancels without loading. An alternative way to cancel loading is to click somewhere not on the pop-up.

% \SRQ[CP-20]{should have}{The \texttt{export picture} option can be used to open the export menu for an image of the mixing result.}
%CPR33
\SRQ[CP-21]{should have}{The \texttt{export performance} option can be used to open the export menu for an image of the current performance graph.}
%CPR35&37
\SRQ[CP-22]{could have}{The \texttt{export animation} button can be used to open the export menu for an animation of the mixing procedure somewhere on the device.}
%CPR39
\SRQ[exportname]{should have}{In the export menu, the user can specify a name for the exported file. }
%CPR33&35&37&39
\SRQ[exportsave]{should have}{If the user clicks on the save button in the \texttt{export} menu, the file is stored at the location where the browser stores files with the name specified in \srqref{exportname}.}
%CPR33&35&37&39
\SRQ[CP-23]{should have}{The user should be able to cancel loading files by clicking somewhere not on the loading popup.}
%CPR33&35&37&39
\SRQ[nameinuseexport]{should have}{If there are no access rights to store the exported/saved data at the selected location when \srqref{exportsave} is executed, the application returns a \texttt{no access rights} message.}
%CPR33&35&37&39, Use case alternative S.1
\SRQ[nomemoryexport]{should have}{If there is no memory space to store exported/saved data when \srqref{exportsave} is executed, the application returns an \texttt{out of memory} message.}
\SRQ[actualNameInUse]{must have}{If a file exists with the same name as the one entered in \srqref{exportname}, the application returns a \texttt{File exists} message.}
%CPR33&35&37&39, Use case alternative S.2
\SRQ[CP-24]{should have}{The popups opened by \srqref{actualNameInUse} and \srqref{nomemoryexport} can be closed with an \texttt{ok} button}.
%CPR33&35&37&39
\SRQ[CP-25]{should have}{The user can overwrite the old file in \srqref{actualNameInUse} with an \texttt{overwrite} button if a saved file already exists.}
%CPR33&35&37&39, Use case alternative S.3.1
\SRQ[CP-26]{should have}{The user can cancel overwriting the old file in \srqref{actualNameInUse} with a \texttt{cancel} button if a saved file already exists.}
%CPR33&35&37&39, Use case alternative S.3.1

\subsubsection{Loading results}
In the results window a \texttt{load results} button is available. This button opens a loading window in which a \texttt{results} file is opened. Multiple results can be selected.

\SRQ[CP-27]{should have}{The \texttt{load results} button can be used to open the load menu for a \texttt{results} file.}
%CPR30&36
\SRQ[selectload]{should have}{In the load menu one or more \texttt{results} files can be selected to be loaded.}
%CPR30&36
\SRQ[loadresults]{should have}{In the load menu, the \texttt{load} button can be used to load the files selected in \srqref{selectload}.}
%CPR30&36
\SRQ[displayload]{should have}{When \srqref{loadresults} is executed, the end mixture and the performance are loaded from files selected in \srqref{selectload}.}
%CPR30&36
\SRQ[CP-28]{should have}{If multiple files are selected in the loading menu, they are all depicted in a single performance graph.}
%CPR36
\SRQ[CP-29]{should have}{If in \srqref{displayload} multiple files are selected, the user is able to browse through the result mixtures.}
%CPR36, extention for consistency.
\SRQ[CP-30]{should have}{In the load menu, the \texttt{cancel} button can be used to exit the loading menu.}
%CPR30&36

\subsubsection{Language selection}
In the menu bar a flag of the currently selected language is visible. When clicked, a popup appears containing several language options, each symbolised by the flag of their country. These flags can be clicked to change the language to that specific language. Standard language is English. The preferred language should be stored on the client device.
\SRQ[selLang]{could have}{
  The user can choose a different language for the application.
}

\subsection[SSC]{Simulator Service Communication}
To execute a mixing protocol, the following SRQs are executed in this order: \srqref{startMixing}, \srqref{sendParams}, \srqref{readParams}, \srqref{execMixing}, \srqref{returnParams} and \srqref{visResults}.

\SRQ[sendParams]{must have}{ % CPR29b
    The parameters (initial concentration distribution, geometry, mixing protocol, mixer and number of protocol applications) are encapsulated into a JSON object which is sent to the server using a socket.
}

\SRQ[readParams]{must have}{ % CPR29c
    The server parses the parameters.
}

\SRQ[returnParams]{must have}{ % CPR29d
    The server encapsulates the resulting concentration distribution and the performance result into a JSON object and sends this back to the client using the socket.
}

\SRQ[SSC-1]{must have}{The Simulator Service Communication must be able to access all the functionality from the Application Persistence.}
\SRQ[SSC-2]{must have}{The Simulator Service Communication must be \todo{fast enough} to serve the data from the Application Persistence to the Client Browser.}

\subsubsection{Requirements regarding the presentation of results}
The user is able to visualise his mixing protocol, as well as able to save and export these results. To this end a results window is available. In the results window there is a \texttt{show statistics} button. When pressed, the performance of the mixing run is shown in a graph. In this new menu the \texttt{show statistics} button changes to \texttt{show result}. When this button is pressed, the image switches back to the default results window. The results window is reached by pressing the \texttt{start mixing} button in the mixing interface. The mixing interface is the window where the user can define his mixing protocol. In the mixing interface there is a \texttt{animate mixing} checkbox. If this checkbox was checked when the \texttt{start mixing} button is pressed, an animation of the mixing procedure is shown in the results window.

\SRQ[SSC-3]{must have}{In the results window the result of the user's mixing protocol is shown. This interface can be reached by pressing the \texttt{start mixing} button.}
%CPR32
\SRQ[SSC-4]{must have}{The user can use the \texttt{show performance} button to view the performance graph of the mixing run.}
%CPR34
\SRQ[SSC-5]{must have}{The user can use the \texttt{show result} button to switch back to the standard results window and view the result distribution of the mixing run.}
%CPR34, extention for consistency.

\SRQ[SSC-6]{could have}{An animation of the mixing run is shown in the results window if the \texttt{animate mixing} checkbox was checked.}
%CPR38

\subsection[SS]{Simulator Service}
\SRQ[execMixing]{must have}{ %CRP29c2
     The server executes the mixing run using the parameters received via the socket. These parameters are passed as input to the \textsc{Fortran} implementation.
}

\subsubsection{Information exchange}
The Simulator Service must be able to send and receive matrix files from the Client Browser and to the Fortran Module. These matrices specify the characteristics of the mixers. 

\SRQ[SS-1]{must have}{The Simulator Service must be able to receive matrix files from the Client Browser.}
\SRQ[SS-2]{must have}{The Simulator Service must be able to receive protocol information from the Client Browser.}
\SRQ[SS-3]{must have}{The Simulator Service must be able to pass matrix files to the Fortran Module.}
\SRQ[SS-4]{must have}{The Simulator Service must be able to pass protocol information to the Fortran Module.}
\SRQ[SS-5]{must have}{The Simulator Service must make sure that that the listed mixers in the Application Persistence are known to the Fortran Module, for each geometry.}

\subsection[FM]{Fortran Module}
\SRQ[FM-1]{must have}{The Fortran Module must accept input (matrix, protocol information) from the Simulator Service.}
\SRQ[FM-2]{must have}{The Fortran Module must provide output (vectors) to the Simulator Service.}

% Non-functional requirements ----------------------------------------------------------------------------------------------------
\section{Non-functional requirements}
\label{sec:nonfuncreq}

\subsection{Performance}
\SRQ[NONF-1]{must have}{Waiting time between submitting input and receiving output is around 5 seconds on average.}
%CNR10

\SRQ[NONF-2]{should have}{Waiting time between submitting input and receiving output is around 3 seconds on average.}
%CNR11

\SRQ[NONF-3]{could have}{Waiting time between submitting input and receiving output is around 1 second on average.}
%CNR12

\subsection{Interface}
\SRQ[NONF-4]{must have}{An English interface is available.}
%CPR40

\SRQ[NONF-5]{should have}{A Dutch interface is available.}
%CPR41

\subsection{Operational}
\subsection{Resource}
\subsection{Verification and testing}
\subsection{Portability}
\SRQ[NONF-6]{must have}{The application runs on iOS Safari version 6.0 and higher.}
%CNR1

\SRQ[NONF-7]{should have}{The application runs on Firefox version 20 and higher.}
%CNR2

\SRQ[NONF-8]{should have}{The application runs on Google Chrome version 26 and
higher.}
%CNR3

\SRQ[NONF-9]{could have}{The application runs on Internet Explorer version 10 and higher.}
%CNR4

\SRQ[NONF-10]{could have}{The application runs on Safari version 6.0 and higher.}
%CNR5

\SRQ[NONF-11]{must have}{The application runs on devices running on iOS version 6 and higher.}
%CNR7

\SRQ[NONF-12]{should have}{The application runs on devices running on Android version 4.0 and higher.}
%CNR8

\SRQ[NONF-13]{could have}{The application runs on devices running on Windows 8.}
%CNR9

\subsection{Maintainability}
\subsection{Reliability}
\subsection{Security}
\subsection{Safety}
\subsection{Documentation}
\subsection{Extensibility}

%%%%%%
%%%%%%%%%
%%%%%%%%%
%%%%%%%%%
%%%%%%%%%
%%%%%%%%%
%%%%%%%%%
%%%%%%%%%
%%%%%%%%%
%%%%%%%%%
%%%%%%%%% Feedback nog te doen
% 3.1.8 Other 1 button for 2 functionalities wat is de andere functionaliteit
% Language selection heeft geen SRQ
% 3.1.7 Q Language select
% SRQ89 toegevoegd; SRQs hierna zijn dus 1 opgeschoven
% SRQ84 is verwijderd
% 3.1.7 Other It's not clear ... Wss is de incorrect gedachte verwoording wel correct want anders is het gelijk aan srq98
%
%
%