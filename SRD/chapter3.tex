\chapter{Specific requirements}
\label{chap:specreq}
\todo{chapter description}
The structure of the software requirements in this section are conform the user requirements described in chapter 3 of the URD \cite{urd}. \\
Note that as \applicationname\ is developed using GWT, some of the software requirements listed in this chapter are described in terms of specific GWT widgets or panels\footnote{The website for the GWT widgetlist can be found at \url{https://developers.google.com/web-toolkit/doc/latest/RefWidgetGallery}.}.

\todo{Nog even vermelden dat we de won't haves uit het URD niet beschrijven in dit SRD.}
\todo{Requirements groeperen naar: MixerProvider, MiserPersistance, Client, ClientPersistence, SimulatorServer}

\section{Functional requirements}
\label{sec:funcreq}

The user can select a rectangular mixer geometry. To this end, a cell browser is available that lists all the available geometries in the first column. This first column will contain several geometries, including ``rectangle", `` square", ``circle" or ``\emph{Journal Bearing}". After clicking on a certain geometry, the available mixers for this geometry are displayed in the second column of the cell browser. The user can now select a mixer of choice and then, the third column in the cell browser lists all the available starting concentration distributions for the selected mixer. There are three types of starting concentration distributions that the user can choose from. The first type is a ``blank" concentration distribution, which simply means that the user will be presented with a clean canvas when this option is selected. The second type is a ``load" option, so the user can load previously saved concentration distributions that are stored on their device. The third type of concentration distribution is a ``predefined" distribution, meaning that the user can choose from a few predefined distributions that are already present in the application. In case of the second and third option, there will be a third column in the cell browser that lists all the available concentration distributions that can be loaded. When the user clicks on an item in the third column, the required distribution is immediately loaded on an appropriate canvas. For the first option, a blank canvas is displayed on the screen. \\
This user requirement can then be described with the following software requirements:

\SRQ{must have}{The user can select ``rectangle" as a geometry, via the first column in a cell browser.} % CPR1
\SRQ{must have}{After having selected a geometry, the user can select a mixer that fits the selected geometry. The mixer can be selected via a second column in a cell browser.} % CPR2
\SRQ{should have}{The user can select ``square" as a geometry, via the first column in a cell browser.}  % CPR3
\SRQ{could have}{The user can select ``circle" as a geometry, via the first column in a cell browser.}  % CPR4
\SRQ{could have}{The user can select ``\emph{Journal Bearing}" as a geometry, via the first column in a cell browser.}  % CPR5

Requirements regarding the initial concentration distribution input: \\

\todo{VRAAG: moeten we er een losse requirement van maken dat de user op een canvas tekent? Ik vind het uit onderstaande
 requirements nu niet geheel duidelijk dat het om exact hetzelfde canvas gaat iedere keer.}
]todo{VRAAG: moeten we naar GWT componenten verwijzen met hoofdletters of niet?}

\SRQ{must have}{The user can select a colour (black or white) to paint with for the initial concentration distribution, via a Toggle Button.} % CPR6
\SRQ{must have}{The user can define an initial concentration distribution with the selected colour on the canvas, by drawing with their finger.} % CPR6
\SRQ{must have}{The user can select a circle-shaped drawing tool via a pop-up panel.} % CPR7
\label{srq:circleshaped}
\SRQ{must have}{The user can draw with a circle-shaped drawing tool on the canvas.} % CPR7, los omdat de square een lagere prioriteit heeft in het URD
\SRQ{must have}{The user can reset the current concentration distribution to a completely white concentration distribution, by clicking on the ``Reset" button.} % CPR8
\SRQ{should have}{The user can select a square-shaped drawing tool via a pop-up panel.} % CPR9
\label{srq:squareshaped}
\SRQ{should have}{The user can draw with a square-shaped drawing tool on the canvas.} % CPR9
\SRQ{should have}{The user can adjust the size of the drawing tool, using a slider in the popup panel from \ref{srq:circleshaped} and \ref{srq:squareshaped}}. % CPR10
\SRQ{should have}{The user can save an initial concentration distribution locally on their device, using a button and popup panel.} \\ % CPR11

The user can remove a previously saved initial concentration distribution. To this end, a ``Remove saved distributions" button is available. When the user clicks this button, a new screen opens with a cell list that shows all the available saved concentration distributions, along with a check box next to each item. The ``Remove saved distributions" button will remain visible in this new screen. The user can then select one or more items to be removed using the check boxes. When the remove button is pressed, all the selected items from the cell list will be removed and the user returns to the previous screen. \todo{Moet ik hier al iets zeggen over de verschillende schermen?} \\
This user requirement can then be described with the following software requirements:

\SRQ{should have}{There is a ``Remove saved distributions" button to remove previously saved distributions.} % CPR12
\SRQ{should have}{After clicking the ``Remove saved distributions" button, a cell list is shown that provides all the previously saved concentration distributions.} % CPR12
\SRQ{should have}{Each item from in the cell list of PREVIOUS contains a check box that can be selected or deselected, to indicate which distributions should be removed.} % CPR12
\SRQ{should have}{After selecting at least one item from PREVIOUS, the user can press the ``Remove saved distributions" button and the selected items are removed from the device.} % CPR12

The user can select an initial concentration distribution from a list of previously saved distributions.

\SRQ{should have}{} % CPR13

Requirements regarding the presentation and exportation of results: \\
\SRQ{should have}{The user can use a 'load' button to open a popup. In this popup he can select the files he wants to load. The 'load' button changes to 'load selected' and now loads all the selected files. The loaded data is then depicted in one graph.}
%CPR36

\SRQ{should have}{The user can use a 'export graph' button`to export the current result graph as a picture. If this button is used, the graph is exported as an SVG using the browser's native exporting support.}
%CPR37

\SRQ{could have}{After the  user has defined a mixing run, he can press the button 'visualise mixing'. After this button is pressed, the application processes the mixing run, together with the mixer data and responds by showing an animation of this mixing simulation to the user.}
%CPR38

\SRQ{could have}{After the user has visualised his mixing run (as described above), he can press the 'export animation' button. When this button is pressed, the animation is exported as an AVG using the browser's native exporting support.}
%CPR39

\section{Non-functional requirements}
\label{sec:nonfuncreq}
\todo{individual sections for non-functional requirements}

\subsection{Performance}
\begin{center}
\begin{tabular}{ >{\bfseries}p{0.84\textwidth} >{\itshape}p{0.16\textwidth}}
SCR & must have \\
\multicolumn{2}{p{\textwidth}}{Waiting time between submitting input and receiving output is no longer than 5 seconds.} \\
\hline

SCR & should have \\
\multicolumn{2}{p{\textwidth}}{Waiting time between submitting input and receiving output is no longer than 3 seconds.} \\
\hline

SCR & could have \\
\multicolumn{2}{p{\textwidth}}{Waiting time between submitting input and receiving output is no longer than 1 seconds.} \\
\hline

\end{tabular}
\end{center}
\subsection{Interface}

\SRQ{could have}{In the menu bar a flag of the currently selected language is visible. When clicked, a popup appears containing several language options, each symbolised by the flag of their country. These flags can be clicked to change the language to that specific language.} \\
%CPR40&41

\todo{Hier zegguh dat wij Fortran servertje gebruikuh.}

\subsection{Operational}
\subsection{Resource}
\subsection{Verification and testing}
\subsection{Portability}
\begin{center}
\begin{tabular}{ >{\bfseries}p{0.84\textwidth} >{\itshape}p{0.16\textwidth}}
SCR & must have \\
\multicolumn{2}{p{\textwidth}}{The application runs on iOS Safari version 6.0 and higher.} \\
\hline

SCR & should have \\
\multicolumn{2}{p{\textwidth}}{The application runs on Firefox version 20 and higher, and Google Chrome version 26 and
higher.} \\
\hline

SCR & could have \\
\multicolumn{2}{p{\textwidth}}{The application runs on Internet Explorer version 10 and higher, Opera version 12.1 and
higher and Safari version 6.0 and higher.} \\
\hline

SCR & must have \\
\multicolumn{2}{p{\textwidth}}{The application runs on devices running on iOS version 6 and higher.} \\
\hline

SCR & should have \\
\multicolumn{2}{p{\textwidth}}{The application runs on devices running on Android version 4.0 and higher.} \\
\hline

SCR & could have \\
\multicolumn{2}{p{\textwidth}}{The application runs on devices running on Windows 8.} \\
\hline

\end{tabular}
\end{center}
\subsection{Maintainability}
\subsection{Reliability}
\subsection{Security}
\subsection{Safety}
\subsection{Documentation}
\subsection{Extensibility}
\begin{center}
\begin{tabular}{ >{\bfseries}p{0.84\textwidth} >{\itshape}p{0.16\textwidth}}
SCR & must have \\
\multicolumn{2}{p{\textwidth}}{The application should be easily extendable with new mixers.} \\
\hline
\end{tabular}
\end{center}
