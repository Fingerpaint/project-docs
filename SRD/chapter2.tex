\chapter{General description}
\todo{short summary of the chapter}

\section{Relation to current projects}
%The context of this project in relation to other current projects
No other current projects are related to FingerPaint.

\section{Relation to predecessor and successor projects}
%The context of this project in relation to past and future projects
\projectname has multiple predecessor projects. These projects resulted in multiple matlab tools that are available on the client's webpage \cite{clientpage}. \projectname will combine some of the functionality of these tools into a mobile web application. \projectname will be developed in such a way that the client can easily extend the application with new mixers. When the development of \projectname is complete, the client will be responsible for maintaining the application. The client may change, add or remove the application's functionality.

\section{Function and purpose}
%A general overview of the function and purpose of the product
\projectname is an application that serves as an education tool for anyone who wants to gain a deeper understanding of the process of mixing in general, and in perticular for students at the TU/e. By `playing' with the application, users can quickly and easily find out what the effects of a certain mixer and mixing protocol on an initial distributian are. This leads to a better understanding of the user about the way this mixer funtions. \projectname may also be used as a quick and convenient way to observe whether a recently thought of mixing protocol renders good or bad mixing results.

\section{Environment description}
%Hardware and operating system of target system and development system
\projectname is a web application that is developed primarily for use on mobile devices. This means the application will mostly be accessed through web browsers on smartphones and tablets. It is expected that the application will mostly be used on iPhones and iPads. Therefore, \projectname must support iOS Safari version 6.0 and above. Furthermore, \projectname should support Firefox versions 20 and above, and Google Chrome versions 26 and above. Lastly, if time permits, \projectname could also support Internet Explorer versions 10 and above, Opera versions 12.1 and above, and Safari versions 6.0 and above. \\
Since it is expected that the application will mostly be used on iPhones and iPads, the \applicationname must run on devices running on iOS versions 6 and higher. Furthermore, since a significant share of smartphones and tablets run on Android, \projectname should run on devices running on Android versions 4.0 and higher. Lastly, if time permits, \projectname could also run on devices running on Windows 8.\\
The hardware used by the users must be able to run at least one of the supported operating systems and browsers.\\
\todo{Moeten we hier nog iets zeggen over de screensize/resolution van de smartphones? Ik kan me voorstellen dat de applicatie niet prettig zal werken op zeer kleine schermen...}

\section{Relation to other systems}
\todo{Is the product an independant system, part of a larger system, replacing another system? The essential characteristics of these other systems}


\section{General counstraints}\
\todo{Reasons why constraints exist: background information and justification(analogous to URD)}

\section{Model description}
\todo{A description of the logical model}