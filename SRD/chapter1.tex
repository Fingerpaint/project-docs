\chapter{Introduction}

\section{Purpose}
The software requirements document provides a translation of all the user requirements listed in section 3 of \todo{URD ref}. Although the URD describes the wishes of the client, the goal of the SRD is to represent the developers\' view of what the \applicationname must be able to do. \\
Note that the software requirements listed in this document are implementation-independent: that is, the requirements describe \emph{what} \projectname must do, but not \emph{how} the requirements will be realized. The requirements are modelled in a logical model, which provides a simplified view of the content and behaviour of the application.

\section{Scope}
\todo{Scope of the software. Identifies the product by name, explains what the software will do}

\section{List of definitions and abbreviations}
\subsection{Definitions}

\begin{description}
\item[Client] Prof.dr.ir. P.D. Anderson.
\item[Firefox] A web browser developed by Mozilla.
\item[Google Chrome] A web browser developed by Google.
\item[Internet Explorer] A web browser developed by Microsoft.
\item[iOS] A mobile operating system developed by Apple.
\item[iOS Safari] A web browser developed by Apple designed for devices running iOS.
\item[iPhone] A line of smartphones developed by Apple.
\item[iPad] A line of tablet computers developed by Apple.
\item[Opera] A web browser developed by Opera Software.
\item[Safari] A web browser developed by Apple.
\end{description}

\subsection{Abbreviations}
\begin{tabular}{l|l}
2IP35 & The Software Engineering Course \\
CM    &Configuration Manager \\
TU/e  &Eindhoven University of Technology \\
SEP   &Software Engineering Project \\
SR    &Software Requirements \\
SRD & Software Requirements Document \\
URD   &User Requirements Document \\
%TBC & To Be Confirmed \\
%TBD & To Be Defined \\
\end{tabular}

\section{List of references}

\bibliographystyle{plain}

\bibliography{../ref}

\section{Overview}

\todo{Short description of the rest of the SRD and how it is organized.}

%The remaining chapters describe the system requirements in more detail. Chapter 2 gives a general description of 
%\begin{itemize}
%\item 2.1 - The relation to other systems,
%\item 2.2 - The main capabilities,
%\item 2.3 - Constraint information and justification,
%\item 2.4 - User charactaristics,
%\item 2.5 - The operational environment, and
%\item 2.6 - Assumptions and dependencies.
%\end{itemize}

%Chapter 3 gives a detailed list of the system's capabilitiy requirements in section 3.1, and a list of the constraint requirements is given in section 3.2.


