\chapter{Introduction}
%\label{chap:intro}
This chapter lists general information about this document.

\section{Purpose}
%\label{sec:purpose}
This document provides a translation of all the user requirements listed in section 3 of the URD \cite{urd}. Although the URD describes the wishes of the client, the goal of the SRD is to represent the developers' view of what the \applicationname\ must be able to do. \\
Note that the software requirements listed in this document are implementation-independent: that is, the requirements describe \emph{what} \projectname\ must do, but not \emph{how} the requirements will be realized. The requirements are modelled in a logical model, which provides a simplified view of the content and behaviour of the application.

\section{Scope}
%\label{sec:scope}
\projectname{} is an application designed and developed by \projectauthor{} for prof. dr. ir. P.D. Anderson. The application provides a cross-platform tool to visualise fluid mixing. Users can define the initial concentration distribution, as well as manipulate the mixing protocol. The resulting fluid distribution can be stored by the user on their device, for later reference.

\section{List of definitions and abbreviations}
%\label{sec:listofdef}

\subsection{Definitions}
%\label{subsec:def}

\begin{description}
%TODO: checken of deze begrippen daadwerkelijk gebruikt worden en aanvullen met nieuwe begrippen.
\item[Client] Prof.dr.ir. P.D. Anderson.
\item[Firefox] A web browser developed by Mozilla.
\item[Google Chrome] A web browser developed by Google.
\item[Internet Explorer] A web browser developed by Microsoft.
\item[iOS] A mobile operating system developed by Apple.
\item[iOS Safari] A web browser developed by Apple designed for devices running iOS.
\item[iPhone] A line of smartphones developed by Apple.
\item[iPad] A line of tablet computers developed by Apple.
\item[Opera] A web browser developed by Opera Software.
\item[Safari] A web browser developed by Apple.
\item[System administrator] A person who is employed to maintain and operate a computer system and/or network. After the SEP project has been completed, this person will be responsible for maintaining the \applicationname{}.
\end{description}

\subsection{Abbreviations}
%\label{subsec:abbrev}
%TODO: checken of al deze afkortingen daadwerkelijk gebruikt worden en aanvullen met nieuwe afkortingen.
\begin{tabular}{l|l}
2IP35 & The Software Engineering Project \\
CM    & Configuration Manager \\
GUI & Graphical User Interface \\
SEP   & Software Engineering Project \\
SR    & Software Requirements \\
SRD   & Software Requirements Document \\
TU/e  & Eindhoven University of Technology \\
URD   & User Requirements Document \\
\end{tabular}

\section{List of references}
\bibliography{../ref}

\section{Overview}
%\label{sec:overview}
The remainder of this document describe the software requirements in more detail. Chapter \ref{chap:gendesc} gives a general description of:
\begin{itemize}
\item relation to current projects (\ref{sec:curproj});
\item relation to predecessor and successor projects (\ref{sec:predsuc});
\item function and purpose (\ref{sec:functpurp});
\item environment (\ref{sec:env});
\item relation to other systems (\ref{sec:othersys});
\item general constraints (\ref{sec:genconst});
\item model description (\ref{sec:moddesc}).
\end{itemize}
Chapter \ref{chap:specreq} gives a detailed description of the functional requirements of the system in \ref{sec:funcreq} and a list of non-functional requirements is given in \ref{sec:nonfuncreq}. The requirements traceability matrix is described in chapter \ref{chap:reqtracematrix}.
